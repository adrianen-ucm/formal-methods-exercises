\documentclass{article}
\usepackage[utf8]{inputenc}
\usepackage[spanish]{babel}
\usepackage{graphicx}
\usepackage{mathtools}
\usepackage{amsfonts}

\title{%
	Teleportación de estados cuánticos \\ 
  \large La transmisión no es superlumínica
}
\author{
	Daniela Ferreiro \and 
	Adrián Enríquez \and 
	Daniel Trujillo
}
\date{\today}

\begin{document}
\maketitle

Después de que Alice mida sus qubits, el de Bob puede estar en cada
uno de los siguientes estados con misma probabilidad: 

\begin{align*}
  &\alpha|0\rangle + \beta|1\rangle \\ 
  &\alpha|1\rangle + \beta|0\rangle \\ 
  &\alpha|0\rangle - \beta|1\rangle \\ 
  &\alpha|1\rangle - \beta|0\rangle \\
\end{align*}

donde $\alpha|0\rangle + \beta|1\rangle$ es el estado del qubit que
se pretende teleportar a Bob.

Supongamos que Bob no espera a recibir el resultado de las medidas,
que no puede viajar más rápido que la velocidad de la luz, y supone
al azar que ha salido cierta medida. Parece que, como con
$\frac{1}{4}$ de probabilidad podría terminar correctamente el
protocolo, en este punto se puede haber transmitido información
a velocidad superlumínica.

Sin embargo, Bob no tiene ninguna manera de obtener información
sobre el estado teleportado a través de su estado actual. Para
formalizar este resultado necesitaríamos algunos conceptos que no
hemos visto en el curso, pero vamos a ver un ejemplo ilustrativo
y simple.

Si intentase obtener información midiendo su estado con la base
canónica, teniendo en cuanta cada una de las cuatro posibles
situaciones, la probabilidad de que saliese $0$ sería 

$$
\frac{1}{4}|\alpha|^2
  + \frac{1}{4}|\beta|^2
  + \frac{1}{4}|\alpha|^2
  + \frac{1}{4}|\beta|^2
= \frac{1}{2}(|\alpha|^2 + |\beta|^2)
= \frac{1}{2}
$$

mientras que la probabilidad de que saliese $1$ sería 

$$
\frac{1}{4}|\beta|^2
  + \frac{1}{4}|\alpha|^2
  + \frac{1}{4}|\beta|^2
  + \frac{1}{4}|\alpha|^2
= \frac{1}{2}(|\beta|^2 + |\alpha|^2)
= \frac{1}{2}
$$

Esto sucede de manera independie a los valores de $\alpha$
y $\beta$, por lo que su intento de obtener información no tendría
éxito fuese cual fuese el estado teleportado.

\end{document}
