\documentclass{article}
\usepackage[english]{babel}
\usepackage{amsmath}

\title{Transfer nets exercise}
\author{Adrián Enríquez Ballester}

\begin{document}
\maketitle

A transfer net is a a Petri net extended with a set $Tr$
of transfer arcs of the form $(p, t, q)$. In a transfer
net, the enabling condition is exactly the same as in plain 
Petri Nets. However, the firing rule of transitions is changed 
so that when a transition $t$ is fired:

\begin{itemize}
  \item First, tokens are removed from the preconditions of $t$.
  \item Then, if there is a transfer arc $(p, t, q) \in Tr$, then 
    all the tokens in $p$ are transferred to $q$.
  \item Finally, tokens are added to the postconditions of $t$.
\end{itemize}

For the sake of simplicity, we assume that for each transition 
$t \in T$ there is at most one transfer arc in $t$ (i.e. one arc 
of the form $(p, t, q) \in Tr$).

\section{Formal definition}

A \textit{net with transfer arcs} $N=(S, T, F, Tr)$ consists 
of two disjoint sets of places and transitions $S$ and $T$, a
set $F \subseteq (S \times T) \cup (T \times S)$ of arcs,
and a set $Tr \subseteq S \times T \times S$ of transfer arcs 
where, if $(p, t, q) \in Tr$ is a transfer arc then:

\begin{enumerate}
  \item $(p, t) \notin F$ and $(t, q) \notin F$ (i.e. transfer 
    arcs do not overlap with regular arcs).
  \item If $(p', t, q') \in Tr$, then $p' = p$ and $q' = q$ 
    (i.e. one single transfer arc per transition at most).
\end{enumerate}

A transition $t \in T$ is enabled at marking $M$ of $N$ if 
$M(s) > 0\;\forall s \in {}^\bullet t$. If $t$ is 
enabled, then it can $fire$ leading to a marking $M'$ defined
as:

$$
M'(s) = 
  \begin{cases}
    M(s) - 1 &\text{if } s \in {}^\bullet t \setminus t^\bullet \\
    M(s) + 1 &\text{if } s \in t^\bullet \setminus {}^\bullet t \\
    0        &\text{if } \exists s' \text{ s.t } (s, t, s') \in Tr \land s \neq s' \\
    M(s')    &\text{if } \exists s' \text{ s.t } (s', t, s) \in Tr \land s \neq s' \\
    M(s)     &\text{otherwise}
  \end{cases}
$$

\section{Informal discussion about its monotonicity}

This Petri net variant seems to satisfy the strict version of 
monotonicity (i.e. if $M_1 \stackrel{t}{\rightarrow} M_2$ for a 
transition $t$ and some markings $M_1$ and $M_2$ of the net, then
$t$ is enabled for any marking $M'_1 > M_1$ and, when fired, results 
in a  marking $M'_2 > M_2$).

Petri nets with reset arcs fail to satisfy this monotonicity version 
because we could add tokens to a reset precondition of a transition 
and, when fired, the resulting marking would not be modified.

When it is a transfer arc instead of a reset one, these tokens would 
be moved to another place instead of being completely removed. It 
allows to satisfy $M'_2 > M_2$ but, as the added tokens 
to $M_1$ do not need to be kept there after the firing, the strongest version 
of monotonicity is not satisfied.
\end{document}
