\documentclass{article}
\usepackage[english]{babel}
\usepackage{amsfonts}
\usepackage{amsmath}
\usepackage{mathtools}
\usepackage{mdframed}
\usepackage{multirow}

\renewcommand{\thesubsection}{\thesection.\alph{subsection}}

%%ARROWS%%
\newcommand{\verylongrightarrow}[1]     %longleft and rightgoing arrows
    {\setlength{\unitlength}{.01in}     %for protocols
     \begin{picture}(#1,1) \put(0,0){\thicklines\vector(1,0){#1}} \end{picture}}

\newcommand{\rightgoing}[2]{{\stackrel{{\displaystyle #1}} {\verylongrightarrow{#2}}}}

\newcommand{\verylongleftarrow}[1]
    {\setlength{\unitlength}{.01in}
         \begin{picture}(#1,1) \put(#1,0){\thicklines\vector(-1,0){#1}} \end{picture}}

\newcommand{\leftgoing}[2]{{\stackrel{{\displaystyle #1}} {\verylongleftarrow{#2}}}}

\newcommand{\longleft}[1]{\leftgoing{#1}{400} }
\newcommand{\longright}[1]{\rightgoing{#1}{400} }
\newcommand{\shortleft}[1]{\leftgoing{#1}{100} }
\newcommand{\shortright}[1]{\rightgoing{#1}{100} }
\newcommand{\shortleftright}[1]{\leftrightgoing{#1}{200} }
\newcommand{\verylongleft}[1]{\leftgoing{#1}{500} }
\newcommand{\verylongright}[1]{\rightgoing{#1}{500} }
\def\sep{\,,\,} %used with arrow figures

\newmdenv[
  skipabove=\topsep,
  skipbelow=\topsep
]{scheme}

\title{Cryptographic Protocols, day 4 exercises}
\author{Adrián Enríquez Ballester}

\begin{document}
\maketitle

\section{Perfect security against key recovery}

A Shannon cipher $\mathcal{E} = (E, D)$ defined over $(\mathcal{K},
\mathcal{M}, \mathcal{C})$ is perfectly secure against key recovery 
if $\forall k_0, k_1 \in \mathcal{K}$ and 
$\forall c \in \mathcal{C}$

$$ Pr[E(k_0, m) = c] = Pr[E(k_1, m) = c] $$

where the probability runs over the choice of $m$, which is chosen 
uniformly at random in $\mathcal{M}$.

\subsection{Perfectly secure against key recovery does not imply perfectly secure}

Consider the identity cipher $\mathcal{E} = (E, D)$ where 
$E(k, m) = m$ and $D(k, c) = c$, which is trivially a Shannon 
cipher because $\forall k \in \mathcal{K}$ and 
$\forall m \in \mathcal{M}$ we have

$$ D(k, E(k, m)) = D(k, m) = m $$

This cipher is not perfectly secure in also a trivial way because 
it leaks everything about the message:

$$
  Pr[E(k, m) = c] = 
  \begin{cases}
    1, & \text{if } m = c \\
    0, & \text{otherwise}
  \end{cases} 
$$

$\forall m \in \mathcal{M}$ and $\forall c \in \mathcal{C}$, where 
the probability runs uniformly over the choice of $k$ 
in $\mathcal{K}$. 

At the same time, it is perfectly 
secure against key recovery because it does not use the provided
key at all:

\[ Pr[E(k, m) = c] = 1 / \left|\mathcal{M}\right| \]

$\forall k \in \mathcal{K}$ and $\forall c \in \mathcal{C}$, being
the probabilistic experiment at $m$, distributed uniformly in 
$\mathcal{M}$.

\subsection{Perfectly secure does not imply perfectly secure against key recovery}

Consider this time a variant of the one-time-pad of size 
$\ell$ where the message space $\mathcal{M}$ does not contain 
the $0^\ell$ element.

On the one hand, it still being perfectly secure because, given 
any message $m \in \mathcal{M}$ and any ciphertext 
$c \in \mathcal{C}$, there exists one single key 
$k \in \mathcal{K}$ such that $E(k, m) = c$, namely

$$k = m \oplus c$$

On the other hand, this cipher is not perfectly secure against 
key recovery because, once anyone observes a ciphertext, 
it is clear that the key which is the same as the ciphertext 
has not been used:

$$
  Pr[E(k, m) = c] = 
  \begin{cases}
    0, & \text{if } k = c \\
    1 / \left|\mathcal{M}\right|, & \text{otherwise}
  \end{cases} 
$$

$\forall k \in \mathcal{K}$ and $\forall c \in \mathcal{C}$, with
the probabilistic experiment at $m$ distributed uniformly in 
$\mathcal{M}$.

This suggests that maybe an analogous version of 
the Shannon theorem for perfect security against key recovery 
could be stated when the message space is smaller than the 
key space.

\subsection{Both perfectly secure and perfectly secure against key recovery are
compatible}

We know that the one-time-pad cipher is perfectly secure. 
This means that $\forall m_0, m_1 \in \mathcal{M}$ and 
$\forall c \in \mathcal{C}$

$$ Pr[E(k, m_0) = c] = Pr[E(k, m_1) = c] $$

where the probability runs over the choice of $k$, which is chosen 
uniformly at random in $\mathcal{K}$.

Given that the spaces $\mathcal{M}$ and $\mathcal{K}$ are the same
and that the $E$ function is symmetric due to the $xor$ 
commutativity, we can transform the probabilistic experiment 
above to run over the message space $\mathcal{M}$ given two keys 
by swapping the function parameters while preserving the equality. 

This turns out to be the condition for perfect security against 
key recovery also satisfied for the one-time-pad.

\section{A wrong attempt of semantically secure RSA scheme}

Consider the following public key encryption scheme 
$\mathcal{E} = (G, E, D)$:

\begin{scheme}
\begin{itemize}
  \item $G(\cdot)$: Same as RSA, namely given a (public) odd 
    integer $e$, and a parameter size $\ell$, generate $p, q$
    prime numbers of $\ell$ bits such that $gcd(e, p-1) = 1$, 
    $gcd(e, q-1) = 1$.

    Compute $N=p \cdot q$, $\varphi (N) = (p - 1) \cdot (q - 1)$
    and $d = e^{-1}$ mod $\varphi (N)$.

    Output $pk = (N, e)$ as public key and $sk = (N, d)$ as 
    private key.

    The message space is $\mathcal{M} = \mathbb{Z}_N^*$ 
    and the ciphertext space is 
    $\mathcal{C} = (\mathbb{Z}_N^*)^2$.
  \item $E(pk, m)$: Choose a random $x \in \mathbb{Z}_N^*$,
    define

    $$c_1 = x^e \text{ mod } N$$

    and 

    $$c_2 = x \cdot m \text{ mod } N$$

    and output the ciphertext $c = (c_1, c_2)$.
  \item $D(sk, c)$: Compute 
    $\widetilde{m} = c_2/c_1^d \text{ mod } N$ 
    and output $\widetilde{m}$.
\end{itemize}
\end{scheme}

\subsection{It is a valid public key encryption scheme}

Let $m \in \mathcal{M}$ be a message and $pk = (N, e)$, 
$sk = (N, d)$ the corresponding public and secret keys 
generated by $G$.

When performing an encryption for $m$ as $E(pk, m)$, let 
$x$ be the randomly chosen element from $\mathbb{Z}_N^*$, so
the resulting ciphertext is

$$c = (x^e \text{ mod } N, x \cdot m \text{ mod } N)$$

Let us check that we can recover the original $m$:

$$
  D(sk, c) = 
  \frac{ x \cdot m}
    {\left( x^e \right)^d} \text{ mod } N =
$$

As 
$(x^a)^b = x^{a \cdot b \text{ mod } \varphi(N)} \text{ mod } N$,
and $d = e^{-1} \text{ mod } \varphi(N)$, the denominator
reduces into $x$:

$$
  \left( x^e \right)^d \text{ mod } N =
    x^{e \cdot e^{-1} \text{ mod } \varphi(N)} \text{ mod } N =
      x
$$

so the whole expression becomes

$$
  D(sk, c) = \frac{x \cdot m}{x} \text{ mod } N = m
$$

and this means that

$$Pr[D(sk, E(pk, m)) = m] = 1$$

when $m$ is randomly chosen from $\mathcal{M}$, because it 
already holds $\forall m \in \mathcal{M}$.

\subsection{It is not semantically secure}

Consider an adversary $\mathcal{A}$ for the semantically secure 
attack game which receives $pk = (N, e)$ from the challenger, 
computes two different messages $m_0, m_1$ and sends them 
to the challenger.

The challenger replies with a ciphertext $c = (c_1, c_2)$ and
$\mathcal{A}$ computes 

$$x_i = c_2 / m_i \text{ mod } N$$

for each $i \in \{0,1\}$.

By checking which one satisfies 
$x_i^e \text{ mod } N = c_1$, it can be distinguished which
one of $m_0, m_1$ has lead to the received ciphertext.

Note that, under this conditions, it is not possible to have
two different $x_0, x_1 \in \mathbb{Z}_N^*$ such that
$x_0^e = x_1^e$. Otherwise, it would contradict 
$(x^e)^d = x \text{ mod } N$ $\forall x \in \mathbb{Z}_N^*$
, which is necessary for this scheme to work, so

$$SSAdv\left[\mathcal{A}, \mathcal{E}\right] = 1$$

$\mathcal{A}$ would be an efficient adversary (i.e. it performs
the kind of operations that the challenger is also performing 
in an efficient way) with a far from negligible advantage in the game, 
so this scheme is not semantically secure.

\section{A DDH-based public key encryption scheme for bits}

Let $\mathbb{G}$ be a group of order $q$ and $g$ a generator
of the group. Consider the following public key encryption
scheme over message and ciphertext spaces 
$\mathcal{M} = \{0, 1\}$ and $\mathcal{C} = \mathbb{G}^2$.

\begin{scheme}
\begin{itemize}
  \item $G(\cdot)$: Chooses $\alpha$ at random in 
    $\mathbb{Z}_q$ and defines $u = g^\alpha$.
    The public key is $pk = u$ and the secret key is
    $sk = \alpha$.
  \item $E(pk, m)$: First, it chooses a random 
    $\beta \in \mathbb{Z}_q$ and defines 
    $c_1 = g^\beta$. Now:

    \begin{itemize}
        \item If $b = 0$, it defines $c_2 = u^\beta$.
        \item If $b = 1$, it takes a random 
          $\gamma \in \mathbb{Z}_q$, and defines 
          $c_2 = g^\gamma$.
    \end{itemize}

    The ciphertext is $c = (c_1, c_2)$.
  \item $D(sk, c)$: ?
\end{itemize}
\end{scheme}

\subsection{The decryption algorithm}

Consider the decryption algorithm to be defined as follows:

$$D(\alpha, (c_1, c_2)) = 
  \begin{cases}
    0, & \text{if } c_1^\alpha = c_2 \\ 
    1, & \text{otherwise}
  \end{cases}
$$

The criteria for assuming when the encryption algorithm 
has taken the branch corresponding to $0$ is based on 
the fact that

$$
  c_1^\alpha = 
    (g^\beta)^\alpha = (g^\alpha)^\beta = u^\beta = c_2
$$

which will be the case when $E$ is ciphering the bit $0$ and
not when ciphering the bit $1$ unless, due to 
$\mathbb{G}=\left<g\right>$ 
being a group of order $q$, the randomly chosen 
$\gamma \in \mathbb{G}$ has been exactly the element 
$\alpha \cdot \beta$. This leads to the decryption
algorithm to output $0$ when the original message was $1$.

There is one element out of $q$ that produces this failure
and, as the choice is uniformly distributed, it will appear with
probability $1 / q$. This means that

$$Pr[D(sk, E(pk, b)) = b] = 1 - 1/q$$

which can be accepted by weakening the definition of public 
key encryption scheme and allowing it to fail with negligible
probability.

\subsection{The scheme is CPA-secure}

Given an adversary $\mathcal{A}$ of the SS game, we can define 
an adversary $\mathcal{B}$ for the DDH game which plays also the
role of challenger for $\mathcal{A}$ in the former game:

\begin{itemize}
  \item The DDH challenger chooses $\alpha$ and $\beta$ at random
    from $\mathbb{Z}_q$. It also computes $u = g^\alpha$ and 
    $v = g^\beta$.

  \item If it is playing the DDH experiment $b=0$, it defines
    $w=g^{\alpha\beta}$, otherwise it chooses a random
    $\gamma$ from $\mathbb{Z}_q$ and defines $w=g^\gamma$.

  \item The DDH challenger sends the triple $(u, v, w)$
    to $\mathcal{B}$.

  \item $\mathcal{B}$ forwards $u$ to $\mathcal{A}$ as the public key.

  \item $\mathcal{A}$ computes two messages $m_0, m_1$ and sends
      them to $\mathcal{B}$.

  \item $\mathcal{B}$ defines $c=E(u, m_{b'})$ and replies to $\mathcal{A}$
        with it, being $b' \in \{0,1\}$ the SS experiment being played. The 
        encryption with $E$ is done assuming $v$ as $g^\beta$ and $w$ as
        $u^\beta$.

  \item $\mathcal{A}$ outputs $\hat{b'}$ as a guess for the SS experiment
    being played.
  \item $\mathcal{B}$ outputs $\hat{b}$ as a guess for the DDH experiment
    being played, where 

    $$\hat{b} = \begin{cases} 
      0, & \text{if } b' = \hat{b'} \\
      1, & \text{otherwise}
    \end{cases} $$
\end{itemize}

The explained game can be summarized with the following diagram:

\[
\begin{array}{ccccc}
{\rm DDH \ Challenger} & 
  & \mathcal{B} & 
  & \mathcal{A}\\
\multirow{2}{*}
{\framebox(80,250){}}
&
&
\multirow{2}{*}
{\framebox(80,250){}}
&
& 
\multirow{2}{*}
{\framebox(80,250){}
}\\
~
  {\rm Exp.} \ b \in \{0,1\}
&
& {\rm Exp. } \ b' \in \{0,1\}
&
&
\\[10pt]
  \alpha \stackrel{R}{\leftarrow} \mathbb{Z}_q, 
  \beta \stackrel{R}{\leftarrow} \mathbb{Z}_q
&
&
&
&
\\ 
~
u \coloneqq g^\alpha, v \coloneqq g^\beta
& 
&
&
&
\\[10pt]
~
  {\rm If} \ b = 0
& 
&
&
&
\\
~ 
w \coloneqq g^{\alpha\beta}
&
&
&
& 
\\
~ 
{\rm otherwise}
&
&
&
& 
\\
~ 
\gamma \stackrel{R}{\leftarrow} \mathbb{Z}_q, 
w \coloneqq g^\gamma
&
&
&
& 
\\[10pt]
~ 
& \rightgoing{(u, v, w)}{40}
& pk \coloneqq u
& \rightgoing{pk}{40}
& 
\\[10pt]
~
&
& 
&\leftgoing{(m_{0},m_{1})}{40}
&
\\
~
&
&
&\rightgoing{c}{40}
&
\\[10pt]
~
&
& {\rm If}\ b'=\hat{b'} 
&
&
\\
~
&
& \hat{b} \coloneqq 0
&
&
\\
~
&
& {\rm otherwise}
&
&
\\
~
&
& \hat{b} \coloneqq 1
&
&
\\
\\
&
&
\downarrow
&
&
\downarrow
\\
&
&
\hat{b}
&
&
\hat{b'}
\\
\end{array}
\]

On the one hand, being $W_1^{DDH}$ the event of $\mathcal{B}$ 
outputting $1$ at the DDH experiment $1$, the ciphered messages 
sent to $\mathcal{A}$ are completely random because $w$ itself is 
random, so we expect that its guess is also random, and also 
the guess of $\mathcal{B}$:

$$Pr\left[W_1^{DDH}\right] = 1 / 2$$

On the other hand, with $W_0^{DDH}$ being the event of 
$\mathcal{B}$ outputting $1$ at the DDH experiment $0$,
we can relate its probability with the SS experiments 
as follows:

\begin{multline*}
$$
  Pr\left[W_0^{DDH}\right] = 
  Pr\left[ \hat{b} = 1 | b = 0 \right] =
  Pr\left[ b' \neq \hat{b'} | b = 0 \right] = \\ 
  Pr\left[ b' = 0 \right] Pr \left[ \hat{b'} = 1 | b' = 0 \right] +
    Pr\left[ b' = 1 \right] Pr \left[ \hat{b'} = 0 | b' = 1 \right] = \\ 
  1/2 \left(1 -
    Pr \left[ \hat{b'} = 1 | b' = 1 \right] +
    Pr \left[ \hat{b'} = 1 | b' = 0 \right]
  \right)
$$
\end{multline*}

As $W_i^{SS}$ is the event of $\mathcal{A}$
outputting $1$ in the SS experiment $i$ for $i \in \{0,1\}$,
we have 

$$
  Pr\left[W_0^{DDH}\right] =
  1/2 - 1/2 \left( Pr\left[W_1^{SS} \right] 
    - Pr\left[W_0^{SS}\right]\right)
$$

and the adversaries corresponding advantages can be related as follows:

\begin{multline*}
$$
  DDHAdv\left[\mathcal{B}, \mathcal{E}\right] =
    \left|Pr\left[W_1^{DDH}\right] - Pr\left[W_0^{DDH}\right]\right| \\ =
    1\slash2 \left|Pr\left[W_1^{SS}\right] - Pr\left[W_0^{SS}\right]\right| =
    1\slash2 \cdot SSAdv\left[\mathcal{A}, \mathcal{E}\right]
$$
\end{multline*}

If the decissional Diffie-Hellman assumption holds for 
$\mathbb{G}$, then 
$DDHAdv\left[\mathcal{B}, \mathcal{E}\right]$ is negligible 
and, due to the relation between both advantages, 
$SSAdv\left[\mathcal{A}, \mathcal{E}\right]$ is also negligible,
thus $\mathcal{E}$ is SS-secure.

Finally, we know by a theorem that if a public-key encryption scheme is
semantically secure, then it is also CPA secure.
\end{document}
