\documentclass{article}
\usepackage[utf8]{inputenc}
\usepackage[spanish]{babel}
\usepackage[shortlabels]{enumitem}
\usepackage{mathtools}
\usepackage{amsfonts}
\usepackage{bbold}

\title{Ejercicios de criptografía cuántica}
\author{Adrián Enríquez Ballester}
\date{\today}

\begin{document}
\maketitle

\subsection*{Ejercicio 1 - Corrección de errores clásica}

Sea $C \subset \mathbb{Z}_2^6$ el código lineal $[6, 3]$ con
matriz generadora 

$$
G = \begin{pmatrix}
  0 & 1 & 0 & 1 & 1 & 0 \\ 
  1 & 1 & 0 & 1 & 0 & 1 \\
  1 & 0 & 1 & 1 & 1 & 0 \\
\end{pmatrix}
$$

Calcular:

\begin{enumerate}[a)]
  \item Todos los elementos de $C$.
  \item Una matriz de check $H$ para $C$.
  \item La distancia mínima de $C$, el número de errores que permite
    detectar, y el número de errores que permite corregir.
\end{enumerate}

Además, corregir los siguientes mensajes:

\begin{enumerate}[I)]
  \item $\begin{pmatrix} 1 & 0 & 0 & 0 & 1 & 0 \end{pmatrix}$
  \item $\begin{pmatrix} 0 & 0 & 1 & 0 & 1 & 1 \end{pmatrix}$
  \item $\begin{pmatrix} 1 & 1 & 1 & 1 & 0 & 1 \end{pmatrix}$
\end{enumerate}

\subsubsection*{Respuesta}

Como $C$ es la imagen de la aplicación lineal dada por la matriz
$G^\top$, sea $(\lambda_1, \lambda_2, \lambda_3) \in
\mathbb{Z}_2^3$:

\begin{align*}
  \begin{pmatrix} 
    0 & 1 & 1 \\
    1 & 1 & 0 \\
    0 & 0 & 1 \\
    1 & 1 & 1 \\
    1 & 0 & 1 \\
    0 & 1 & 0 \\
  \end{pmatrix}
  \begin{pmatrix} 
    \lambda_1 \\  
    \lambda_2 \\ 
    \lambda_3 \\ 
  \end{pmatrix}
  = &\lambda_1 \begin{pmatrix}
    0 & 1 & 0 & 1 & 1 & 0 
  \end{pmatrix}^\top \\
  + &\lambda_2 \begin{pmatrix} 
    1 & 1 & 0 & 1 & 0 & 1 
  \end{pmatrix}^\top \\
  + &\lambda_3 \begin{pmatrix} 
    1 & 0 & 1 & 1 & 1 & 0 
  \end{pmatrix}^\top
\end{align*}

es decir, $C$ es el subespacio lineal de $\mathbb{Z}_2^6$ generado
por la base dada por los vectores fila de $G$, que tiene $2^3$
elementos:

\begin{align*}
  C = \{\;
    &\lambda_1 \begin{pmatrix} 
      0 & 1 & 0 & 1 & 1 & 0 
    \end{pmatrix} \\
    + &\lambda_2 \begin{pmatrix}
      1 & 1 & 0 & 1 & 0 & 1 
    \end{pmatrix} \\ 
    + &\lambda_3 \begin{pmatrix} 
      1 & 0 & 1 & 1 & 1 & 0 
    \end{pmatrix} \\
    :\;&\forall (\lambda_1, \lambda_2, \lambda_3) 
      \in \mathbb{Z}_2^3
  \;\}
\end{align*}

Con tal de obtener una matriz de check $H$ para $C$, podemos
realizar las siguientes transformaciones elementales partiendo de
$C$:

\begin{align*}
\begin{pmatrix}
  0 & 1 & 0 & 1 & 1 & 0 \\ 
  1 & 1 & 0 & 1 & 0 & 1 \\
  1 & 0 & 1 & 1 & 1 & 0 \\
\end{pmatrix}
&\rightsquigarrow
\begin{pmatrix}
  0 & 1 & 0 & 1 & 1 & 0 \\ 
  1 & 0 & 0 & 0 & 1 & 1 \\
  1 & 0 & 1 & 1 & 1 & 0 \\
\end{pmatrix} \\
&\rightsquigarrow
\begin{pmatrix}
  0 & 1 & 0 & 1 & 1 & 0 \\ 
  1 & 0 & 0 & 0 & 1 & 1 \\
  0 & 0 & 1 & 1 & 0 & 1 \\
\end{pmatrix} \\
&\rightsquigarrow
\begin{pmatrix}
  1 & 0 & 0 & 0 & 1 & 1 \\
  0 & 1 & 0 & 1 & 1 & 0 \\ 
  0 & 0 & 1 & 1 & 0 & 1 \\
\end{pmatrix}
\end{align*}

Al tener una diagonal en la mitad izquierda, sabemos que la
matriz que resulta de transponer la mitad derecha seguida de una
diagonal 

$$
H = \begin{pmatrix}
  0 & 1 & 1 & 1 & 0 & 0 \\
  1 & 1 & 0 & 0 & 1 & 0 \\ 
  1 & 0 & 1 & 0 & 0 & 1 \\
\end{pmatrix}
$$

determina una aplicación lineal en la que $C = \mathsf{Ker}(H)$ y,
por tanto, $H$ es una matriz de check para $C$.

En este caso, al tratarse de un código lineal, podemos obtener la
distancia mínima de $C$ como el número mínimo de columnas
linealmente dependientes de $H$, que es $d = 3$. Esto quiere decir
que el código puede detectar hasta $d - 1 = 2$ errores y corregir
$\frac{d - 1}{2} = 1$.

Vamos a aplicar los resultados anteriores a los mensajes propuestos.
En cuanto al primero, obtenemos el siguiente síndrome:

\begin{align*}
  H
  \begin{pmatrix} 
    1 \\
    0 \\
    0 \\
    0 \\
    1 \\
    0 \\
  \end{pmatrix}
  = \begin{pmatrix} 
    0 \\ 
    0 \\
    1 \\
  \end{pmatrix}
\end{align*}

Al no tratarse del vector nulo, esto quiere decir que se ha
producido algún error, y como 

\begin{align*}
  H
  \begin{pmatrix} 
    0 \\
    0 \\
    0 \\
    0 \\
    0 \\
    1 \\
  \end{pmatrix}
  = \begin{pmatrix} 
    0 \\ 
    0 \\
    1 \\
  \end{pmatrix}
\end{align*}

el mensaje corregido sería $\begin{pmatrix}
1 & 0 & 0 & 0 & 1 & 1 \end{pmatrix}$.

En cuanto al segundo mensaje, su síndrome es

\begin{align*}
  H
  \begin{pmatrix} 
    0 \\
    0 \\
    1 \\
    0 \\
    1 \\
    1 \\
  \end{pmatrix}
  = \begin{pmatrix} 
    1 \\ 
    1 \\
    0 \\
  \end{pmatrix}
\end{align*}

que de nuevo es no nulo. Como 

\begin{align*}
  H
  \begin{pmatrix} 
    0 \\
    1 \\
    0 \\
    0 \\
    0 \\
    0 \\
  \end{pmatrix}
  = \begin{pmatrix} 
    1 \\ 
    1 \\
    0 \\
  \end{pmatrix}
\end{align*}

el mensaje corregido sería $\begin{pmatrix}
0 & 1 & 1 & 0 & 1 & 1 \end{pmatrix}$.

En cuanto al último mensaje, su síndrome es 

\begin{align*}
  H
  \begin{pmatrix} 
    1 \\
    1 \\
    1 \\
    1 \\
    0 \\
    1 \\
  \end{pmatrix}
  = \begin{pmatrix} 
    1 \\ 
    0 \\
    1 \\
  \end{pmatrix}
\end{align*}

y como

\begin{align*}
  H
  \begin{pmatrix} 
    0 \\
    0 \\
    1 \\
    0 \\
    0 \\
    0 \\
  \end{pmatrix}
  = \begin{pmatrix} 
    1 \\ 
    0 \\
    1 \\
  \end{pmatrix}
\end{align*}

el mensaje corregido sería $\begin{pmatrix}
1 & 1 & 0 & 1 & 0 & 1 \end{pmatrix}$.

\subsection*{Ejercicio 2 - Corrección de errores cuántica}

El código de repetición \textit{bitflip} está definido por 

\begin{align*}
  |0\rangle_L &= |000\rangle \\ 
  |1\rangle_L &= |111\rangle \\
\end{align*}

En vez de los operadores de medida del síndrome que hemos visto en
clase, supongamos que al recibir un mensaje codificado, cada qubit
se mide en la base canónica ($|0\rangle$ y $|1\rangle$).

\begin{enumerate}[a)]
  \item Calcular los 8 operadores (proyecciones) que definen la
    medida.
  \item Explicar cómo detectar la posición de un error a partir del
    síndrome.
  \item Demostrar que la corrección del error solo funciona si el
    estado codificado era en la base canónica.
  \item Asumiendo que el canal es perfecto y que no se producen
    errores, calcular la fidelidad mínima al usar este código de
    corrección de errores.
\end{enumerate}

\subsubsection*{Respuesta}

Como cada qubit se mide en la base canónica, utilizamos los
operadores de medida asociados a la proyección ortogonal de
$(\mathbb{C}^2)^{\otimes 3}$:

\begin{align*}
  P_0 &= |000 \rangle\langle 000| 
  \;\;\;\;\;\;\; 
  (\mathbb{1} \otimes \mathbb{1} \otimes \mathbb{1}) \\
  P_1 &= |001 \rangle\langle 001| 
  \;\;\;\;\;\;\; 
  (\mathbb{1} \otimes \mathbb{1} \otimes X) \\
  P_2 &= |010 \rangle\langle 010| 
  \;\;\;\;\;\;\; 
  (\mathbb{1} \otimes X \otimes \mathbb{1}) \\
  P_3 &= |011 \rangle\langle 011| 
  \;\;\;\;\;\;\;
  (X \otimes \mathbb{1} \otimes \mathbb{1}) \\
  P_4 &= |100 \rangle\langle 100|
  \;\;\;\;\;\;\; 
  (X \otimes \mathbb{1} \otimes \mathbb{1}) \\
  P_5 &= |101 \rangle\langle 101| 
  \;\;\;\;\;\;\; 
  (\mathbb{1} \otimes X \otimes \mathbb{1}) \\
  P_6 &= |110 \rangle\langle 110| 
  \;\;\;\;\;\;\; 
  (\mathbb{1} \otimes \mathbb{1} \otimes X) \\
  P_7 &= |111 \rangle\langle 111| 
  \;\;\;\;\;\;\; 
  (\mathbb{1} \otimes \mathbb{1} \otimes \mathbb{1}) \\
\end{align*}

los cuales sabemos que definen una medida proyectiva válida. 

Al lado de cada operador está indicado el error más probable que se
habría producido en caso de que saliese su medida asociada y, por
tanto, la corrección que se tendría que aplicar para corregirlo
(e.g.  $\mathbb{1} \otimes \mathbb{1} \otimes X$ quiere decir que se
ha producido un \textit{bitflip} en el tercer qubit).

Esta medida tiene un problema con respecto a la que hemos visto en
clase y es que, al estar midiendo todos los qubits, el hecho de
medir provoca un cambio en el estado que hace imposible corregir el
error para ciertos mensajes.

Supongamos que se codifica un mensaje $|\varphi\rangle = a|0\rangle
+ b|1\rangle$ como $|\tilde{\varphi}\rangle = a|000\rangle
+ b|111\rangle$ y, tras ser enviado y posiblemente alterado, se
recibe $|\tilde{\varphi}'\rangle$. En el caso de que no se hubiese
producido ningún error, $|\varphi\rangle = |\tilde{\varphi}\rangle$
y las probabilidades de cada medida serían 

\begin{align*}
  p(0) &= \langle\varphi|P_0|\varphi\rangle 
      = |a|^2\langle 000|000\rangle\langle 000|000\rangle 
      + |b|^2\langle 111|000\rangle\langle 000|111\rangle 
      = |a|^2 \\
  p(7) &= \langle\varphi|P_7|\varphi\rangle 
      = |a|^2\langle 000|111\rangle\langle 111|000\rangle 
      + |b|^2\langle 111|111\rangle\langle 111|111\rangle 
      = |b|^2 \\
\end{align*}

y $0$ para el resto, puesto que $|a|^2 + |b|^2 = 1$.

En caso de que saliese $0$, el estado cambiaría a 

$$
\frac{P_0|\varphi\rangle}{\sqrt{|a|^2}} 
  = \frac{
    a|000\rangle\langle 000|000\rangle + 
    b|000\rangle\langle 000|111\rangle}{|a|}
  = \frac{a}{|a|}|000\rangle
$$

y, en caso de que saliese $1$, con un cálculo análogo se obtendría
$\frac{b}{|b|}|111\rangle$.

Esto quiere decir que medir habría destruido la información y ya no
admitiría la corrección que se pretende. Por otra parte, si $a = 0$
o $b = 0$, solo una de las medidas ocurriría con probabilidad 1 y el
estado sería el mismo tras realizar la medida.

Con un razonamiento similar, los cálculos para un error en el primer
qubit serían los siguientes:

$$
|\tilde{\varphi}\rangle 
  = (X \otimes \mathbb{1} \otimes \mathbb{1})|\varphi\rangle
  = a|100\rangle + b|011\rangle
$$

\begin{align*}
  p(3) &= \langle\tilde{\varphi}|P_3|\tilde{\varphi}\rangle 
      = |a|^2\langle 100|011\rangle\langle 011|100\rangle 
      + |b|^2\langle 011|011\rangle\langle 011|011\rangle 
      = |b|^2 \\
  p(4) &= \langle\tilde{\varphi}|P_4|\tilde{\varphi}\rangle 
      = |a|^2\langle 100|100\rangle\langle 100|100\rangle 
      + |b|^2\langle 011|100\rangle\langle 100|011\rangle 
      = |a|^2 \\
\end{align*}

\begin{align*}
\frac{P_3|\tilde{\varphi}\rangle}{\sqrt{|b|^2}} 
  &= \frac{
    a|011\rangle\langle 011|100\rangle + 
    b|011\rangle\langle 011|011\rangle}{|b|}
  = \frac{b}{|b|}|011\rangle \\
\frac{P_4|\tilde{\varphi}\rangle}{\sqrt{|a|^2}} 
  &= \frac{
    a|100\rangle\langle 100|100\rangle + 
    b|100\rangle\langle 100|011\rangle}{|a|}
  = \frac{a}{|a|}|100\rangle \\
\end{align*}

En conclusión, los resultados serían similares para el resto de
casos de error: para cada par de medidas que corresponden a un error
en cierta posición, una de ellas tiene probabilidad $|a|^2$, la otra
$|b|^2$ y al medir se destruye la información salvo si $a = 0$ o $b
= 0$.

Por último, en cuanto a la fidelidad mímima al utilizar el código
asumiendo que no se producen errores por el canal, el resultado tras
enviar un mensaje $|\varphi\rangle = a|0\rangle + b|1\rangle$ es 

$$
\begin{cases}
  \frac{a}{|a|}|0\rangle 
    &\text{con probabilidad } |a|^2 \\
  \frac{b}{|b|}|1\rangle 
    &\text{con probabilidad } |b|^2
\end{cases}
$$

por lo que la fidelidad para un mensaje genérico sería 

\begin{align*}
  F_c(a|0\rangle + b|1\rangle) &= |a|^2\Big|(
    \overline{a}\langle0| + \overline{b}\langle 1| 
  )\frac{a}{|a|}|0\rangle\Big|^2 
  + |b|^2\Big|(
    \overline{a}\langle0| + \overline{b}\langle 1| 
  )\frac{b}{|b|}|1\rangle\Big|^2 \\ 
  &= |a|^2\Big|
    \frac{|a|^2}{|a|} 
  \Big|^2 
  + |b|^2\Big|
    \frac{|b|^2}{|b|}
  \Big|^2
  = |a|^2|a|^2 + |b|^2|b|^2
  = |a|^4 + |b|^4
\end{align*}

Para minimizar la expresión anterior, como $|a|^2 + |b|^2 = 1$,
podemos describir $|b|^2$ en función de $|a|^2$ y reescribirla como:

$$
  (|a|^2)^2 + |b|^4 = (1 - |b|^2)^2 + |b|^4 = 2|b|^4 - 2|b|^2 + 1
$$

Si derivamos con respecto a $|b|^2$ e igualamos a $0$ obtenemos 

\begin{align*}
  4|b|^2 - 2 &= 0 \\ 
  |b|^2 &= \frac{1}{2} \\
\end{align*}

por lo que la fidelidad mímima del canal es, teniendo en cuenta que
si $|b|^2 = \frac{1}{2}$ entonces $|a|^2 = \frac{1}{2}$:

\begin{align*}
  F_c^{min} 
    &= \Big(\frac{1}{2}\Big)^2 + \Big(\frac{1}{2}\Big)^2
    = \frac{1}{4} + \frac{1}{4} 
    = \frac{1}{2} 
\end{align*}

\subsection*{Ejercicio 3 - Protocolo BB84}

Durante la realización del protocolo BB84, Alice ha elegido como
cadenas de bits iniciales 

\begin{align*}
  \vec{a} &= (01100100) \\ 
  \vec{s} &= (ZZXXXZXZ) \\
\end{align*}

mientras que Bob ha elegido 

\begin{align*}
  \vec{v} &= (ZXXZXXXX) \\
\end{align*}

Asumiendo que no haya errores ni ataques durante el protocolo,
calcular:

\begin{enumerate}[a)]
  \item Una posible cadena de bits $\vec{b}$ de los resultados de
    las medidas de Bob.
  \item La clave compartida entre Alice y Bob al terminar el
    protocolo.
\end{enumerate}

\subsubsection*{Respuesta}

Vamos a seguir el protocolo para hallar una posible cadena de bits
que sea el resultado de las medidas de Bob.

En primer lugar, Alice codifica cada bit $a_i$ de $\vec{a}$ en la
base $|0\rangle, |1\rangle$ si $s_i$ es $Z$, y en la base
$|+\rangle, |-\rangle$ si $s_i$ es $X$, siendo $s_i$ el elemento en
la posición $i$ de $\vec{s}$. Esto significa que, tras ser enviados,
Bob recibe la siguiente secuencia de qubits:

$$
|0\rangle,\;|1\rangle,\;|-\rangle,\;|+\rangle,\;
|+\rangle,\;|1\rangle,\;|+\rangle,\;|0\rangle
$$

A continuación, Bob mide cada uno de esos qubits según la base
asociada a su cadena $\vec{v}$. Medir $|0\rangle$ o $|1\rangle$ en
la base $|0\rangle, |1\rangle$ provoca que el resultado sea
respectivamente $0$ o $1$ con probabilidad $1$. Lo mismo sucede al
medir $|+\rangle$ o $|-\rangle$ en la base $|+\rangle, |-\rangle$:

$$
\vec{b} = (0?1?0?0?)
$$

Por otra parte, medir $|+\rangle$ o $|-\rangle$ en la base
$|0\rangle, |1\rangle$ provoca que el resultado sea $0$ o $1$ con
probabilidad $0.5$. Lo mismo sucede al medir $|0\rangle$
o $|1\rangle$ en la base $|+\rangle, |-\rangle$. Un posible
resultado final podría ser el siguiente:

$$
\vec{b} = (00110101)
$$

Para finalizar el protocolo, Alice y Bob revelan $\vec{s}$
y $\vec{v}$, y descartan las posiciones de sus cadenas de bits en
las que $s_i \neq v_i$. En este caso, la clave compartida resultante
sería

$$
(0100)
$$

\end{document}
