\documentclass{article}
\usepackage[english]{babel}
\usepackage{hyperref}
\usepackage{amssymb}
\usepackage{mathtools}
\usepackage{mathpartir}
\usepackage{xcolor}

\renewcommand{\thesubsection}{Exercise \arabic{subsection}}

\newenvironment{changemargin}[2]{%
\begin{list}{}{%
\setlength{\topsep}{0pt}%
\setlength{\leftmargin}{#1}%
\setlength{\rightmargin}{#2}%
\setlength{\listparindent}{\parindent}%
\setlength{\itemindent}{\parindent}%
\setlength{\parsep}{\parskip}%
}%
\item[]}{\end{list}}

\title{Simply-typed $\lambda$-calculus exercises}
\author{Adrián Enríquez Ballester}

\begin{document}
\maketitle

\subsection{}\label{ex:1}

Give a proof in TA$_\lambda$ for the following type assignment
for function composition:

$$
  \vdash \lambda f.\lambda g. \lambda x.f\;(g\;x) : 
    (b \rightarrow c) \rightarrow 
    (a \rightarrow b) \rightarrow
    (a \rightarrow c)
$$

Its proof tree is the following:

\begin{mathpar}
  \inferrule*[Right=ABS]{
  \inferrule*[Right=ABS]{
  \inferrule*[Right=ABS]{
  \inferrule*[Right=APP]{
  \inferrule*[Left=VAR]{\;}{
    f : b \rightarrow c \vdash f: b \rightarrow c
  } \\
  \inferrule*[Right=APP]{
  \inferrule*[Left=VAR]{\;}{
    g:a \rightarrow b \vdash g:a \rightarrow b
  } \\ 
  \inferrule*[Right=VAR]{\;}{
    x:a \vdash x:a
  }
  }{
    g:a \rightarrow b, x:a \vdash g\;x:b
  }
  }{
      f:b \rightarrow c, g:a \rightarrow b, x:a 
      \vdash 
      f\;(g\;x):c
  }
  }{
    f:b \rightarrow c, g:a \rightarrow b
    \vdash 
    \lambda x.f\;(g\;x):  
      a \rightarrow c
  }
  }{
    f:b \rightarrow c \vdash
      \lambda g. \lambda x.f\;(g\;x):  
        (a \rightarrow b) \rightarrow
        (a \rightarrow c)
  }
  }{
    \vdash \lambda f.\lambda g. \lambda x.f\;(g\;x): 
    (b \rightarrow c) \rightarrow 
    (a \rightarrow b) \rightarrow
    (a \rightarrow c)
  }
\end{mathpar}

\subsection{}\label{ex:2}

Provide a type for the combinator $\mathsf{S} = 
\lambda x. \lambda y. \lambda z. (x\;z)\;(y\;z)$.

Its type can be

$$
    (a \rightarrow b \rightarrow c) \rightarrow 
    (a \rightarrow b) \rightarrow
    (a \rightarrow c)
$$

and the corresponding proof tree is the following:

\begin{changemargin}{-2cm}{-2cm}
\begin{mathpar}
  \inferrule*[Right=ABS]{
  \inferrule*[Right=ABS]{
  \inferrule*[Right=ABS]{
  \inferrule*[Right=APP]{
  \inferrule*[Left=APP]{
  \inferrule*[Left=VAR]{\;}{
    x:a \rightarrow b \rightarrow c
    \vdash 
    x:a \rightarrow b \rightarrow c
  }\\
  \inferrule*[Left=VAR]{\;}{
    z:a
    \vdash 
    z:a
  }
  }{
    x:a \rightarrow b \rightarrow c,
    z:a
    \vdash 
    x\;z:b \rightarrow c
  } \\
  \inferrule*[Right=APP]{
  \inferrule*[Right=VAR]{\;}{
    y:a \rightarrow b \vdash y:a \rightarrow b
  } \\ 
  \inferrule*[Right=VAR]{\;}{
    z:a \vdash z:a
  }
  }{
     y:a \rightarrow b,
     z:a
     \vdash 
     y\;z:b
  }
  }{
    x:a \rightarrow b \rightarrow c,
    y:a \rightarrow b, 
    z:a
    \vdash
    (x\;z)\;(y\;z):c
  }
  }{
    x:a \rightarrow b \rightarrow c,
    y:a \rightarrow b
    \vdash 
    \lambda z.(x\;z)\;(y\;z):  
      a \rightarrow c
  }
  }{
    x:a \rightarrow b \rightarrow c \vdash
      \lambda y. \lambda z.(x\;z)\;(y\;z):  
        (a \rightarrow b) \rightarrow
        (a \rightarrow c)
  }
  }{
    \vdash \lambda x.\lambda y. \lambda z.(x\;z)\;(y\;z): 
    (a \rightarrow b \rightarrow c) \rightarrow 
    (a \rightarrow b) \rightarrow
    (a \rightarrow c)
  }
\end{mathpar}
\end{changemargin}

\subsection{}\label{ex:3}

Provide a type for the term $\mathsf{II} = 
(\lambda x.x)\;(\lambda x.x)$.

Its type can be

$$
  a \rightarrow a
$$

and the corresponding proof tree is the following:

\begin{mathpar}
  \inferrule*[Right=APP]{
  \inferrule*[Left=ABS]{
  \inferrule*[Left=VAR]{\;}{
    x:a \rightarrow a \vdash x:a \rightarrow a
  }
  }{
    \vdash \lambda x.x:
      (a \rightarrow a) \rightarrow (a \rightarrow a)
  } \\
  \inferrule*[Right=ABS]{
  \inferrule*[Right=VAR]{\;}{
    x:a \vdash x:a
  }
  }{
    \vdash \lambda x.x:
      a \rightarrow a
  } 
  }{
    \vdash (\lambda x.x)\;(\lambda x.x):a \rightarrow a
  }
\end{mathpar}

\subsection{}\label{ex:4}

Provide types for the terms in exercise 5 of the previous sheet
(Church booleans) along with their corresponding derivation.

$\mathsf{TRUE}$ can be assigned the type $a \rightarrow b \rightarrow a$, 
and $\mathsf{FALSE}$ can be assigned the type $a \rightarrow b \rightarrow b$.
Their respective proof trees are the following, from left to right:

\begin{minipage}[t][3.3cm][t]{.3\textwidth}
\begin{mathpar}
  \inferrule*[Right=ABS]{
  \inferrule*[Right=ABS]{
  \inferrule*[Right=VAR]{\;}{
    x:a,y:b\vdash x:a
  }
  }{
    x:a\vdash \lambda y.x:
      b \rightarrow a
  } 
  }{
    \vdash \lambda x.\lambda y.x:a \rightarrow b \rightarrow a
  }
\end{mathpar}
\end{minipage}
\hfill
\begin{minipage}[t]{.3\textwidth}
\begin{mathpar}
  \inferrule*[Right=ABS]{
  \inferrule*[Right=ABS]{
  \inferrule*[Right=VAR]{\;}{
    x:a,y:b\vdash y:b
  }
  }{
    x:a\vdash \lambda y.y:
      b \rightarrow b
  } 
  }{
    \vdash \lambda x.\lambda y.y:a \rightarrow b \rightarrow b
  }
\end{mathpar}
\end{minipage}

The term $\mathsf{NOT}$ can be assigned the type 
$(a \rightarrow b \rightarrow c) \rightarrow b \rightarrow a 
\rightarrow c$, as proved in the following derivation tree:

\begin{mathpar}
  \inferrule*[Right=ABS]{
  \inferrule*[Right=ABS]{
  \inferrule*[Right=ABS]{
  \inferrule*[Right=APP]{
  \inferrule*[Left=APP]{
  \inferrule*[Left=VAR]{\;}{
    b:a \rightarrow b \rightarrow c
    \vdash 
    b:a \rightarrow b \rightarrow c
  } \\
  \inferrule*[Right=VAR]{\;}{
    y:a \vdash y:a
  } 
  }{
    b:a \rightarrow b \rightarrow c,
    y:a
    \vdash 
    b\;y:b \rightarrow c 
  } \\ 
  \inferrule*[Right=VAR]{\;}{
    x:b \vdash x:b
  }
  }{
    b:a \rightarrow b \rightarrow c,
    x:b,
    y:a \vdash 
      b\;y\;x:c
  }
  }{
    b:a \rightarrow b \rightarrow c,
    x:b \vdash 
      \lambda y.b\;y\;x:
        a \rightarrow c
  }
  }{
    b:a \rightarrow b \rightarrow c \vdash 
      \lambda x.\lambda y.b\;y\;x:
        b \rightarrow a \rightarrow c
  } 
  }{
    \vdash \lambda b.\lambda x.\lambda y.b\;y\;x:
      (a \rightarrow b \rightarrow c) 
      \rightarrow b \rightarrow a \rightarrow c
  }
\end{mathpar}

The term $\mathsf{CONJ}$ can be assigned the type 

$$(a \rightarrow b \rightarrow c) \rightarrow
(d \rightarrow b \rightarrow a) \rightarrow
d \rightarrow 
b \rightarrow 
c
$$

as proved in the following derivation tree:

\begin{changemargin}{-2cm}{-2cm}
\begin{mathpar}
  \inferrule*[Right=ABS]{
  \inferrule*[Right=ABS]{
  \inferrule*[Right=ABS]{
  \inferrule*[Right=ABS]{
  \inferrule*[Right=APP]{
  \inferrule*[Left=APP]{
  \inferrule*[Left=VAR]{\;}{
    b:a \rightarrow b \rightarrow c
    \vdash
    b:a \rightarrow b \rightarrow c
  } \\ 
  \Delta_1
  }{
    b:a \rightarrow b \rightarrow c,
    c:d \rightarrow b \rightarrow a,
    x:d,
    y:b
    \vdash
    b\;(c\;x\;y):b\rightarrow c
  } \\
  \inferrule*[Right=VAR]{\;}{
    y:b
    \vdash
    y:b
  }
  }{
    b:a \rightarrow b \rightarrow c,
    c:d \rightarrow b \rightarrow a,
    x:d,
    y:b
    \vdash
      b\;(c\;x\;y)\;y:
      c 
  }
  }{
    b:a \rightarrow b \rightarrow c,
    c:d \rightarrow b \rightarrow a,
    x:d
    \vdash \lambda y.
      b\;(c\;x\;y)\;y:
      b \rightarrow 
      c 
  }
  }{
    b:a \rightarrow b \rightarrow c,
    c:d \rightarrow b \rightarrow a
    \vdash \lambda x.\lambda y.
      b\;(c\;x\;y)\;y:
      d \rightarrow 
      b \rightarrow 
      c 
  }
  }{
    b:a \rightarrow b \rightarrow c
    \vdash \lambda c.\lambda x.\lambda y.
      b\;(c\;x\;y)\;y:
      (d \rightarrow b \rightarrow a) \rightarrow
      d \rightarrow 
      b \rightarrow 
      c 
  } 
  }{
    \vdash \lambda b.\lambda c.\lambda x.\lambda y.
      b\;(c\;x\;y)\;y:
      (a \rightarrow b \rightarrow c) \rightarrow
      (d \rightarrow b \rightarrow a) \rightarrow
      d \rightarrow 
      b \rightarrow 
      c 
  }
\end{mathpar}
\end{changemargin}

where the subtree $\Delta_1$ is the following:

\begin{mathpar}
\inferrule*[Right=APP]{
\inferrule*[Left=APP]{
\inferrule*[Left=VAR]{\;}{
  c:d \rightarrow b \rightarrow a
  \vdash 
  c:d \rightarrow b \rightarrow a
} \\ 
\inferrule*[Left=VAR]{\;}{
  x:d
  \vdash 
  x:d
}
}{
  c:d \rightarrow b \rightarrow a,
  x:d
  \vdash
  c\;x:b \rightarrow a
} \\
\inferrule*[Left=VAR]{\;}{
  y:b
  \vdash
  y:b
}
}{
  c:d \rightarrow b \rightarrow a,
  x:d,
  y:b
  \vdash
  c\;x\;y:a
}
\end{mathpar}

The term $\mathsf{DISJ}$ can be assigned the type 

$$
(a \rightarrow b \rightarrow c) \rightarrow
(a \rightarrow d \rightarrow b) \rightarrow
a \rightarrow 
d \rightarrow 
c 
$$

as proved in the following derivation tree:

\begin{changemargin}{-2cm}{-2cm}
\begin{mathpar}
  \inferrule*[Right=ABS]{
  \inferrule*[Right=ABS]{
  \inferrule*[Right=ABS]{
  \inferrule*[Right=ABS]{
  \inferrule*[Right=APP]{
  \Delta_2
  \\
  \inferrule*[Right=APP]{
  \inferrule*[Right=APP]{
  \inferrule*[Left=VAR]{\;}{
    c:a \rightarrow d \rightarrow b
    \vdash
    c:a \rightarrow d \rightarrow b
  } \\ 
  \inferrule*[Left=VAR]{\;}{
    x:a
    \vdash
    x:a
  }
  }{
    c:a \rightarrow d \rightarrow b,
    x:a
    \vdash
    c\;x:d \rightarrow b
  } \\
  \inferrule*[Right=VAR]{\;}{
    y:d
    \vdash
    y:d
  }
  }{
    c:a \rightarrow d \rightarrow b,
    x:a,
    y:d
    \vdash
    c\;x\;y:b
  }
  }{
    b:a \rightarrow b \rightarrow c,
    c:a \rightarrow d \rightarrow b,
    x:a,
    y:d
    \vdash
      b\;x\;(c\;x\;y):
      c 
  }
  }{
    b:a \rightarrow b \rightarrow c,
    c:a \rightarrow d \rightarrow b,
    x:a
    \vdash \lambda y.
      b\;x\;(c\;x\;y):
      d \rightarrow 
      c 
  }
  }{
    b:a \rightarrow b \rightarrow c,
    c:a \rightarrow d \rightarrow b
    \vdash \lambda x.\lambda y.
      b\;x\;(c\;x\;y):
      a \rightarrow 
      d \rightarrow 
      c 
  }
  }{
    b:a \rightarrow b \rightarrow c
    \vdash \lambda c.\lambda x.\lambda y.
    b\;x\;(c\;x\;y):
      (a \rightarrow d \rightarrow b) \rightarrow
      a \rightarrow 
      d \rightarrow 
      c 
  } 
  }{
    \vdash \lambda b.\lambda c.\lambda x.\lambda y.
      b\;x\;(c\;x\;y):
      (a \rightarrow b \rightarrow c) \rightarrow
      (a \rightarrow d \rightarrow b) \rightarrow
      a \rightarrow 
      d \rightarrow 
      c 
  }
\end{mathpar}
\end{changemargin}

where the subtree $\Delta_2$ is the following:

\begin{mathpar}
\inferrule*[Left=APP]{
\inferrule*[Left=VAR]{\;}{
  b:a \rightarrow b \rightarrow c
  \vdash
  b:a \rightarrow b \rightarrow c
} \\
\inferrule*[Left=VAR]{\;}{
  x:a
  \vdash
  x:a
} \\
}{
  b:a \rightarrow b \rightarrow c,
  x:a
  \vdash
  b\;x:b \rightarrow c
}
\end{mathpar}

\subsection{}\label{ex:5}

Prove that if $\Gamma \vdash M: \sigma$ then 
$\Gamma[a \mapsto \tau] \vdash M: \sigma[a \mapsto \tau]$. 

Recall the definition of a type-variable substitution, simplified 
for one substitution at a time, as

\begin{align*}
  a[a \mapsto \tau] &= \tau \\
  b[a \mapsto \tau] &= b \\
  (\tau_1 \rightarrow \tau_2)[a \mapsto \tau] &= 
    \tau_1[a \mapsto \tau] \rightarrow \tau_2[a \mapsto \tau] \\
\end{align*}

where $a$ and $b$ are distinct type variables, and $\tau, \tau_1$ and 
$\tau_2$ are arbitrary types.

This definition is also extended to type contexts as

$$
  \Gamma[a \mapsto \tau] = 
    x_1: \tau_1[a \mapsto \tau], 
    x_2: \tau_2[a \mapsto \tau], 
    ..., x_n: \tau_n[a \mapsto \tau]
$$

where $a$ is a type variable, $\tau$ is an arbitrary type, and
$\Gamma = x_1: \tau_1, x_2: \tau_2, ..., x_n: \tau_n$ is a context.

Note also that if a context $\Gamma$ is consistent, then also 
is $\Gamma[a \mapsto \tau]$ for any type variable $a$ and type $\tau$.
This is because if a variable $x$ were assigned two different types 
$\tau_1$ and $\tau_2$ in $\Gamma[a \mapsto \tau]$ and a single type 
$\tau$ in $\Gamma$, then $\tau[a \mapsto \tau] \neq \tau[a \mapsto \tau]$,
which is not possible.

Every type judgement must follow from a proof involving
only the rules of the deductive system (i.e. VAR, ABS and APP).
We are going to prove the statement by induction over the 
proof tree of the type judgement. As the case distinction is made
over type judgements of different shape, the context, types and variables 
used in each case do not refer exactly to the ones used in the general 
statement of the exercise.

If the bottommost rule of the proof tree is VAR, we have a judgement 
with shape $\Gamma, x: \sigma \vdash x: \sigma$. Trivially, $(\Gamma, x: \sigma)
[a \mapsto \tau] \vdash x: \sigma[a \mapsto \tau]$ also follows 
due to the VAR axiom because $(\Gamma, x: \sigma)
[a \mapsto \tau] = \Gamma[a \mapsto \tau], x: \sigma[a \mapsto \tau]$
and type-variable substitution preserves the context consistence.

Proceeding with the first of the inductive cases, if the bottommost rule of 
the proof tree is ABS, we have a judgement with shape 
$\Gamma \vdash \lambda x. M: \sigma \rightarrow \sigma'$. The 
ABS rule requires that $\Gamma, x: \sigma \vdash M: \sigma'$.
By rule induction hypothesis on the proof tree of this premise, we 
also have 

\begin{align*}
(\Gamma, x: \sigma)[a \mapsto \tau] 
  &\vdash M: \sigma'[a \mapsto \tau] \\
\Gamma[a \mapsto \tau], x: \sigma[a \mapsto \tau]
  &\vdash M: \sigma'[a \mapsto \tau]
\end{align*}

where the consistence of $(\Gamma, x: \sigma)[a \mapsto \tau]$
has been preserved with the type-variable substitution. This satisfies 
also the ABS rule yielding to the conclusion that we were looking for:

\begin{align*}
\Gamma[a \mapsto \tau] 
  &\vdash \lambda x. M: \sigma[a \mapsto \tau] \rightarrow \sigma'[a \mapsto \tau] \\
\Gamma[a \mapsto \tau] 
  &\vdash \lambda x. M: (\sigma \rightarrow \sigma')[a \mapsto \tau]
\end{align*}

Finally, for the last of the inductive cases, if the bottommost rule of the proof 
tree is APP, we have a judgement with shape
$\Gamma \cup \Gamma' \vdash M\;M': \sigma'$. The 
APP rule requires that $\Gamma \vdash M : \sigma \rightarrow \sigma'$
and $\Gamma' \vdash M' : \sigma$.

By rule induction hypothesis on the proof tree of both premises, we
also have

\begin{align*}
\Gamma[a \mapsto \tau] 
  &\vdash M : (\sigma \rightarrow \sigma')[a \mapsto \tau] \\
\Gamma[a \mapsto \tau]
  &\vdash M : \sigma[a \mapsto \tau] \rightarrow \sigma'[a \mapsto \tau]
\end{align*}

and 

\begin{align*}
\Gamma'[a \mapsto \tau] \vdash M' : \sigma[a \mapsto \tau] 
\end{align*}

These last statements satisfy the APP rule yielding to 
the conclusion that we were looking for:

\begin{align*}
\Gamma[a \mapsto \tau] \cup \Gamma'[a \mapsto \tau] 
  &\vdash M\;M': \sigma'[a \mapsto \tau] \\
(\Gamma \cup \Gamma')[a \mapsto \tau] 
  &\vdash M\;M': \sigma'[a \mapsto \tau]
\end{align*}

Note again that the consistence of $\Gamma \cup \Gamma'$ has been 
preserved with the type-variable substitution.

\subsection{}\label{ex:6}

Prove that if $\Gamma, x: \sigma \vdash M: \tau$ and 
$\Gamma \vdash N: \sigma$ then 
$\Gamma \vdash M[x \mapsto N]: \tau$.

This time, we are going to prove the statement by structural
induction on the lambda term where the substitution is
taking place (i.e. $M$ from the statement). 
Again, as the case distinction is made
over lambda terms of different shape, the context, types and variables 
used in each case do not refer exactly to the ones used in the general 
statement of the exercise.

If the lambda term is a variable, we have two possible 
cases, according to the definition of capture-avoiding substitution, 
depending on whether the variable being replaced is the same as the 
term or not:

\begin{itemize}
  \item If the term is a variable $x$, its stated type judgement 
    has shape $\Gamma, x: \sigma \vdash x: \sigma$ due to the VAR 
    rule and, as $x[x \mapsto N] = N$, the property holds for any 
    $N$ such that $\Gamma \vdash N: \sigma$.
  \item If the term is a variable $y$ with $x \neq y$, its stated 
    type judgement has shape $\Gamma, y: \tau, x: \sigma \vdash y: \tau$ 
    due to the VAR rule and, as $y[x \mapsto N] = y$, 
    and $\Gamma, y: \tau \vdash y: \tau$ due to the VAR rule, the 
    property also holds.
\end{itemize}

Proceeding with one of the inductive cases, if the lambda term is 
an application, its stated type judgement has shape 
$\Gamma \cup \Gamma', x: \sigma \vdash M\;M': \tau'$,
that is the same as $(\Gamma, x: \sigma) \cup (\Gamma', x: \sigma) 
\vdash M\;M': \tau'$
where its APP rule requires that
$\Gamma, x: \sigma \vdash M : \tau \rightarrow \tau'$ and 
$\Gamma', x: \sigma \vdash M': \tau$.

By structural induction on the respective subterms $M$ and $M'$
for these last type judgements, if $\Gamma \vdash N: \sigma$ then 
we also have

\begin{align*}
  \Gamma
  &\vdash M[x \mapsto N] : \tau \rightarrow \tau' \\
  \Gamma' &\vdash M'[x \mapsto N]: \tau
\end{align*}

By applying the APP rule to these last premises, we reach the 
conclusion that we were looking for:

\begin{align*}
  \Gamma \cup \Gamma' &\vdash M[x \mapsto N]\;M'[x \mapsto N]: \tau' \\
  \Gamma \cup \Gamma' &\vdash (M\;M')[x \mapsto N]: \tau' \\
\end{align*}

Finally, if the lambda term is an abstraction, we have again several 
cases to consider according to the definition of capture-avoiding 
substitution:

\begin{itemize}
  \item Consider that the statement has shape 
    $\Gamma, x: \sigma \vdash \lambda x. M : \tau \rightarrow \tau'$
    (i.e. we are trying to replace the variable that is bound by an 
    abstraction).

    On the one hand, its ABS rule premise requires 
    $\Gamma, x: \tau \vdash M : \tau'$, and it means that 
    $\Gamma \vdash \lambda x. M : \tau \rightarrow \tau'$.

    On the other hand, $(\lambda x. M)[x \mapsto N] = \lambda x. M$,
    thus 
    
    $$\Gamma \vdash (\lambda x. M)[x \mapsto N] : \tau \rightarrow \tau'$$

    and the property is satisfied.
  \item Consider now the case where it has shape 
    $\Gamma, x: \sigma \vdash \lambda y. M : \tau \rightarrow \tau'$ with 
    $x \notin FV(M)$.

    On the one hand, as $x$ does not appear free in $M$, it also does not appear 
    free in $\lambda y. M$ and the type judgment 
    will still true even without the type assignment for $x$ in the context
    \footnote{This claim, although simple, may require its own proof.}:

    $$\Gamma \vdash \lambda y. M : \tau \rightarrow \tau'$$

    On the other hand, $(\lambda y. M)[x \mapsto N] = \lambda y. M$,
    thus 
    
    $$\Gamma \vdash (\lambda y. M)[x \mapsto N] : \tau \rightarrow \tau'$$

    and the property is satisfied.
  \item Consider now the case where it has shape 
    $\Gamma, x: \sigma \vdash \lambda y. M : \tau \rightarrow \tau'$ with 
    $x \in FV(M)$ and $y \notin FV(N)$ for a term $N$ such that 
    $\Gamma \vdash N: \sigma$.

    On the one hand, its ABS rule premise requires 
    $\Gamma, x: \sigma, y: \tau \vdash M : \tau'$
    and, by rule induction hypothesis on the subterm $M$
    for this last type judgement, we also have
    
    $$
      \Gamma, y: \tau \vdash M[x \mapsto N] : \tau'
    $$

    This satisfies the ABS rule leading to 

    $$
      \Gamma \vdash \lambda y. M[x \mapsto N] : \tau \rightarrow \tau'
    $$

    On the other hand, in this case 
    $(\lambda y. M)[x \mapsto N] = \lambda y. M[x \mapsto N]$,
    thus
    
    $$\Gamma \vdash (\lambda y. M)[x \mapsto N] : \tau \rightarrow \tau'$$

    and the property is satisfied.
  \item Finally, consider that it has shape 
    $\Gamma, x: \sigma \vdash \lambda y. M : \tau \rightarrow \tau'$ with 
    $x \in FV(M)$ and $y \in FV(N)$ for a term $N$ such that 
    $\Gamma \vdash N: \sigma$.

    On the one hand, its ABS rule premise requires 
    $\Gamma, x: \sigma, y: \tau \vdash M : \tau'$.

    On the one hand, in this case 
    $(\lambda y. M)[x \mapsto N] = \lambda z. M[y \mapsto z][x \mapsto N]$
    where $z \notin FV(M) \cup FV(N)$.

    As $z$ does not occur free neither in $M$ nor $N$, we can assign 
    it type $\tau$ by keeping the consistency of 
    the respective contexts obtaining the following type judgements
    \footnote{Again, this claim, although simple, may require its own proof.}:

    \begin{align*}
      \Gamma, z: \tau &\vdash z: \tau \\
      \Gamma, x: \sigma, y: \tau, z: \tau &\vdash M : \tau' \\
      \Gamma, z: \tau &\vdash N: \sigma \\ 
    \end{align*}
    
    By applying the induction hypothesis on the subterm $M$ 
    for the first two judgements above, we have

    $$
    \Gamma, x: \sigma, z: \tau \vdash M[y \mapsto z] : \tau'
    $$

    and, by applying it again over $M[y \mapsto z]$ for this 
    last judgement and the third one, we also have

    $$
    \Gamma, z: \tau \vdash M[y \mapsto z][x \mapsto N] : \tau'
    $$

    These reached type judgement satisfy the ABS rule leading to

    \begin{align*}
      \Gamma &\vdash \lambda z. M[y \mapsto z][x \mapsto N] : \tau \rightarrow \tau' \\
      \Gamma &\vdash (\lambda y. M)[x \mapsto N] : \tau \rightarrow \tau'
    \end{align*}

    and therefore the property is satisfied.
\end{itemize}

\subsection{}\label{ex:7}

Prove that if $M \leadsto^{*}_{\beta} M'$ and 
$\Gamma \vdash M: \sigma$ then 
$\Gamma \vdash M': \sigma$. Informally, this states that 
$\beta$-reduction preserves types.

We are going to prove the statement by induction on the length of the
beta reduction chain.

On the one hand, if its length is $0$ (i.e. $M \leadsto^{0}_{\beta} M'$),
as no $\beta$-reduction has been performed, $M = M'$ and it is trivial that
if $\Gamma \vdash M: \sigma$ then $\Gamma \vdash M: \sigma$.

On the other hand, if the chain has length $k$ then 

$$M \leadsto^{k - 1}_{\beta} M'' \leadsto_{\beta} M'$$

for some lambda term $M''$. By induction hypothesis, if 
$\Gamma \vdash M: \sigma$ then $\Gamma \vdash M'': \sigma$.

For this last $\beta$-reduction to be possible, $M''$ has to be 
of the shape $(\lambda x. X)\;N$ with $M' = X[x \mapsto N]$ and, 
due to its type judgement stated above, it has a proof tree as 
follows:

\begin{mathpar}
  \inferrule*[Right=APP]{
  \inferrule*[Left=ABS]{
  \inferrule*{
    \Delta_1
  }{
    \Gamma_1, x: \sigma' \vdash X: \sigma
  }
  }{
    \Gamma_1 \vdash \lambda x. X: \sigma' \rightarrow \sigma
  } \\
  \inferrule*{
    \Delta_2
  }{
    \Gamma_2 \vdash N: \sigma'
  } 
  }{
    \Gamma_1 \cup \Gamma_2 \vdash (\lambda x. X)\;N:\sigma
  }
\end{mathpar}

The bottomost few rules allow us to obtain the judgements
$\Gamma_1, x: \sigma' \vdash X: \sigma$ and $\Gamma_2 \vdash N: \sigma'$ 
independently of the proof trees $\Delta_1$ and $\Delta_2$
and, as $\Gamma_1 \cup \Gamma_2$ must be consistent, it also implies that 
the type judgements

\begin{align*}
  \Gamma_1 \cup \Gamma_2, x: \sigma' &\vdash X: \sigma \\
  \Gamma_1 \cup \Gamma_2 &\vdash N: \sigma'
\end{align*}

also hold\footnote{
  If $x: \sigma'$ were not consistent with $\Gamma_2$, we could 
  have performed an $\alpha$-conversion to the abstraction 
  $(\lambda x. X)$ in order to choose a bound variable which does 
  not lead to this problem.
}. Therefore, by the \ref{ex:6}, 
$\Gamma_1 \cup \Gamma_2 \vdash X[x \mapsto N] : \sigma$, which is precisely
$M$, the resulting term of $\beta$-reducing $M''$.

\end{document}
