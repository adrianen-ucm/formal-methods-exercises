\documentclass{article}
\usepackage[english]{babel}
\usepackage{hyperref}
\usepackage{amssymb}
\usepackage{mathtools}
\usepackage{mathpartir}

\renewcommand{\thesubsection}{
  Exercise \arabic{subsection}
}

\renewcommand{\thesubsubsection}{
  Answer
}

\newenvironment{changemargin}[2]{%
\begin{list}{}{%
\setlength{\topsep}{0pt}%
\setlength{\leftmargin}{#1}%
\setlength{\rightmargin}{#2}%
\setlength{\listparindent}{\parindent}%
\setlength{\itemindent}{\parindent}%
\setlength{\parsep}{\parskip}%
}%
\item[]}{\end{list}}

\title{Curry-Howard/$\lambda 2$}
\author{Adrián Enríquez Ballester}

\begin{document}
\maketitle

\section*{The Curry-Howard isomorphism}

\subsection{}\label{ex:1}

Provide a $\lambda$-term equivalent to a proof in NJ of the
following:

$$
  p 
    \rightarrow (q \rightarrow r) 
    \rightarrow ((p \rightarrow q) 
    \rightarrow (p \rightarrow r))
$$

\subsubsection{}

The term can be 

$$
  \lambda x. \lambda f. \lambda g. \lambda y. f\;(g\;x)
$$

and the proof tree for its proposed type is the following:

\begin{mathpar}
  \inferrule*[Right=ABS]{
  \inferrule*[Right=ABS]{
  \inferrule*[Right=ABS]{
  \inferrule*[Right=ABS]{
  \inferrule*[Right=APP]{
  \inferrule*[Left=VAR]{\;}{
    f : p \rightarrow r \vdash f: p \rightarrow r
  } \\
  \inferrule*[Right=APP]{
  \inferrule*[Left=VAR]{\;}{
    g:p \rightarrow q \vdash g:p \rightarrow q
  } \\ 
  \inferrule*[Right=VAR]{\;}{
    x:p \vdash x:p
  }
  }{
    g:p \rightarrow q, x:p \vdash g\;x:p
  }
  }{
    x: p, f: q \rightarrow r, g: p \rightarrow q, y: p 
      \vdash 
        f\;(g\;x): r
  }
  }{
    x: p, f: q \rightarrow r, g: p \rightarrow q \vdash 
      \lambda y.
        f\;(g\;x): 
    p \rightarrow r
  }
  }{
    x: p, f: q \rightarrow r \vdash 
      \lambda g. \lambda y.
        f\;(g\;x): 
    (p \rightarrow q) 
    \rightarrow (p \rightarrow r)
  }
  }{
    x: p \vdash 
      \lambda f.\lambda g. \lambda y.
        f\;(g\;x): 
    (q \rightarrow r) 
    \rightarrow ((p \rightarrow q) 
    \rightarrow (p \rightarrow r))
  }
  }{
    \vdash 
      \lambda x. \lambda f.\lambda g. \lambda y.
        f\;(g\;x): 
    p 
    \rightarrow (q \rightarrow r) 
    \rightarrow ((p \rightarrow q) 
    \rightarrow (p \rightarrow r))
  }
\end{mathpar}

Here is also its proof in NJ, in order to show their
analogy:

\begin{mathpar}
  \inferrule*[Right=$\rightarrow i^a$]{
  \inferrule*[Right=$\rightarrow i^b$]{
  \inferrule*[Right=$\rightarrow i^c$]{
    \inferrule*[Right=$\rightarrow i^d$]{
      \inferrule*[Right=$\rightarrow e$]{
        [q \rightarrow r]^b
        \\
        \inferrule*[Right=$\rightarrow e$]{
          \left[p \rightarrow q\right]^c 
          \\ \left[p\right]^a
        }{
          q
        }
        }{
          r
        }
      }{
        p \rightarrow r
      }
    }{
      (p \rightarrow q) 
      \rightarrow (p \rightarrow r)
    }
  }{
    (q \rightarrow r) 
    \rightarrow ((p \rightarrow q) 
    \rightarrow (p \rightarrow r))
  }
  }{
    p 
    \rightarrow (q \rightarrow r) 
    \rightarrow ((p \rightarrow q) 
    \rightarrow (p \rightarrow r))
  }
\end{mathpar}

This can be checked in the Haskell REPL, for example, 
by declaring the term with the type specified and 
observing that it is accepted without no complain:

\begin{verbatim}
Prelude> :{
  Prelude| term :: p -> (q -> r) -> ((p -> q) -> (p -> r))
  Prelude| term = \x f g y -> f (g x)
  Prelude| :}
Prelude>
\end{verbatim}

\subsection{}\label{ex:2}

Extend the Curry-Howard isomorphism to consider the 
conjunction operator ($\land$). Define a translation
of proofs in the subset of NJ with implication and 
conjunction into extended $\lambda$-terms and provide 
rules for $\beta$-reduction. Justify your decisions.

\subsubsection{}

On the one hand, we add the following to the syntax 
of lambda terms:

\begin{align*}
  &\vdots\\
  &|\;\langle T, T' \rangle\\
  &|\;\pi^1\;T\\
  &|\;\pi^2\;T\\
\end{align*}

which correspond to a pair term, and the first and second 
component projections. The following $\beta$-reduction 
rules are also added so we can effectively use these 
projections:

\begin{align*}
  \pi^1\;\langle T, T' \rangle &\leadsto_\beta T\\
  \pi^2\;\langle T, T' \rangle &\leadsto_\beta T'\\
\end{align*}

On the other hand, we add the following to the syntax 
of types:

\begin{align*}
  &\vdots\\
  &|\tau \times \tau'\\
\end{align*}

which corresponds to a product type and is intended to be
the type for pair terms. We define the following typing rules 
for these new terms:

\begin{mathpar}
  \inferrule*[Right=PROD]{
    \Gamma \vdash T: \tau 
    \\ 
    \Gamma' \vdash T': \tau' 
  }{
    \Gamma \cup \Gamma' \vdash \langle T, T' \rangle 
      : \tau \times \tau'
  }
\end{mathpar}

\begin{minipage}[t][2cm][t]{.4\textwidth}
\begin{mathpar}
  \inferrule*[Right=PROJ1]{
    \Gamma \vdash T : \tau \times \tau'
  }{
    \Gamma \vdash \pi^1\;T : \tau
  }
\end{mathpar}
\end{minipage}
\hfill
\begin{minipage}[t]{.4\textwidth}
\begin{mathpar}
  \inferrule*[Right=PROJ2]{
    \Gamma \vdash T : \tau \times \tau'
  }{
    \Gamma \vdash \pi^2\;T : \tau'
  }
\end{mathpar}
\end{minipage}

Finally, the traduction of proofs in the subset of NJ 
to terms taking into account the conjunction operator 
adds the following to the traduction that we already 
have:

\begin{minipage}[c][2cm][c]{.4\textwidth}
\begin{mathpar}
  \inferrule*[Right=$\land i$]{
    p \\ q
  }{
    p \land q
  }
\end{mathpar}
\end{minipage}
$\Longrightarrow$
\begin{minipage}[c]{.4\textwidth}
$$\langle P, Q \rangle$$
\end{minipage}

\begin{minipage}[c][2cm][c]{.4\textwidth}
\begin{mathpar}
  \inferrule*[Right=$\land e_1$]{
    p \land q
  }{
    p
  }
\end{mathpar}
\end{minipage}
$\Longrightarrow$
\begin{minipage}[c]{.4\textwidth}
$$\pi^1\;T$$
\end{minipage}

\begin{minipage}[c][2cm][c]{.4\textwidth}
\begin{mathpar}
  \inferrule*[Right=$\land e_2$]{
    p \land q
  }{
    q
  }
\end{mathpar}
\end{minipage}
$\Longrightarrow$
\begin{minipage}[c]{.4\textwidth}
$$\pi^2\;T$$
\end{minipage}

where $P$ is the term corresponding to the proof of $p$, 
$Q$ is the term corresponding to the proof of $q$
and $T$ is the term corresponding to the proof of 
$p \land q$.

\subsection{}\label{ex:3}

Provide a type $\lambda$-term equivalent to a proof in 
NJ of the following:

$$(p \land q \land r) \rightarrow (p \land r)$$

\subsubsection{}

The required term can be

$$
  \lambda x. \langle \pi^1\;x, \pi^2\;(\pi^2\;x) \rangle
$$

and the proof tree for its proposed type is the following:

\begin{mathpar}
  \inferrule*[Right=ABS]{
  \inferrule*[Right=PROD]{
  \inferrule*[Left=PROJ1]{
  \inferrule*[Left=VAR]{\;}{
    x: p \times q \times r 
    \vdash 
    x: p \times q \times r
  }
  }{
    x: p \times q \times r \vdash 
      \pi^1\;x: p
  } \\
  \inferrule*[Right=PROJ2]{
  \inferrule*[Right=PROJ2]{
  \inferrule*[Right=VAR]{\;}{
    x: p \times q \times r
    \vdash 
    x: p \times q \times r
  }
  }{
    x: p \times q \times r \vdash 
      \pi^2\;x: q \times r
  }
  }{
    x: p \times q \times r \vdash 
      \pi^2\;(\pi^2\;x): r
  }
  }{
    x: p \times q \times r \vdash 
      \langle \pi^1\;x, \pi^2\;(\pi^2\;x) \rangle: 
        p \times r
  }
  }{
    \vdash 
      \lambda x. \langle \pi^1\;x, \pi^2\;(\pi^2\;x) \rangle: 
        (p \times q \times r) \rightarrow (p \times r)
  }
\end{mathpar}

As in the \ref{ex:1}, we can check this with the Haskell REPL 
by declaring the term with this specific type signature
and observing that it is accepted with no complain:

\begin{verbatim}
Prelude> :{
Prelude| term :: (p, (q, r)) -> (p, r)
Prelude| term = \x -> (fst x, snd (snd x))
Prelude| :}
Prelude>
\end{verbatim}

\subsection{}\label{ex:4}

Extend the Curry-Howard isomorphism to support disjunction
($\lor$) analogously to the \ref{ex:2}.

\subsubsection{}

On the one hand, we add the following to the syntax 
of lambda terms:

\begin{align*}
  &\vdots\\
  &|\;\iota^1\;T\\
  &|\;\iota^2\;T\\
  &|\;\delta\;T_1\;T_2\;T\\
\end{align*}

which correspond to a left and right sum term, 
and its eliminator. The following $\beta$-reduction 
rules are also added so we can effectively use the 
sum eliminator:

\begin{align*}
  \delta\;T_1\;T_2\;(\iota^1\;T) &\leadsto_\beta T_1\;T\\
  \delta\;T_1\;T_2\;(\iota^2\;T) &\leadsto_\beta T_2\;T\\
\end{align*}

On the other hand, we add the following to the syntax of types:

\begin{align*}
  &\vdots\\
  &|\tau + \tau'\\
\end{align*}

which corresponds to a sum type and is intended to be 
the type for sum terms. We define the 
following typing rules for these new terms:

\begin{mathpar}
  \inferrule*[Right=UNSUM]{
    \Gamma_1 \vdash T_1: \tau \rightarrow \alpha
    \\ 
    \Gamma_2 \vdash T_2: \tau' \rightarrow \alpha
    \\ 
    \Gamma \vdash T: \tau + \tau'
  }{
    \Gamma_1 \cup \Gamma_2 \cup \Gamma \vdash 
      \delta\;T_1\;T_2\;T : \alpha
  }
\end{mathpar}

\begin{minipage}[t][2cm][t]{.4\textwidth}
\begin{mathpar}
  \inferrule*[Right=SUML]{
    \Gamma \vdash T : \tau
  }{
    \Gamma \vdash \iota^1\;T : \tau + \tau'
  }
\end{mathpar}
\end{minipage}
\hfill
\begin{minipage}[t]{.4\textwidth}
\begin{mathpar}
  \inferrule*[Right=SUMR]{
    \Gamma \vdash T : \tau'
  }{
    \Gamma \vdash \iota^2\;T : \tau + \tau'
  }
\end{mathpar}
\end{minipage}

Finally, the traduction of proofs to terms in the subset 
of NJ taking into account the disjunction operator adds 
the following to the traduction that we already have:

\begin{minipage}[c][2cm][c]{.4\textwidth}
\begin{mathpar}
  \inferrule*[Right=$\lor e$]{
    p \lor q \\ p \rightarrow e \\ q \rightarrow e
  }{
    r
  }
\end{mathpar}
\end{minipage}
$\Longrightarrow$
\begin{minipage}[c]{.4\textwidth}
$$\delta\;E_p\;E_q\;T$$
\end{minipage}

\begin{minipage}[c][2cm][c]{.4\textwidth}
\begin{mathpar}
  \inferrule*[Right=$\lor i_1$]{
    p
  }{
    p \lor q
  }
\end{mathpar}
\end{minipage}
$\Longrightarrow$
\begin{minipage}[c]{.4\textwidth}
$$\iota^1\;P$$
\end{minipage}

\begin{minipage}[c][2cm][c]{.4\textwidth}
\begin{mathpar}
  \inferrule*[Right=$\lor i_2$]{
    q
  }{
    p \lor q
  }
\end{mathpar}
\end{minipage}
$\Longrightarrow$
\begin{minipage}[c]{.4\textwidth}
$$\iota^2\;Q$$
\end{minipage}

where $P$ is the term corresponding to the proof of $p$, 
$Q$ is the term corresponding to the proof of $q$,
$T$ is the term corresponding to the proof of 
$p \lor q$, $E_p$ is the proof of $p \rightarrow e$
and $E_q$ is the proof of $q \rightarrow e$.

\subsection{}\label{ex:5}

Provide a $\lambda$-term equivalent to a proof in NJ of:

$$p \rightarrow (p \land (p \lor q))$$

\subsubsection{}

The term can be

$$
  \lambda x. \langle x, \iota^1\;x \rangle
$$

and the proof tree for its proposed type is the following:

\begin{mathpar}
  \inferrule*[Right=ABS]{
  \inferrule*[Right=PROD]{
  \inferrule*[Left=VAR]{\;}{
    x: p \vdash x: p
  } \\
  \inferrule*[Right=SUML]{
  \inferrule*[Right=VAR]{\;}{
    x: p \vdash x: p
  }
  }{
    \iota^1\;x: p + q
  }
  }{
    x: p \vdash 
      \langle x, \iota^1\;x \rangle: 
      p \times (p + q)
  }
  }{
    \vdash \lambda x. \langle x, \iota^1\;x \rangle: 
      p \rightarrow (p \times (p + q))
  }
\end{mathpar}

Again, we can check this in the Haskell REPL by 
declaring this therm together with its type 
signature, and observing that it is accepted:
\begin{verbatim}
Prelude> :{
Prelude| term :: p -> (p, Either p q)
Prelude| term = \x -> (x, Left x)
Prelude| :}
Prelude>
\end{verbatim}

\section*{The polymorphic $\lambda$-calculus}

\subsection{}\label{ex:6}

Provide a polymorphic definition of function composition
(Exercise 10 in the previous exercise sheet) in $\lambda 2$.
Use it to provide an alternate \textit{cheap} definition for 
the Church numerals $\mathsf{2}, \mathsf{3}$, 
and $\mathsf{4}$. Verify your answer via reduction.

\subsubsection{}

The term is 

$$
\mathsf{COMP} \triangleq
  \Lambda a. \Lambda b. \Lambda c. 
    \lambda f^{b \rightarrow c}.
    \lambda g^{a \rightarrow b}.
    \lambda x^a.
      f\;(g\;x)
$$

and its proof tree is as follows:

\begin{mathpar}
  \inferrule*[Right=$\forall i$]{
  \inferrule*[Right=$\forall i$]{
  \inferrule*[Right=$\forall i$]{
  \inferrule*[Right=ABS]{
  \inferrule*[Right=ABS]{
  \inferrule*[Right=ABS]{
  \inferrule*[Right=APP]{
  \inferrule*[Left=VAR]{\;}{
    f : b \rightarrow c \vdash f: b \rightarrow c
  } \\
  \inferrule*[Right=APP]{
  \inferrule*[Left=VAR]{\;}{
    g:a \rightarrow b \vdash g:a \rightarrow b
  } \\
  \inferrule*[Right=VAR]{\;}{
    x:a \vdash x:a
  }
  }{
    g:a \rightarrow b, x:a \vdash g\;x:b
  }
  }{
      f:b \rightarrow c, g:a \rightarrow b, x:a
      \vdash
      f\;(g\;x):c
  }
  }{
    f:b \rightarrow c, g:a \rightarrow b \vdash
      \lambda x^a.
        f\;(g\;x):
      a \rightarrow c
  }
  }{
    f:b \rightarrow c \vdash
      \lambda g^{a \rightarrow b}. 
      \lambda x^a.
        f\;(g\;x):
      (a \rightarrow b) \rightarrow
      (a \rightarrow c)
  }
  }{
    \vdash 
      \lambda f^{b \rightarrow c}.
      \lambda g^{a \rightarrow b}. 
      \lambda x^a.
        f\;(g\;x):
      (b \rightarrow c) \rightarrow
      (a \rightarrow b) \rightarrow
      (a \rightarrow c)
  }
  }{
    \vdash \Lambda c. 
      \lambda f^{b \rightarrow c}.
      \lambda g^{a \rightarrow b}. 
      \lambda x^a.
        f\;(g\;x): \forall c.
      (b \rightarrow c) \rightarrow
      (a \rightarrow b) \rightarrow
      (a \rightarrow c)
  }
  }{
    \vdash \Lambda b. \Lambda c. 
      \lambda f^{b \rightarrow c}.
      \lambda g^{a \rightarrow b}. 
      \lambda x^a.
        f\;(g\;x): \forall b.\forall c.
      (b \rightarrow c) \rightarrow
      (a \rightarrow b) \rightarrow
      (a \rightarrow c)
  }
  }{
    \vdash \Lambda a. \Lambda b. \Lambda c. 
      \lambda f^{b \rightarrow c}.
      \lambda g^{a \rightarrow b}. 
      \lambda x^a.
        f\;(g\;x): \forall a.\forall b.\forall c.
      (b \rightarrow c) \rightarrow
      (a \rightarrow b) \rightarrow
      (a \rightarrow c)
  }
\end{mathpar}

By having $\mathsf{1}$ defined as $\Lambda t. \lambda 
f^{t \rightarrow t}. \lambda x^t. f\;x$, we can, for 
example, define the following numbers as

\begin{align*}
  \mathsf{2} &\triangleq \Lambda t.
    \lambda f^{t \rightarrow t}.
      \mathsf{COMP}[t][t][t]\;
      (\mathsf{1}[t]\;f)\;
      (\mathsf{1}[t]\;f) \\
  \mathsf{3} &\triangleq \Lambda t.
    \lambda f^{t \rightarrow t}.
      \mathsf{COMP}[t][t][t]\;
      (\mathsf{1}[t]\;f)\;
      (\mathsf{2}[t]\;f) \\
  \mathsf{4} &\triangleq \Lambda t.
    \lambda f^{t \rightarrow t}.
      \mathsf{COMP}[t][t][t]\;
      (\mathsf{2}[t]\;f)\;
      (\mathsf{2}[t]\;f) \\
\end{align*}

First, this shows how the defined term $\mathsf{2}$ 
reduces as expected:

\begingroup
\allowdisplaybreaks
\begin{align*}
  \mathsf{2} 
    \equiv&
      \Lambda t.
      \lambda f^{t \rightarrow t}.
        \mathsf{COMP}[t][t][t]\;
        (\mathsf{1}[t]\;f)\;
        (\mathsf{1}[t]\;f) \\
    \equiv&
      \Lambda t.
      \lambda f^{t \rightarrow t}.
        \\&\;\;\;\;\;\;\;\;(\Lambda a. \Lambda b. \Lambda c. 
        \lambda f^{b \rightarrow c}.
        \lambda g^{a \rightarrow b}.
        \lambda x^a.
          f\;(g\;x))[t][t][t]\;
        \\&\;\;\;\;\;\;\;\;((
          \Lambda t. \lambda f^{t \rightarrow t}. \lambda x^t. f\;x
          )[t]\;f)\;
        \\&\;\;\;\;\;\;\;\;((\Lambda t. \lambda f^{t \rightarrow t}. \lambda x^t. f\;x)[t]\;f) \\
    \leadsto^{5}& 
      \Lambda t.
      \lambda f^{t \rightarrow t}.
        \\&\;\;\;\;\;\;\;\;(
        \lambda f^{t \rightarrow t}.
        \lambda g^{t \rightarrow t}.
        \lambda x^t.
          f\;(g\;x))\;
        \\&\;\;\;\;\;\;\;\;((
          \lambda f^{t \rightarrow t}. \lambda x^t. f\;x
          )\;f)\;
        \\&\;\;\;\;\;\;\;\;((\lambda f^{t \rightarrow t}. 
          \lambda x^t. f\;x)\;f) \\
    \leadsto^{2}& 
      \Lambda t.
      \lambda f^{t \rightarrow t}.
        (
        \lambda f^{t \rightarrow t}.
        \lambda g^{t \rightarrow t}.
        \lambda x^t.
          f\;(g\;x))\;
        \;(\lambda x^t. f\;x)
        \;(\lambda x^t. f\;x) \\
    \leadsto^{2}& 
      \Lambda t.
      \lambda f^{t \rightarrow t}.
        \lambda x^t.
        (\lambda x^t. f\;x)\;((\lambda x^t. f\;x)\;x)
      \\
    \leadsto& 
      \Lambda t.
      \lambda f^{t \rightarrow t}.
        \lambda x^t.
        (\lambda x^t. f\;x)\;(f\;x)
      \\
    \leadsto& 
      \Lambda t.
      \lambda f^{t \rightarrow t}.
        \lambda x^t. f\;(f\;x)
      \\
\end{align*}
\endgroup

Second, this shows how the defined $\mathsf{3}$ 
reduces to the expected term by also using the previous 
reduction for $\mathsf{2}$:

\begingroup
\allowdisplaybreaks
\begin{align*}
  \mathsf{3} 
    \equiv&
      \Lambda t.
      \lambda f^{t \rightarrow t}.
        \mathsf{COMP}[t][t][t]\;
        (\mathsf{1}[t]\;f)\;
        (\mathsf{2}[t]\;f) \\
    \equiv&
      \Lambda t.
      \lambda f^{t \rightarrow t}.
        \\&\;\;\;\;\;\;\;\;(\Lambda a. \Lambda b. \Lambda c. 
        \lambda f^{b \rightarrow c}.
        \lambda g^{a \rightarrow b}.
        \lambda x^a.
          f\;(g\;x))[t][t][t]\;
        \\&\;\;\;\;\;\;\;\;((
          \Lambda t. \lambda f^{t \rightarrow t}. \lambda x^t. f\;x
          )[t]\;f)\;
        \\&\;\;\;\;\;\;\;\;(\mathsf{2}[t]\;f) \\
    \leadsto^{*}&
      \Lambda t.
      \lambda f^{t \rightarrow t}.
        \\&\;\;\;\;\;\;\;\;(\Lambda a. \Lambda b. \Lambda c. 
        \lambda f^{b \rightarrow c}.
        \lambda g^{a \rightarrow b}.
        \lambda x^a.
          f\;(g\;x))[t][t][t]\;
        \\&\;\;\;\;\;\;\;\;((
          \Lambda t. \lambda f^{t \rightarrow t}. \lambda x^t. f\;x
          )[t]\;f)\;
        \\&\;\;\;\;\;\;\;\;((
          \Lambda t. \lambda f^{t \rightarrow t}. \lambda x^t. 
            f\;(f\;x))[t]\;f) \\
    \leadsto^{5}& 
      \Lambda t.
      \lambda f^{t \rightarrow t}.
        \\&\;\;\;\;\;\;\;\;(
        \lambda f^{t \rightarrow t}.
        \lambda g^{t \rightarrow t}.
        \lambda x^t.
          f\;(g\;x))\;
        \\&\;\;\;\;\;\;\;\;((\lambda f^{t \rightarrow t}. 
          \lambda x^t. f\;x)\;f)\;
        \\&\;\;\;\;\;\;\;\;((\lambda f^{t \rightarrow t}. 
          \lambda x^t. f\;(f\;x))\;f) \\
    \leadsto^{2}& 
      \Lambda t.
      \lambda f^{t \rightarrow t}.
        (
        \lambda f^{t \rightarrow t}.
        \lambda g^{t \rightarrow t}.
        \lambda x^t.
          f\;(g\;x))\;
        \;(\lambda x^t. f\;x)
        \;(\lambda x^t. f\;(f\;x)) \\
    \leadsto^{2}& 
      \Lambda t.
      \lambda f^{t \rightarrow t}.
        \lambda x^t.
        (\lambda x^t. f\;x)\;((\lambda x^t. f\;(f\;x))\;x)
      \\
    \leadsto& 
      \Lambda t.
      \lambda f^{t \rightarrow t}.
        \lambda x^t.
        (\lambda x^t. f\;x)\;(f\;(f\;x))
      \\
    \leadsto& 
      \Lambda t.
      \lambda f^{t \rightarrow t}.
        \lambda x^t. f\;(f\;(f\;x))
      \\
\end{align*}
\endgroup

Finally, this shows how the defined $\mathsf{4}$ 
also reduces to the expected term by also using the previous 
reduction for $\mathsf{2}$:

\begingroup
\allowdisplaybreaks
\begin{align*}
  \mathsf{4} 
    \equiv&
      \Lambda t.
      \lambda f^{t \rightarrow t}.
        \mathsf{COMP}[t][t][t]\;
        (\mathsf{2}[t]\;f)\;
        (\mathsf{2}[t]\;f) \\
    \equiv&
      \Lambda t.
      \lambda f^{t \rightarrow t}.
        \\&\;\;\;\;\;\;\;\;(\Lambda a. \Lambda b. \Lambda c. 
        \lambda f^{b \rightarrow c}.
        \lambda g^{a \rightarrow b}.
        \lambda x^a.
          f\;(g\;x))[t][t][t]\;
        \\&\;\;\;\;\;\;\;\;(\mathsf{2}[t]\;f)\;
        \\&\;\;\;\;\;\;\;\;(\mathsf{2}[t]\;f) \\
    \leadsto^{*}&
      \Lambda t.
      \lambda f^{t \rightarrow t}.
        \\&\;\;\;\;\;\;\;\;(\Lambda a. \Lambda b. \Lambda c. 
        \lambda f^{b \rightarrow c}.
        \lambda g^{a \rightarrow b}.
        \lambda x^a.
          f\;(g\;x))[t][t][t]\;
        \\&\;\;\;\;\;\;\;\;((
          \Lambda t. \lambda f^{t \rightarrow t}. \lambda 
            x^t. f\;(f\;x))[t]\;f)\;
        \\&\;\;\;\;\;\;\;\;((
          \Lambda t. \lambda f^{t \rightarrow t}. \lambda 
            x^t. f\;(f\;x))[t]\;f) \\
    \leadsto^{5}& 
      \Lambda t.
      \lambda f^{t \rightarrow t}.
        \\&\;\;\;\;\;\;\;\;(
        \lambda f^{t \rightarrow t}.
        \lambda g^{t \rightarrow t}.
        \lambda x^t.
          f\;(g\;x))\;
        \\&\;\;\;\;\;\;\;\;((
          \lambda f^{t \rightarrow t}. \lambda 
            x^t. f\;(f\;x))\;f)\;
        \\&\;\;\;\;\;\;\;\;((
          \lambda f^{t \rightarrow t}. \lambda 
            x^t. f\;(f\;x))\;f) \\
    \leadsto^{2}& 
      \Lambda t.
      \lambda f^{t \rightarrow t}.
        (
        \lambda f^{t \rightarrow t}.
        \lambda g^{t \rightarrow t}.
        \lambda x^t.
          f\;(g\;x))\;
        \;(\lambda x^t. f\;(f\;x))
        \;(\lambda x^t. f\;(f\;x)) \\
    \leadsto^{2}& 
      \Lambda t.
      \lambda f^{t \rightarrow t}.
        \lambda x^t.
        (\lambda x^t. f\;(f\;x))\;
          ((\lambda x^t. f\;(f\;x))\;x)
      \\
    \leadsto& 
      \Lambda t.
      \lambda f^{t \rightarrow t}.
        \lambda x^t.
        (\lambda x^t. f\;(f\;x))\;(f\;(f\;x))
      \\
    \leadsto& 
      \Lambda t.
      \lambda f^{t \rightarrow t}.
        \lambda x^t. f\;(f\;(f\;(f\;x)))
      \\
\end{align*}
\endgroup

\subsection{}\label{ex:7}

Revisit Church booleans in $\lambda 2$ (Exercise 5 of the
previous sheet). Use reduction to check the correctness
of your definitions (i.e. of the type \textit{bool} and the
terms $\mathsf{TRUE}, \mathsf{FALSE}, \mathsf{NEG}, 
\mathsf{CONJ}$ and $\textsf{DISJ}$).

\subsubsection{}

The type \textit{bool} is defined as 

$$
\textit{bool} = \forall t. t \rightarrow t \rightarrow t
$$

and the terms $\mathsf{TRUE}$ and $\mathsf{FALSE}$ as

\begin{align*}
  \mathsf{TRUE} &\triangleq 
    \Lambda t. \lambda x^t. \lambda y^t. x \\
  \mathsf{FALSE} &\triangleq 
    \Lambda t. \lambda x^t. \lambda y^t. y \\
\end{align*}

whose proof for its \textit{bool} type assignment is 
the following, from left to right:

\begin{minipage}[t][3.3cm][t]{.4\textwidth}
\begin{mathpar}
  \inferrule*[Right=$\forall i$]{
  \inferrule*[Right=ABS]{
  \inferrule*[Right=ABS]{
  \inferrule*[Right=VAR]{\;}{
    x:t,y:t\vdash x:t
  }
  }{
    x:t\vdash \lambda y^t.x:
      t \rightarrow t
  } 
  }{
    \vdash \lambda x^t.\lambda y^t.x:
      t \rightarrow t \rightarrow t
  }
  }{
    \vdash \Lambda t. \lambda x^t.\lambda y^t.x:
      \forall t. t \rightarrow t \rightarrow t
  }
\end{mathpar}
\end{minipage}
\hfill
\begin{minipage}[t]{.4\textwidth}
\begin{mathpar}
  \inferrule*[Right=$\forall i$]{
  \inferrule*[Right=ABS]{
  \inferrule*[Right=ABS]{
  \inferrule*[Right=VAR]{\;}{
    x:t,y:t\vdash y:t
  }
  }{
    x:t\vdash \lambda y^t.y:
      t \rightarrow t
  } 
  }{
    \vdash \lambda x^t.\lambda y^t.y:
      t \rightarrow t \rightarrow t
  }
  }{
    \vdash \Lambda t. \lambda x^t.\lambda y^t.y:
      \forall t. t \rightarrow t \rightarrow t
  }
\end{mathpar}
\end{minipage}

The term $\mathsf{NOT}$ can be defined as

$$
\lambda b^\textit{bool}.\Lambda t. \lambda x^t.\lambda y^t.
  b[t]\;y\;x
$$

On the one hand, its proof to be assigned type 
$\textit{bool} \rightarrow \textit{bool}$
is as follows:

\begin{mathpar}
  \inferrule*[Right=ABS]{
  \inferrule*[Right=$\forall i$]{
  \inferrule*[Right=ABS]{
  \inferrule*[Right=ABS]{
  \inferrule*[Right=APP]{
  \inferrule*[Left=APP]{
  \inferrule*[Left=$\forall e$]{
  \inferrule*[Left=VAR]{\;}{
    b:\forall t. t \rightarrow t \rightarrow t
    \vdash 
    b:\forall t. t \rightarrow t \rightarrow t
  }
  }{
    b:\forall t. t \rightarrow t \rightarrow t
    \vdash 
    b[t]:t \rightarrow t \rightarrow t
  } \\
  \inferrule*[Right=VAR]{\;}{
    y:t \vdash y:t
  } 
  }{
    b:\forall t. t \rightarrow t \rightarrow t,
    y:t
    \vdash 
    b[t]\;y: t \rightarrow t 
  } \\ 
  \inferrule*[Right=VAR]{\;}{
    x:t \vdash x:t
  }
  }{
    b:\forall t. t \rightarrow t \rightarrow t, 
    x: t, y: t \vdash
      b[t]\;y\;x: t
  }
  }{
    b:\forall t. t \rightarrow t \rightarrow t, 
    x: t \vdash
      \lambda y^t. b[t]\;y\;x: 
        t \rightarrow t
  }
  }{
    b:\forall t. t \rightarrow t \rightarrow t \vdash 
      \lambda x^t.
      \lambda y^t. b[t]\;y\;x:
        t \rightarrow t \rightarrow t
  }
  }{
    b:\forall t. t \rightarrow t \rightarrow t \vdash 
    \Lambda t.
    \lambda x^t.
    \lambda y^t. b[t]\;y\;x:
      \forall t. t \rightarrow t \rightarrow t
  } 
  }{
    \vdash 
      \lambda b^{\forall t. t \rightarrow t \rightarrow t}.
      \Lambda t.
      \lambda x^t.
      \lambda y^t.
        b[t]\;y\;x:
      (\forall t. t \rightarrow t \rightarrow t) 
      \rightarrow 
      (\forall t. t \rightarrow t \rightarrow t)
  }
\end{mathpar}

On the other hand, its behavior is the following:

\begin{align*}
  \mathsf{NEG}\;\mathsf{TRUE} & \\
    \equiv&
      (\lambda b^{\forall t. t \rightarrow t \rightarrow t}.
      \Lambda t.
      \lambda x^t.
      \lambda y^t.
        b[t]\;y\;x
      )\;
      (\Lambda t. \lambda x^t. \lambda y^t. x) \\
    \leadsto&
      \Lambda t.
      \lambda x^t.
      \lambda y^t.
        (\Lambda t. \lambda x^t. \lambda y^t. x)[t]\;y\;x \\
    \leadsto&
      \Lambda t.
      \lambda x^t.
      \lambda y^t.
      (\lambda x^t. \lambda y^t. x)\;y\;x \\
    \leadsto&
      \Lambda t.
      \lambda x^t.
      \lambda y^t. y \\
    \equiv& \mathsf{FALSE}
\end{align*}

\begin{align*}
  \mathsf{NEG}\;\mathsf{FALSE} & \\
    \equiv&
      (\lambda b^{\forall t. t \rightarrow t \rightarrow t}.
      \Lambda t.
      \lambda x^t.
      \lambda y^t.
        b[t]\;y\;x
      )\;
      (\Lambda t. \lambda x^t. \lambda y^t. y) \\
    \leadsto&
      \Lambda t.
      \lambda x^t.
      \lambda y^t.
        (\Lambda t. \lambda x^t. \lambda y^t. y)[t]\;y\;x \\
    \leadsto&
      \Lambda t.
      \lambda x^t.
      \lambda y^t.
      (\lambda x^t. \lambda y^t. y)\;y\;x \\
    \leadsto&
      \Lambda t.
      \lambda x^t.
      \lambda y^t. x \\
    \equiv& \mathsf{TRUE}
\end{align*}

The term $\mathsf{CONJ}$ can be defined as

$$
\lambda b^\textit{bool}. \lambda c^\textit{bool}.
  \Lambda t. \lambda x^t.\lambda y^t.
  b[t]\;(c[t]\;x\;y)\;y
$$

On the one hand, its proof to be assigned type 
$\textit{bool} \rightarrow \textit{bool}
\rightarrow \textit{bool}$ is as follows:

\begin{changemargin}{-2cm}{-2cm}
\begin{mathpar}
  \inferrule*[Right=ABS]{
  \inferrule*[Right=ABS]{
  \inferrule*[Right=$\forall i$]{
  \inferrule*[Right=ABS]{
  \inferrule*[Right=ABS]{
  \inferrule*[Right=APP]{
  \inferrule*[Left=APP]{
  \inferrule*[Left=$\forall e$]{
  \inferrule*[Left=VAR]{\;}{
    b:\textit{bool} \vdash b:\textit{bool}
  }
  }{
    b:\textit{bool}
    \vdash
    b[t]:t \rightarrow t \rightarrow t
  } \\ 
  \Delta_1
  }{
    b: \textit{bool},
    c: \textit{bool},
    x: t,
    y: t
    \vdash
    b\;(c\;x\;y):t\rightarrow t
  } \\
  \inferrule*[Right=VAR]{\;}{
    y:t
    \vdash
    y:t
  }
  }{
    b: \textit{bool}, c: \textit{bool}, x: t, y: t \vdash 
      b[t]\;(c[t]\;x\;y)\;y: t
  }
  }{
    b: \textit{bool}, c: \textit{bool}, x: t \vdash 
      \lambda y^t.
      b[t]\;(c[t]\;x\;y)\;y: t \rightarrow t
  }
  }{
    b: \textit{bool}, c: \textit{bool} \vdash 
      \lambda x^t.
      \lambda y^t.
      b[t]\;(c[t]\;x\;y)\;y:
        t \rightarrow t \rightarrow t
  }
  }{
    b: \textit{bool}, c: \textit{bool} \vdash 
      \Lambda t.
      \lambda x^t.
      \lambda y^t.
      b[t]\;(c[t]\;x\;y)\;y:
      \textit{bool}
  }
  }{
    b: \textit{bool} \vdash 
      \lambda c^\textit{bool}.
      \Lambda t.
      \lambda x^t.
      \lambda y^t.
      b[t]\;(c[t]\;x\;y)\;y:
      \textit{bool} \rightarrow
      \textit{bool}
  } 
  }{
    \vdash 
      \lambda b^\textit{bool}.
      \lambda c^\textit{bool}.
      \Lambda t.
      \lambda x^t.
      \lambda y^t.
      b[t]\;(c[t]\;x\;y)\;y:
      \textit{bool} \rightarrow
      \textit{bool} \rightarrow
      \textit{bool}
  }
\end{mathpar}
\end{changemargin}

where the subtree $\Delta_1$ is the following:

\begin{mathpar}
\inferrule*[Right=APP]{
\inferrule*[Left=APP]{
\inferrule*[Left=$\forall e$]{
\inferrule*[Left=VAR]{\;}{
  c: \textit{bool}
  \vdash 
  c: \textit{bool}
}
}{
  c: \textit{bool}
  \vdash 
  c[t]: t \rightarrow t \rightarrow t
} \\ 
\inferrule*[Left=VAR]{\;}{
  x:t
  \vdash 
  x:t
}
}{
  c: \textit{bool},
  x: t
  \vdash
  c[t]\;x:t \rightarrow t
} \\
\inferrule*[Left=VAR]{\;}{
  y:t
  \vdash
  y:t
}
}{
  c: \textit{bool},
  x: t,
  y: t
  \vdash
  c[t]\;x\;y:t
}
\end{mathpar}

On the other hand, its behavior is the following:

\begin{align*}
  \mathsf{CONJ}\;\mathsf{FALSE} & \\
    \equiv&
      (\lambda b^\textit{bool}.
      \lambda c^\textit{bool}.
      \Lambda t.
      \lambda x^t.
      \lambda y^t.
      b[t]\;(c[t]\;x\;y)\;y)\;
      (\Lambda t. \lambda x^t.\lambda y^t.y) \\
    \leadsto&
      \lambda c^\textit{bool}.
      \Lambda t.
      \lambda x^t.
      \lambda y^t.
      (\Lambda t. \lambda x^t.\lambda y^t.y)[t]\;(c[t]\;x\;y)\;y \\
    \leadsto&
      \lambda c^\textit{bool}.
      \Lambda t.
      \lambda x^t.
      \lambda y^t.
      (\lambda x^t.\lambda y^t.y)\;(c[t]\;x\;y)\;y \\
    \leadsto^{2}&
      \lambda c^\textit{bool}.
      \Lambda t.
      \lambda x^t.
      \lambda y^t.
      y \\
    \equiv&
      \lambda c^\textit{bool}.\mathsf{FALSE}
\end{align*}

\begin{align*}
  \mathsf{CONJ}\;\mathsf{TRUE} & \\
    \equiv&
      (\lambda b^\textit{bool}.
      \lambda c^\textit{bool}.
      \Lambda t.
      \lambda x^t.
      \lambda y^t.
      b[t]\;(c[t]\;x\;y)\;y)\;
      (\Lambda t. \lambda x^t.\lambda y^t.x) \\
    \leadsto&
      \lambda b^\textit{bool}.
      \lambda c^\textit{bool}.
      \Lambda t.
      \lambda x^t.
      \lambda y^t.
      (\Lambda t. \lambda x^t.\lambda y^t.x)[t]\;(c[t]\;x\;y)\;y \\
    \leadsto&
      \lambda b^\textit{bool}.
      \lambda c^\textit{bool}.
      \Lambda t.
      \lambda x^t.
      \lambda y^t.
      (\lambda x^t.\lambda y^t.x)\;(c[t]\;x\;y)\;y \\
    \leadsto^{2}&
      \lambda b^\textit{bool}.
      \lambda c^\textit{bool}.
      \Lambda t.
      \lambda x^t.
      \lambda y^t.
      c[t]\;x\;y \\
    \rightsquigarrow_\eta^2&
      \lambda c^\textit{bool}.
      \Lambda t. c[t]
\end{align*}

The reduction for $\mathsf{CONJ}\;\mathsf{TRUE}$ shows that 
it will reduce to the same \textit{bool} term that can 
be applied to it, although we cannot reduce it more unless 
we apply the remaining parameter. Maybe a similar reduction 
rule as the known $\eta$-reduction for terms can be defined 
for type abstractions and applications.

The term $\mathsf{DISJ}$ can be defined as

$$
\lambda b^\textit{bool}. \lambda c^\textit{bool}.
  \Lambda t. \lambda x^t.\lambda y^t.
  b[t]\;x\;(c[t]\;x\;y)
$$

On the one hand, its proof to be assigned type 
$\textit{bool} \rightarrow \textit{bool}
\rightarrow \textit{bool}$ is as follows:

\begin{changemargin}{-2cm}{-2cm}
\begin{mathpar}
  \inferrule*[Right=ABS]{
  \inferrule*[Right=ABS]{
  \inferrule*[Right=$\forall i$]{
  \inferrule*[Right=ABS]{
  \inferrule*[Right=ABS]{
  \inferrule*[Right=APP]{
  \Delta_2
  \\
  \inferrule*[Right=APP]{
  \inferrule*[Right=APP]{
  \inferrule*[Left=$\forall e$]{
  \inferrule*[Left=VAR]{\;}{
    c: \textit{bool} \vdash c: \textit{bool}
  }
  }{
    c: \textit{bool}
    \vdash
    c[t]: t \rightarrow t \rightarrow t
  } \\ 
  \inferrule*[Left=VAR]{\;}{
    x:t
    \vdash
    x:t
  }
  }{
    c: \textit{bool},
    x: t
    \vdash
    c[t]\;x:t \rightarrow t
  } \\
  \inferrule*[Right=VAR]{\;}{
    y:d
    \vdash
    y:d
  }
  }{
    c: \textit{bool},
    x: t,
    y: t
    \vdash
    c[t]\;x\;y:t
  }
  }{
    b: \textit{bool}, c: \textit{bool}, x: t, y: t \vdash
      b[t]\;x\;(c[t]\;x\;y): t
  }
  }{
    b: \textit{bool}, c: \textit{bool}, x: t \vdash
      \lambda y^t.
      b[t]\;x\;(c[t]\;x\;y):
      t \rightarrow t
  }
  }{
    b: \textit{bool}, c: \textit{bool} \vdash
      \lambda x^t.
      \lambda y^t.
      b[t]\;x\;(c[t]\;x\;y):
      t \rightarrow t \rightarrow t
  }
  }{
    b: \textit{bool}, c: \textit{bool} \vdash
      \Lambda t.
      \lambda x^t.
      \lambda y^t.
      b[t]\;x\;(c[t]\;x\;y):
      \textit{bool}
  }
  }{
    b: \textit{bool} \vdash
      \lambda c^\textit{bool}.
      \Lambda t.
      \lambda x^t.
      \lambda y^t.
      b[t]\;x\;(c[t]\;x\;y):
      \textit{bool} \rightarrow
      \textit{bool}
  } 
  }{
    \vdash 
      \lambda b^\textit{bool}.
      \lambda c^\textit{bool}.
      \Lambda t.
      \lambda x^t.
      \lambda y^t.
      b[t]\;x\;(c[t]\;x\;y):
      \textit{bool} \rightarrow
      \textit{bool} \rightarrow
      \textit{bool}
  }
\end{mathpar}
\end{changemargin}

where the subtree $\Delta_2$ is the following:

\begin{mathpar}
\inferrule*[Left=APP]{
\inferrule*[Left=$\forall e$]{
\inferrule*[Left=VAR]{\;}{
  b: \textit{bool} \vdash b: \textit{bool}
}
}{
  b: \textit{bool}
  \vdash
  b[t]:t \rightarrow t \rightarrow t
} \\
\inferrule*[Left=VAR]{\;}{
  x:t
  \vdash
  x:t
} \\
}{
  b: \textit{bool},
  x: t
  \vdash
  b[t]\;x: t \rightarrow t
}
\end{mathpar}

On the other hand, its behavior is the following:

\begin{align*}
  \mathsf{DISJ}\;\mathsf{TRUE} & \\
    \equiv&
      (\lambda b^\textit{bool}.
      \lambda c^\textit{bool}.
      \Lambda t.
      \lambda x^t.
      \lambda y^t.
      b[t]\;x\;(c[t]\;x\;y)\;
      (\Lambda t. \lambda x^t.\lambda y^t.x) \\
    \leadsto&
      \lambda c^\textit{bool}.
      \Lambda t.
      \lambda x^t.
      \lambda y^t.
      (\Lambda t. \lambda x^t.\lambda y^t.x)[t]\;x\;(c[t]\;x\;y) \\
    \leadsto&
      \lambda c^\textit{bool}.
      \Lambda t.
      \lambda x^t.
      \lambda y^t.
      (\lambda x^t.\lambda y^t.x)\;x\;(c[t]\;x\;y) \\
    \leadsto^{2}&
      \lambda c^\textit{bool}.
      \Lambda t.
      \lambda x^t.
      \lambda y^t.
      x \\
    \equiv&
      \lambda c^\textit{bool}.\mathsf{TRUE}
\end{align*}

\begin{align*}
  \mathsf{DISJ}\;\mathsf{FALSE} & \\
    \equiv&
      (\lambda b^\textit{bool}.
      \lambda c^\textit{bool}.
      \Lambda t.
      \lambda x^t.
      \lambda y^t.
      b[t]\;x\;(c[t]\;x\;y)\;
      (\Lambda t. \lambda x^t.\lambda y^t.y) \\
    \leadsto&
      \lambda b^\textit{bool}.
      \lambda c^\textit{bool}.
      \Lambda t.
      \lambda x^t.
      \lambda y^t.
      (\Lambda t. \lambda x^t.\lambda y^t.y)[t]\;x\;(c[t]\;x\;y) \\
    \leadsto&
      \lambda b^\textit{bool}.
      \lambda c^\textit{bool}.
      \Lambda t.
      \lambda x^t.
      \lambda y^t.
      (\lambda x^t.\lambda y^t.y)\;x\;(c[t]\;x\;y) \\
    \leadsto^{2}&
      \lambda b^\textit{bool}.
      \lambda c^\textit{bool}.
      \Lambda t.
      \lambda x^t.
      \lambda y^t.
      c[t]\;x\;y \\
    \rightsquigarrow_\eta^2&
      \lambda c^\textit{bool}.
      \Lambda t. c[t]
\end{align*}

As mentioned in the $\mathsf{CONJ}\;\mathsf{TRUE}$ reduction,
this last one will reduce to the same \textit{bool} term 
that can be applied to it, but it could be seen more clearly 
if we had a new rule of $\eta$-reduction for type 
abstractions and applications.

\subsection{}\label{ex:8}

Implement the two previous exercises in Haskell using the 
extension for \textit{rank n polymorphic types}. 

\subsubsection{}

The implementation is provided as a Haskell project
managed with Cabal within the folder \verb|church-encodings|.

Although it contains implementations for booleans, numerals, 
pairs and lists, the ones of interest for this exercise 
correspond to the modules \verb|Encoding.BoolChurch|
and \verb|Encoding.NatChurch|.

A test suite is also available, which can be executed with the 
script \verb|test.sh| as follows:

\begin{verbatim}
chmod u+x test.sh && ./test.sh
\end{verbatim}

\end{document}
