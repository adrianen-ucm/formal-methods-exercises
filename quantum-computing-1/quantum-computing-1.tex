\documentclass{article}
\usepackage[utf8]{inputenc}
\usepackage[spanish]{babel}
\usepackage{mathtools}
\usepackage{amsfonts}
\usepackage{bbold}

\title{Ejercicios de introducción a la física cuántica}
\author{Adrián Enríquez Ballester}
\date{\today}

\begin{document}
\maketitle

\subsection*{Ejercicio 1}

Demostrar que una matriz normal $U$ es unitaria si y sólo
si existe una matriz hermítica $H$ tal que $U = \text{exp}
(iH)$.

\subsubsection*{Respuesta}

Para demostrar la primera implicación, supongamos que $U$ 
es unitaria. Como esta es normal, sea $n$ el número de 
columnas de $U$, existe una base $\{v_j\}_{j=1}^n$ 
ortonormal de $\mathbb{C}^n$ donde cada $v_j$ es un 
autovector con autovalor $\lambda_j$ de $U$, y por tanto

$$
U = \sum_{j=1}^n \lambda_j |v_j \rangle\langle v_j|
$$

Sea $H$ la matriz definida como 

$$
  H \coloneqq \sum_{j=1}^n \frac{\text{ln}(\lambda_j)}{i} 
    |v_j \rangle\langle v_j|
$$

podemos comprobar que

$$
\text{exp}(iH) 
    = \sum_{j=1}^n e^{i\text{ln}(\lambda_j)/i}
      |v_j \rangle\langle v_j|
    = \sum_{j=1}^n e^{\text{ln}(\lambda_j)}
      |v_j \rangle\langle v_j|
    = \sum_{j=1}^n \lambda_j
      |v_j \rangle\langle v_j|
    = U
$$

por lo que sólo quedaría verificar que $H$ es hermítica.

Sabemos que 
$
\text{ln}(\lambda_j) = \ln(|\lambda_j|) + i\theta_j
  \text{ para un valor } \theta_j \in (-\pi, \pi] 
$, y también que $U$ es normal y por tanto sus autovalores 
han de tener módulo $1$, lo cual implica que 

$$
  \frac{\text{ln}(\lambda_j)}{i}
    = \frac{\text{ln}(|\lambda_j|) + i\theta_j}{i}
    = \frac{\text{ln}(1) + i\theta_j}{i}
    = \frac{i\theta_j}{i}
    = \theta_j
$$

y, al tratarse de valores en $\mathbb{R}$, permite comprobar que
$H$ es hermítica:

$$
  H^\dagger 
    = \sum_{j=1}^n \overline{\theta_j} 
        (|v_j \rangle\langle v_j|)^\dagger
    = \sum_{j=1}^n \theta_j 
      |v_j \rangle\langle v_j|
    = H
$$

En cuanto a la implicación restante, supongamos que existe 
una matriz $H$ hermítica tal que $U = \text{exp}(iH)$.

El hecho de que $H$ sea hermítica implica que también 
es normal, por lo que existe una base $\{v_j\}_{j=1}^n$ 
ortonormal de $\mathbb{C}^n$ donde cada $v_j$ es un 
autovector con autovalor $\theta_j$ de $H$, y por tanto

$$
  H = \sum_{j=1}^n \theta_j |v_j \rangle\langle v_j|
$$

Además, como $H$ es hermítica, sabemos que cada uno de
estos autovalores $\theta_j$ ha de estar en $\mathbb{R}$.
También tenemos que

$$
  U = \text{exp}(iH) 
    = \sum_{j=1}^n e^{i \theta_j} |v_j \rangle\langle v_j|
$$

con lo cual podemos proceder a comprobar que $U$ es unitaria:

\begin{align*}
  UU^\dagger &= \Big(
      \sum_{j=1}^n e^{i \theta_j} 
        |v_j \rangle\langle v_j|
    \Big) \Big(
      \sum_{k=1}^n \overline{e^{i \theta_k}} 
        (|v_k \rangle\langle v_k|)^\dagger
      \Big) \\
    &= \sum_{j,k=1}^n e^{i \theta_j}\overline{e^{i \theta_k}} 
        |v_j \rangle\langle v_j|v_k \rangle \langle v_k| \\
    &= \sum_{j,k=1}^n e^{i \theta_j}\overline{e^{i \theta_k}} 
        |v_j \rangle \delta_{jk} \langle v_k| \\
    &= \sum_{j=1}^n e^{i \theta_j}\overline{e^{i \theta_j}} 
        |v_j \rangle \langle v_j| \\
    &= \sum_{j=1}^n |e^{i \theta_j}|^2 
        |v_j \rangle \langle v_j| \\
    &= \sum_{j=1}^n |v_j \rangle \langle v_j| \\
    &= \mathbb{1}
\end{align*}

puesto que para cada $j$ se cumple
$
|e^{i\theta_j}|^2
  = |\text{cos}(\theta_j) + i\text{sin}(\theta_j)|^2
  = \text{cos}^2(\theta_j) + \text{sin}^2(\theta_j)
  = 1
$.

\subsection*{Ejercicio 2}

¿Cuál es la probabilidad de obtener cada salida, y el estado
después de la medida si el estado inicial 
(de dos partes A y B)
es $|\Phi_+ \rangle_{AB} = \frac{1}{\sqrt{2}}(|00\rangle + 
|11\rangle)$, y tanto Alice como Bob miden en la base 
$|+\rangle$, $|-\rangle$?

\subsubsection*{Respuesta}

Las matrices con las que se está midiendo en el sistema
compuesto son:

\begin{align*}
  P_{++} &= |+\rangle\langle+| \otimes |+\rangle\langle+| \\
  P_{--} &= |-\rangle\langle-| \otimes |-\rangle\langle-| \\
  P_{+-} &= |+\rangle\langle+| \otimes |-\rangle\langle-| \\
  P_{-+} &= |-\rangle\langle-| \otimes |+\rangle\langle+| \\
\end{align*}

Al tratarse de medidas proyectivas, podemos calcular la 
probabilidad de que salga $++$ como 

\begin{align*}
  \text{p}(++) &= \langle \Phi_+ | P_{++} |\Phi_+ \rangle \\ 
    &= \frac{1}{2}
        \Big( \langle 00| + \langle 11| \Big)
        P_{++}
        \Big(|00\rangle + |11\rangle\Big) \\
    &= \frac{1}{2}\Big(  
      \langle 00|P_{++}|00\rangle + \langle 00|P_{++}|11\rangle +
      \langle 11|P_{++}|00\rangle + \langle 11|P_{++}|11\rangle
    \Big) \\
    &= \frac{1}{2}\Big(
      \frac{1}{4} + \frac{1}{4} + \frac{1}{4} + \frac{1}{4}
    \Big) \\ 
    &= \frac{1}{2}
\end{align*}

puesto que 

\begin{align*}
  \langle 00|P_{++}|00\rangle 
    &= \langle 0|+\rangle\langle +|0\rangle
      \otimes \langle 0|+\rangle\langle +|0\rangle
    = \frac{1}{\sqrt{2}}\frac{1}{\sqrt{2}}
      \otimes \frac{1}{\sqrt{2}}\frac{1}{\sqrt{2}}
    = \frac{1}{4} \\
  \langle 00|P_{++}|11\rangle 
    &= \langle 0|+\rangle\langle +|1\rangle
      \otimes \langle 0|+\rangle\langle +|1\rangle
    = \frac{1}{\sqrt{2}}\frac{1}{\sqrt{2}}
      \otimes \frac{1}{\sqrt{2}}\frac{1}{\sqrt{2}}
    = \frac{1}{4} \\
  \langle 11|P_{++}|00\rangle 
    &= \langle 1|+\rangle\langle +|0\rangle
      \otimes \langle 1|+\rangle\langle +|0\rangle
    = \frac{1}{\sqrt{2}}\frac{1}{\sqrt{2}}
      \otimes \frac{1}{\sqrt{2}}\frac{1}{\sqrt{2}}
    = \frac{1}{4} \\
  \langle 11|P_{++}|11\rangle 
    &= \langle 1|+\rangle\langle +|1\rangle
      \otimes \langle 1|+\rangle\langle +|1\rangle
    = \frac{1}{\sqrt{2}}\frac{1}{\sqrt{2}}
      \otimes \frac{1}{\sqrt{2}}\frac{1}{\sqrt{2}}
    = \frac{1}{4} \\
\end{align*}

En caso de que saliese $++$, el estado final sería

\begin{align*}
  |\psi_f\rangle &= \frac{P_{++}|\Phi_+\rangle}{\sqrt{p(++)}} \\ 
    &= \sqrt{2} \frac{1}{\sqrt{2}} 
      P_{++}\Big(|00\rangle + |11\rangle\Big) \\
    &= \Big( 
      |+\rangle\langle+|\otimes|+\rangle\langle+| 
    \Big)|00\rangle + 
    \Big( 
      |+\rangle\langle+|\otimes|+\rangle\langle+| 
    \Big)|11\rangle \\
    &= \Big(
      |+\rangle\langle +|0\rangle \otimes 
      |+\rangle\langle +|0\rangle\Big
    ) +
    \Big(
      |+\rangle\langle +|1\rangle \otimes 
      |+\rangle\langle +|1\rangle
    \Big)\\
    &= \Big(
      |+\rangle\frac{1}{\sqrt{2}} \otimes 
      |+\rangle\frac{1}{\sqrt{2}}
    \Big) +
    \Big(
      |+\rangle\frac{1}{\sqrt{2}} \otimes 
      |+\rangle\frac{1}{\sqrt{2}}
    \Big) \\
    &= \frac{1}{2}|++\rangle +
      \frac{1}{2}|++\rangle \\
    &= |++\rangle
\end{align*}

Por otra parte, la probabilidad de que salga $--$ es 

\begin{align*}
  \text{p}(--) &= \langle \Phi_+ | P_{--}|\Phi_+ \rangle \\ 
    &= \frac{1}{2}\Big( \langle 00| + \langle 11| \Big)
        P_{--}
        \Big(|00\rangle + |11\rangle\Big) \\
    &= \frac{1}{2}\Big(  
      \langle 00|P_{--}|00\rangle + \langle 00|P_{--}|11\rangle +
      \langle 11|P_{--}|00\rangle + \langle 11|P_{--}|11\rangle
    \Big) \\
    &= \frac{1}{2}\Big(
      \frac{1}{4} + \frac{1}{4} + \frac{1}{4} + \frac{1}{4}
    \Big) \\ 
    &= \frac{1}{2}
\end{align*}

puesto que 

\begin{align*}
  \langle 00|P_{--}|00\rangle 
    &= \langle 0|-\rangle\langle -|0\rangle
      \otimes \langle 0|-\rangle\langle -|0\rangle
    = \frac{1}{\sqrt{2}}\frac{1}{\sqrt{2}}
      \otimes \frac{1}{\sqrt{2}}\frac{1}{\sqrt{2}}
    = \frac{1}{4} \\
  \langle 00|P_{--}|11\rangle 
    &= \langle 0|-\rangle\langle -|1\rangle
      \otimes \langle 0|-\rangle\langle -|1\rangle
    = \frac{1}{\sqrt{2}}(-\frac{1}{\sqrt{2}})
      \otimes \frac{1}{\sqrt{2}}(-\frac{1}{\sqrt{2}})
    = \frac{1}{4} \\
  \langle 11|P_{--}|00\rangle 
    &= \langle 1|-\rangle\langle -|0\rangle
      \otimes \langle 1|-\rangle\langle -|0\rangle
    = (-\frac{1}{\sqrt{2}})\frac{1}{\sqrt{2}}
      \otimes (-\frac{1}{\sqrt{2}})\frac{1}{\sqrt{2}}
    = \frac{1}{4} \\
  \langle 11|P_{--}|11\rangle 
    &= \langle 1|-\rangle\langle -|1\rangle
      \otimes \langle 1|-\rangle\langle -|1\rangle
    = (-\frac{1}{\sqrt{2}})(-\frac{1}{\sqrt{2}})
      \otimes 
      (-\frac{1}{\sqrt{2}})(-\frac{1}{\sqrt{2}})
    = \frac{1}{4} \\
\end{align*}

El estado final en este caso sería

\begin{align*}
  |\psi_f\rangle &= \frac{P_{--}|\Phi_+\rangle}{\sqrt{p(--)}} \\ 
    &= \sqrt{2} \frac{1}{\sqrt{2}} 
      P_{--}\Big(|00\rangle + |11\rangle\Big) \\
    &= \Big( 
      |-\rangle\langle-|\otimes|-\rangle\langle-| 
    \Big)|00\rangle + 
    \Big( 
      |-\rangle\langle-|\otimes|-\rangle\langle-| 
    \Big)|11\rangle \\
    &= \Big(
      |-\rangle\langle -|0\rangle \otimes 
      |-\rangle\langle -|0\rangle
    \Big) + \Big(
      |-\rangle\langle -|1\rangle \otimes 
      |-\rangle\langle -|1\rangle 
    \Big)\\
    &= \Big(
      |-\rangle\frac{1}{\sqrt{2}}\otimes 
      |-\rangle\frac{1}{\sqrt{2}} 
    \Big) + \Big(
      |-\rangle(-\frac{1}{\sqrt{2}}) \otimes 
      |-\rangle(-\frac{1}{\sqrt{2}}) 
    \Big) \\
    &= \frac{1}{2}|--\rangle +
      \frac{1}{2}|--\rangle \\
    &= |--\rangle
\end{align*}

Por último, en cuanto al resto de medidas, ambas tienen 
probabilidad nula de salir:

\begin{align*}
  \text{p}(+-) &= \langle \Phi_+ | P_{+-} |\Phi_+ \rangle \\ 
    &= \frac{1}{2}\Big( \langle 00| + \langle 11| \Big)
        P_{+-}
        \Big(|00\rangle + |11\rangle\Big) \\
    &= \frac{1}{2}\Big(  
      \langle 00|P_{+-}|00\rangle + \langle 00|P_{+-}|11\rangle +
      \langle 11|P_{+-}|00\rangle + \langle 11|P_{+-}|11\rangle
    \Big) \\
    &= \frac{1}{2}\Big(
      \frac{1}{4} - \frac{1}{4} - \frac{1}{4} + \frac{1}{4}
    \Big) \\ 
    &= 0 \\
  \text{p}(-+) &= \langle \Phi_+ | P_{-+} |\Phi_+ \rangle \\ 
    &= \frac{1}{2}\Big( \langle 00| + \langle 11| \Big)
        P_{-+}
        \Big(|00\rangle + |11\rangle\Big) \\
    &= \frac{1}{2}\Big(  
      \langle 00|P_{-+}|00\rangle + \langle 00|P_{-+}|11\rangle +
      \langle 11|P_{-+}|00\rangle + \langle 11|P_{-+}|11\rangle
    \Big) \\
    &= \frac{1}{2}\Big(
      \frac{1}{4} - \frac{1}{4} - \frac{1}{4} + \frac{1}{4}
    \Big) \\ 
    &= 0 \\
\end{align*}

puesto que

\begin{align*}
  \langle 00|P_{+-}|00\rangle 
    &= \langle 0|+\rangle\langle +|0\rangle
      \otimes \langle 0|-\rangle\langle -|0\rangle
    = \frac{1}{\sqrt{2}}\frac{1}{\sqrt{2}}
      \otimes \frac{1}{\sqrt{2}}\frac{1}{\sqrt{2}}
    = \frac{1}{4} \\
  \langle 00|P_{+-}|11\rangle 
    &= \langle 0|+\rangle\langle +|1\rangle
      \otimes \langle 0|-\rangle\langle -|1\rangle
    = \frac{1}{\sqrt{2}}\frac{1}{\sqrt{2}}
      \otimes \frac{1}{\sqrt{2}}(-\frac{1}{\sqrt{2}})
    = -\frac{1}{4} \\
  \langle 11|P_{+-}|00\rangle 
    &= \langle 1|+\rangle\langle +|0\rangle
      \otimes \langle 1|-\rangle\langle -|0\rangle
    = \frac{1}{\sqrt{2}}\frac{1}{\sqrt{2}}
      \otimes (-\frac{1}{\sqrt{2}})\frac{1}{\sqrt{2}}
    = -\frac{1}{4} \\
  \langle 11|P_{+-}|11\rangle 
    &= \langle 1|+\rangle\langle +|1\rangle
      \otimes \langle 1|-\rangle\langle -|1\rangle
    = \frac{1}{\sqrt{2}}\frac{1}{\sqrt{2}}
      \otimes (-\frac{1}{\sqrt{2}})(-\frac{1}{\sqrt{2}})
    = \frac{1}{4} \\
\end{align*}

y

\begin{align*}
  \langle 00|P_{-+}|00\rangle 
    &= \langle 0|-\rangle\langle -|0\rangle
      \otimes \langle 0|+\rangle\langle +|0\rangle
    = \frac{1}{\sqrt{2}}\frac{1}{\sqrt{2}}
      \otimes \frac{1}{\sqrt{2}}\frac{1}{\sqrt{2}}
    = \frac{1}{4} \\
  \langle 00|P_{-+}|11\rangle 
    &= \langle 0|-\rangle\langle -|1\rangle
      \otimes \langle 0|+\rangle\langle +|1\rangle
    = \frac{1}{\sqrt{2}}(-\frac{1}{\sqrt{2}})
      \otimes \frac{1}{\sqrt{2}}\frac{1}{\sqrt{2}}
    = -\frac{1}{4} \\
  \langle 11|P_{-+}|00\rangle 
    &= \langle 1|-\rangle\langle -|0\rangle
      \otimes \langle 1|+\rangle\langle +|0\rangle
    = (-\frac{1}{\sqrt{2}})\frac{1}{\sqrt{2}}
      \otimes \frac{1}{\sqrt{2}}\frac{1}{\sqrt{2}}
    = -\frac{1}{4} \\
  \langle 11|P_{-+}|11\rangle 
    &= \langle 1|-\rangle\langle -|1\rangle
      \otimes \langle 1|+\rangle\langle +|1\rangle
    = (-\frac{1}{\sqrt{2}})(-\frac{1}{\sqrt{2}})
      \otimes \frac{1}{\sqrt{2}}\frac{1}{\sqrt{2}}
    = \frac{1}{4} \\
\end{align*}

\subsection*{Ejercicio 3}

Partiendo del estado de dos partes A y B, 
$|\Phi_+\rangle_{AB} = \frac{1}{\sqrt{2}}(|00\rangle + |11\rangle)$,
encontrar unitarias locales de Alice que lo transformen en cada
uno de los otros elementos de la base de Bell.

\subsubsection*{Respuesta}

Sea $U$ una matriz unitaria, el resultado de aplicarla localmente
para Alice se calcularía como 

\begin{align*}
  |\psi_f\rangle 
    &= \Big(U \otimes \mathbb{1}\Big)|\Phi_+\rangle
    = \frac{1}{\sqrt{2}}\Big(
      \big(U \otimes \mathbb{1}\big)|00\rangle +
      \big(U \otimes \mathbb{1}\big)|11\rangle
    \Big) \\
    &= \frac{1}{\sqrt{2}}\Big(
      U|0\rangle \otimes \mathbb{1} |0\rangle +
      U|1\rangle \otimes \mathbb{1} |1\rangle
    \Big)
    = \frac{1}{\sqrt{2}}\Big(
      U|0\rangle \otimes |0\rangle +
      U|1\rangle \otimes |1\rangle
    \Big) 
\end{align*}

Si queremos que el estado final sea $|\Phi_+\rangle$,
se debería de cumplir lo siguiente:

$$
  U|0\rangle \otimes |0\rangle +
    U|1\rangle \otimes |1\rangle = |00\rangle + |11\rangle
$$

Es decir,

$$
  \begin{cases}
    U|0\rangle = |0\rangle &\\
    U|1\rangle = |1\rangle &\\
  \end{cases}
$$

y sabemos que la matriz identidad $\mathbb{1}$ es unitaria y 
lo cumple.

Por otra parte, si queremos que el estado final sea $\Phi_-$,
se debería de cumplir lo siguiente: 

$$
  U|0\rangle \otimes |0\rangle +
    U|1\rangle \otimes |1\rangle = |00\rangle - |11\rangle
$$

Es decir,

$$
  \begin{cases}
    U|0\rangle = |0\rangle &\\
    U|1\rangle = -|1\rangle &\\
  \end{cases}
$$

lo cual, si expresamos la matriz $U$ como $(u_{ij})_{i,j=1}^2$
y desarrollamos las ecuaciones, determina $U$ como la matriz de 
Pauli $\sigma_z$.

Por otra parte, si queremos que el estado final sea $\Psi_+$,
se debería de cumplir lo siguiente: 

$$
  U|0\rangle \otimes |0\rangle +
    U|1\rangle \otimes |1\rangle = |01\rangle + |10\rangle
$$

Es decir,

$$
  \begin{cases}
    U|0\rangle = |1\rangle &\\
    U|1\rangle = |0\rangle &\\
  \end{cases}
$$

lo cual, si expresamos de nuevo la matriz $U$ como 
$(u_{ij})_{i,j=1}^2$ y desarrollamos las ecuaciones, determina $U$ 
como la matriz de Pauli $\sigma_x$.

Por último, si queremos que el estado final sea $\Psi_-$,
se debería de cumplir lo siguiente: 

$$
  U|0\rangle \otimes |0\rangle +
    U|1\rangle \otimes |1\rangle = |01\rangle - |10\rangle
$$

Es decir,

$$
  \begin{cases}
    U|0\rangle = -|1\rangle &\\
    U|1\rangle = |0\rangle &\\
  \end{cases}
$$

lo cual, si expresamos de nuevo la matriz $U$ como 
$(u_{ij})_{i,j=1}^2$ y desarrollamos las ecuaciones, determina $U$ 
como la matriz

$$
\begin{pmatrix}
   0 & 1 \\ 
  -1 & 0 \\
\end{pmatrix}
$$

\section*{Ejercicio 4}

Tenemos un sistema de tres qubits (1, 2 y 3), que está en el 
estado inicial $|0\rangle_1 \otimes |\Phi_+\rangle_{23}$.
Supongamos que Alice tiene los qubits 1 y 2 y mide ese 
sistema formado por dos qubits en la base de Bell. (Bob tiene 
el qubit 3 pero no hace nada). ¿Cuál es la probabilidad de obtener 
cada salida, y el estado final después de la medida?

\subsubsection*{Respuesta}

Las matrices con las que mide Alice son 

\begin{align*}
  P_{\Phi_+} &= |\Phi_+\rangle\langle\Phi_+| \\
  P_{\Phi_-} &= |\Phi_-\rangle\langle\Phi_-| \\
  P_{\Psi_+} &= |\Psi_+\rangle\langle\Psi_+| \\
  P_{\Psi_-} &= |\Psi_-\rangle\langle\Psi_-| \\
\end{align*}

y, como sabemos que la base de Bell es ortonormal, podemos calcular
la probabilidad de que salga $\Phi_+$ como

\begin{align*}
  p(\Phi_+) 
    &= \Big(\langle 0| \otimes \langle \Phi_+|\Big)
       \Big(|\Phi_+\rangle\langle\Phi_+|\otimes \mathbb{1}\Big)  
       \Big(|0\rangle \otimes |\Phi_+\rangle\Big) \\
    &= \frac{1}{2}
    \Big(\langle 000| + \langle 011|\Big)
    \Big(|\Phi_+\rangle\langle \Phi_+| \otimes \mathbb{1}\Big)
    \Big(|000\rangle + |011\rangle\Big) \\
    &= \frac{1}{2}
      \Big(\langle 000|
        \big(|\Phi_+ \rangle\langle \Phi_+|\otimes \mathbb{1}\big)
      |000\rangle +
      \langle 000|
        \big(|\Phi_+ \rangle\langle \Phi_+|\otimes \mathbb{1}\big)
      |011\rangle \\ &\;\;\;\;\;\;\;\;+
      \langle 011|
        \big(|\Phi_+ \rangle\langle \Phi_+|\otimes \mathbb{1}\big)
      |000\rangle +
      \langle 011|
        \big(|\Phi_+ \rangle\langle \Phi_+|\otimes \mathbb{1}\big)
      |011\rangle
    \Big) \\
    &= \frac{1}{2}\Big(
      \langle 00|\Phi_+ \rangle\langle \Phi_+|00\rangle
      \otimes 
      \langle 0|\mathbb{1}|0\rangle +
      \langle 00|\Phi_+ \rangle\langle \Phi_+|01\rangle
      \otimes 
      \langle 0|\mathbb{1}|1\rangle \\ &\;\;\;\;\;\;\;\;+
      \langle 01|\Phi_+ \rangle\langle \Phi_+|00\rangle
      \otimes 
      \langle 1|\mathbb{1}|0\rangle +
      \langle 01|\Phi_+ \rangle\langle \Phi_+|01\rangle
      \otimes 
      \langle 1|\mathbb{1}|1\rangle
      \Big) \\ 
   &= \frac{1}{2}\Big(
      \frac{1}{2} \otimes 1 + 0 + 0 + 0
      \Big) \\ 
   &= \frac{1}{4} 
\end{align*}

puesto que 

\begin{align*}
  \langle \Phi_+|00\rangle 
    &= \langle 00|\Phi_+\rangle 
    = \frac{1}{\sqrt{2}}\Big(
    \langle 00|00\rangle + \langle 00|11\rangle\Big)
    = \frac{1}{\sqrt{2}}(1 + 0)
    = \frac{1}{\sqrt{2}} \\
  \langle \Phi_+|01\rangle
    &=\langle 01|\Phi_+\rangle 
    = \frac{1}{\sqrt{2}}\Big(
      \langle 01|00\rangle + \langle 01|11\rangle\Big)
    = \frac{1}{\sqrt{2}}(0 + 0)
    = 0
\end{align*}

El estado final en caso de que salga $\Phi_+$ es 

\begin{align*}
  |\psi_f\rangle &= \frac
    {\Big(P_{\Phi_+} \otimes \mathbb{1}\Big)
    \Big(|0\rangle\otimes|\Phi_+\rangle\Big)}
    {\sqrt{p(\Phi_+)}} \\ 
  &= \sqrt{4}\frac{1}{\sqrt{2}}
    \Big(|\Phi_+\rangle\langle \Phi_+| \otimes \mathbb{1}\Big)
    \Big(|000\rangle + |011\rangle\Big) \\
  &= \sqrt{2}\Big(
  \big(|\Phi_+\rangle\langle \Phi_+| \otimes \mathbb{1}\big)|000
    \rangle +
  \big(|\Phi_+\rangle\langle \Phi_+| \otimes \mathbb{1}\big)|011
    \rangle
  \Big) \\ 
  &= \sqrt{2}\Big(
  |\Phi_+\rangle\langle \Phi_+|00\rangle \otimes 
    \mathbb{1}|0\rangle +
  |\Phi_+\rangle\langle \Phi_+|01\rangle \otimes 
    \mathbb{1}|1\rangle
  \Big) \\ 
  &= \sqrt{2}\Big(
    \frac{1}{\sqrt{2}}|\Phi_+\rangle \otimes |0 \rangle + 0
  \Big) \\ 
  &= |\Phi_+\rangle \otimes |0 \rangle
\end{align*}

La probabilidad de que salga $\Phi_-$ es

\begin{align*}
  p(\Phi_-) 
    &= \Big(\langle 0| \otimes \langle \Phi_+|\Big)
       \Big(|\Phi_-\rangle\langle\Phi_-|\otimes \mathbb{1}\Big)  
       \Big(|0\rangle \otimes |\Phi_+\rangle\Big) \\
    &= \frac{1}{2}
    \Big(\langle 000| + \langle 011|\Big)
    \Big(|\Phi_-\rangle\langle \Phi_-| \otimes \mathbb{1}\Big)
    \Big(|000\rangle + |011\rangle\Big) \\
    &= \frac{1}{2}
      \Big(\langle 000|
        \big(|\Phi_- \rangle\langle \Phi_-|\otimes \mathbb{1}\big)
      |000\rangle +
      \langle 000|
        \big(|\Phi_- \rangle\langle \Phi_-|\otimes \mathbb{1}\big)
      |011\rangle \\ &\;\;\;\;\;\;\;\;+
      \langle 011|
        \big(|\Phi_- \rangle\langle \Phi_-|\otimes \mathbb{1}\big)
      |000\rangle +
      \langle 011|
        \big(|\Phi_- \rangle\langle \Phi_-|\otimes \mathbb{1}\big)
      |011\rangle
    \Big) \\
    &= \frac{1}{2}\Big(
      \langle 00|\Phi_- \rangle\langle \Phi_-|00\rangle
      \otimes 
      \langle 0|\mathbb{1}|0\rangle +
      \langle 00|\Phi_- \rangle\langle \Phi_-|01\rangle
      \otimes 
      \langle 0|\mathbb{1}|1\rangle \\ &\;\;\;\;\;\;\;\;+
      \langle 01|\Phi_- \rangle\langle \Phi_-|00\rangle
      \otimes 
      \langle 1|\mathbb{1}|0\rangle +
      \langle 01|\Phi_- \rangle\langle \Phi_-|01\rangle
      \otimes 
      \langle 1|\mathbb{1}|1\rangle
      \Big) \\ 
   &= \frac{1}{2}\Big(
      \frac{1}{2} \otimes 1 + 0 + 0 + 0
      \Big) \\ 
   &= \frac{1}{4} 
\end{align*}

puesto que 

\begin{align*}
  \langle \Phi_-|00\rangle 
    &= \langle 00|\Phi_-\rangle 
    = \frac{1}{\sqrt{2}}\Big(
    \langle 00|00\rangle - \langle 00|11\rangle\Big)
    = \frac{1}{\sqrt{2}}(1 - 0)
    = \frac{1}{\sqrt{2}} \\
  \langle \Phi_-|01\rangle
    &=\langle 01|\Phi_-\rangle 
    = \frac{1}{\sqrt{2}}\Big(
      \langle 01|00\rangle - \langle 01|11\rangle\Big)
    = \frac{1}{\sqrt{2}}(0 - 0)
    = 0
\end{align*}

El estado final en caso de que salga $\Phi_-$ es 

\begin{align*}
  |\psi_f\rangle &= \frac
    {\Big(P_{\Phi_-} \otimes \mathbb{1}\Big)
    \Big(|0\rangle\otimes|\Phi_+\rangle\Big)}
    {\sqrt{p(\Phi_-)}} \\ 
  &= \sqrt{4}\frac{1}{\sqrt{2}}
    \Big(|\Phi_-\rangle\langle \Phi_-| \otimes \mathbb{1}\Big)
    \Big(|000\rangle + |011\rangle\Big) \\
  &= \sqrt{2}\Big(
  \big(|\Phi_-\rangle\langle \Phi_-| \otimes \mathbb{1}\big)|000
    \rangle +
  \big(|\Phi_-\rangle\langle \Phi_-| \otimes \mathbb{1}\big)|011
    \rangle
  \Big) \\ 
  &= \sqrt{2}\Big(
  |\Phi_-\rangle\langle \Phi_-|00\rangle \otimes 
    \mathbb{1}|0\rangle +
  |\Phi_-\rangle\langle \Phi_-|01\rangle \otimes 
    \mathbb{1}|1\rangle
  \Big) \\ 
  &= \sqrt{2}\Big(
    \frac{1}{\sqrt{2}}|\Phi_-\rangle \otimes |0 \rangle + 0
  \Big) \\ 
  &= |\Phi_-\rangle \otimes |0 \rangle
\end{align*}

La probabilidad de que salga $\Psi_+$ es

\begin{align*}
  p(\Psi_+) 
    &= \Big(\langle 0| \otimes \langle \Phi_+|\Big)
       \Big(|\Psi_+\rangle\langle\Psi_+|\otimes \mathbb{1}\Big)  
       \Big(|0\rangle \otimes |\Phi_+\rangle\Big) \\
    &= \frac{1}{2}
    \Big(\langle 000| + \langle 011|\Big)
    \Big(|\Psi_+\rangle\langle \Psi_+| \otimes \mathbb{1}\Big)
    \Big(|000\rangle + |011\rangle\Big) \\
    &= \frac{1}{2}
      \Big(\langle 000|
        \big(|\Psi_+ \rangle\langle \Psi_+|\otimes \mathbb{1}\big)
      |000\rangle +
      \langle 000|
        \big(|\Psi_+ \rangle\langle \Psi_+|\otimes \mathbb{1}\big)
      |011\rangle \\ &\;\;\;\;\;\;\;\;+
      \langle 011|
        \big(|\Psi_+ \rangle\langle \Psi_+|\otimes \mathbb{1}\big)
      |000\rangle +
      \langle 011|
        \big(|\Psi_+ \rangle\langle \Psi_+|\otimes \mathbb{1}\big)
      |011\rangle
    \Big) \\
    &= \frac{1}{2}\Big(
      \langle 00|\Psi_+ \rangle\langle \Psi_+|00\rangle
      \otimes 
      \langle 0|\mathbb{1}|0\rangle +
      \langle 00|\Psi_+ \rangle\langle \Psi_+|01\rangle
      \otimes 
      \langle 0|\mathbb{1}|1\rangle \\ &\;\;\;\;\;\;\;\;+
      \langle 01|\Psi_+ \rangle\langle \Psi_+|00\rangle
      \otimes 
      \langle 1|\mathbb{1}|0\rangle +
      \langle 01|\Psi_+ \rangle\langle \Psi_+|01\rangle
      \otimes 
      \langle 1|\mathbb{1}|1\rangle
      \Big) \\ 
   &= \frac{1}{2}\Big(
      0 + 0 + 0 + \frac{1}{2} \otimes 1
      \Big) \\ 
   &= \frac{1}{4} 
\end{align*}

puesto que 

\begin{align*}
  \langle \Psi_+|00\rangle 
    &= \langle 00|\Psi_+\rangle 
    = \frac{1}{\sqrt{2}}\Big(
    \langle 00|01\rangle + \langle 00|10\rangle\Big)
    = \frac{1}{\sqrt{2}}(0 + 0)
    = 0 \\
  \langle \Psi_+|01\rangle
    &=\langle 01|\Psi_+\rangle 
    = \frac{1}{\sqrt{2}}\Big(
      \langle 01|01\rangle + \langle 01|10\rangle\Big)
    = \frac{1}{\sqrt{2}}(1 + 0)
    = \frac{1}{\sqrt{2}}
\end{align*}

El estado final en caso de que salga $\Psi_+$ es 

\begin{align*}
  |\psi_f\rangle &= \frac
    {\Big(P_{\Psi_+} \otimes \mathbb{1}\Big)
    \Big(|0\rangle\otimes|\Phi_+\rangle\Big)}
    {\sqrt{p(\Psi_+)}} \\ 
  &= \sqrt{4}\frac{1}{\sqrt{2}}
    \Big(|\Psi_+\rangle\langle \Psi_+| \otimes \mathbb{1}\Big)
    \Big(|000\rangle + |011\rangle\Big) \\
  &= \sqrt{2}\Big(
  \big(|\Psi_+\rangle\langle \Psi_+| \otimes \mathbb{1}\big)|000
    \rangle +
  \big(|\Psi_+\rangle\langle \Psi_+| \otimes \mathbb{1}\big)|011
    \rangle
  \Big) \\ 
  &= \sqrt{2}\Big(
  |\Psi_+\rangle\langle \Psi_+|00\rangle \otimes 
    \mathbb{1}|0\rangle +
  |\Psi_+\rangle\langle \Psi_+|01\rangle \otimes 
    \mathbb{1}|1\rangle
  \Big) \\ 
  &= \sqrt{2}\Big(
    0 + \frac{1}{\sqrt{2}}|\Psi_+\rangle \otimes |1 \rangle
  \Big) \\ 
  &= |\Psi_+\rangle \otimes |1 \rangle
\end{align*}

Por último, la probabilidad de que salga $\Psi_-$ es

\begin{align*}
  p(\Psi_-) 
    &= \Big(\langle 0| \otimes \langle \Phi_+|\Big)
       \Big(|\Psi_-\rangle\langle\Psi_-|\otimes \mathbb{1}\Big)  
       \Big(|0\rangle \otimes |\Phi_+\rangle\Big) \\
    &= \frac{1}{2}
    \Big(\langle 000| + \langle 011|\Big)
    \Big(|\Psi_-\rangle\langle \Psi_-| \otimes \mathbb{1}\Big)
    \Big(|000\rangle + |011\rangle\Big) \\
    &= \frac{1}{2}
      \Big(\langle 000|
        \big(|\Psi_- \rangle\langle \Psi_-|\otimes \mathbb{1}\big)
      |000\rangle +
      \langle 000|
        \big(|\Psi_- \rangle\langle \Psi_-|\otimes \mathbb{1}\big)
      |011\rangle \\ &\;\;\;\;\;\;\;\;+
      \langle 011|
        \big(|\Psi_- \rangle\langle \Psi_-|\otimes \mathbb{1}\big)
      |000\rangle +
      \langle 011|
        \big(|\Psi_- \rangle\langle \Psi_-|\otimes \mathbb{1}\big)
      |011\rangle
    \Big) \\
    &= \frac{1}{2}\Big(
      \langle 00|\Psi_- \rangle\langle \Psi_-|00\rangle
      \otimes 
      \langle 0|\mathbb{1}|0\rangle +
      \langle 00|\Psi_- \rangle\langle \Psi_-|01\rangle
      \otimes 
      \langle 0|\mathbb{1}|1\rangle \\ &\;\;\;\;\;\;\;\;+
      \langle 01|\Psi_- \rangle\langle \Psi_-|00\rangle
      \otimes 
      \langle 1|\mathbb{1}|0\rangle +
      \langle 01|\Psi_- \rangle\langle \Psi_-|01\rangle
      \otimes 
      \langle 1|\mathbb{1}|1\rangle
      \Big) \\ 
   &= \frac{1}{2}\Big(
      0 + 0 + 0 + \frac{1}{2} \otimes 1
      \Big) \\ 
   &= \frac{1}{4} 
\end{align*}

puesto que 

\begin{align*}
  \langle \Psi_-|00\rangle 
    &= \langle 00|\Psi_-\rangle 
    = \frac{1}{\sqrt{2}}\Big(
    \langle 00|01\rangle - \langle 00|10\rangle\Big)
    = \frac{1}{\sqrt{2}}(0 - 0)
    = 0 \\
  \langle \Psi_-|01\rangle
    &=\langle 01|\Psi_-\rangle 
    = \frac{1}{\sqrt{2}}\Big(
      \langle 01|01\rangle - \langle 01|10\rangle\Big)
    = \frac{1}{\sqrt{2}}(1 - 0)
    = \frac{1}{\sqrt{2}}
\end{align*}

El estado final en caso de que salga $\Psi_-$ es 

\begin{align*}
  |\psi_f\rangle &= \frac
    {\Big(P_{\Psi_-} \otimes \mathbb{1}\Big)
    \Big(|0\rangle\otimes|\Phi_+\rangle\Big)}
    {\sqrt{p(\Psi_-)}} \\ 
  &= \sqrt{4}\frac{1}{\sqrt{2}}
    \Big(|\Psi_-\rangle\langle \Psi_-| \otimes \mathbb{1}\Big)
    \Big(|000\rangle + |011\rangle\Big) \\
  &= \sqrt{2}\Big(
  \big(|\Psi_-\rangle\langle \Psi_-| \otimes \mathbb{1}\big)|000
    \rangle +
  \big(|\Psi_-\rangle\langle \Psi_-| \otimes \mathbb{1}\big)|011
    \rangle
  \Big) \\ 
  &= \sqrt{2}\Big(
  |\Psi_-\rangle\langle \Psi_-|00\rangle \otimes 
    \mathbb{1}|0\rangle +
  |\Psi_-\rangle\langle \Psi_-|01\rangle \otimes 
    \mathbb{1}|1\rangle
  \Big) \\ 
  &= \sqrt{2}\Big(
    0 + \frac{1}{\sqrt{2}}|\Psi_-\rangle \otimes |1 \rangle
  \Big) \\ 
  &= |\Psi_-\rangle \otimes |1 \rangle
\end{align*}

\end{document}
