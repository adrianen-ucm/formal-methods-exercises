\documentclass{article}
\usepackage[english]{babel}
\usepackage[shortlabels]{enumitem}
\usepackage{amsmath}
\usepackage{amsfonts}
\usepackage{amssymb}
\usepackage{mathtools}
\usepackage{stmaryrd}
\usepackage{fancyvrb}

\title{Assignments 2 and 3 on Program Semantics}
\author{Adrián Enríquez Ballester}

\begin{document}
\maketitle

\begin{enumerate}[2.]
  \item Assume we extend the syntax of \textit{While} statements
    with a new construct: \verb|repeat| $S$ \verb|until| $b$. 
    This statement is executed as follows:

    \begin{enumerate}[(1)]
      \item Execute $S$.
      \item Check whether $b$ is false. In this case, step back
        to (1). Otherwise, finish.
    \end{enumerate}

    Define the big-step and small-step semantic rules for this 
    new construct. You cannot rely on the rules of \verb|while|
    to define the rules of \verb|repeat|. Finally, prove that
    \verb|repeat| $S$ \verb|until| $b$ is equivalent to 
    ($S$; \verb|while| $\neg b$ \verb|do| $S$).

    Its big-step semantics can be stated by the rules

    $$
    \frac
      {
        \langle S, \sigma \rangle \Downarrow \sigma' \;\;\;\;
        \mathcal{B}\llbracket b \rrbracket \sigma' = true
      }
      {
        \langle 
          \text{\Verb|repeat| } S \text{ \Verb|until| } b,
          \sigma
        \rangle \Downarrow \sigma'
      }
    \left[\text{RepeatT}_\text{BS}\right]
    $$

    $$
    \frac
      {
        \langle S, \sigma \rangle \Downarrow \sigma' \;\;\;\;
        \mathcal{B}\llbracket b \rrbracket \sigma' = false \;\;\;\;
        \langle 
          \text{\Verb|repeat| } S \text{ \Verb|until| } b,
          \sigma'
        \rangle \Downarrow \sigma''
      }
      {
        \langle 
          \text{\Verb|repeat| } S \text{ \Verb|until| } b,
          \sigma
        \rangle \Downarrow \sigma''
      }
    \left[\text{RepeatF}_\text{BS}\right]
    $$

    and its small-step semantics can be defined as

    $$
    \frac
      {
      }
      {
      \splitfrac{
        \langle 
          \text{\Verb|repeat| } S \text{ \Verb|until| } b,
          \sigma
        \rangle \longrightarrow
      }
      {
        \langle 
          S;
          \text{\Verb|if |} b \text{ \Verb|then skip else| } 
          \text{\Verb|repeat| } S \text{ \Verb|until| } b,
          \sigma
        \rangle
      }
      }
    \left[\text{Repeat}_\text{SS}\right]
    $$

    We know by a theorem that both big-step and small-step
    semantics are equivalent for the current constructs of 
    \textit{While}, so let us check that it is also true for our 
    new construct definition.

    On the one hand, suppose that 
    $
      \langle 
        \text{\Verb|repeat| } S \text{ \Verb|until| } b,
        \sigma
      \rangle \Downarrow \sigma'
    $. According to the big-step rules, the simplest possibility 
    is that the premises 
    $\langle S, \sigma \rangle \Downarrow \sigma'$ and
    $\mathcal{B}\llbracket b \rrbracket \sigma' = true$ are 
    satisfied. 
    By applying the rule induction hypothesis on the proof subtree, 
    we have 
    $\langle S, \sigma \rangle \longrightarrow^{*} \sigma'$, so

    \begin{align*}
      \langle
        \text{\Verb|repeat| } S &\text{ \Verb|until| } b,
        \sigma
      \rangle \\&\longrightarrow
      \langle
        S;
        \text{\Verb|if |} b \text{ \Verb|then skip else| } 
        \text{\Verb|repeat| } S \text{ \Verb|until| } b,
        \sigma
      \rangle \\&\longrightarrow^{*}
      \langle 
        \text{\Verb|if |} b \text{ \Verb|then skip else| } 
        \text{\Verb|repeat| } S \text{ \Verb|until| } b,
        \sigma'
      \rangle \\&\longrightarrow
      \langle 
        \text{\Verb|skip|},
        \sigma'
      \rangle \\&\longrightarrow
      \sigma'
    \end{align*}

    where the sequencing derivation chain step comes from a 
    known lemma whose proof can be easily revisited for taking 
    into account this new construct.

    Another possibility is that the satisfied premises are 
    $\langle S, \sigma \rangle \Downarrow \sigma''$,
    $\mathcal{B} \llbracket b \rrbracket \sigma'' = false$ and 
    $
      \langle 
        \text{\Verb|repeat| } S \text{ \Verb|until| } b,
        \sigma''
      \rangle \Downarrow \sigma'
    $. By applying the induction hypothesis to both proof
    subtrees, this time we have

    \begin{align*}
    \langle S, \sigma \rangle &\longrightarrow^{*} \sigma'' \\
    \langle 
      \text{\Verb|repeat| } S \text{ \Verb|until| } b,
      \sigma''
    \rangle &\longrightarrow^{*} \sigma'
    \end{align*}

    so, with all this together and the same sequencing lemma, 
    we conclude with

    \begin{align*}
      \langle
        \text{\Verb|repeat| } S &\text{ \Verb|until| } b,
        \sigma
      \rangle \\&\longrightarrow
      \langle
        S;
        \text{\Verb|if |} b \text{ \Verb|then skip else| } 
        \text{\Verb|repeat| } S \text{ \Verb|until| } b,
        \sigma
      \rangle \\&\longrightarrow^{*}
      \langle 
        \text{\Verb|if |} b \text{ \Verb|then skip else| } 
        \text{\Verb|repeat| } S \text{ \Verb|until| } b,
        \sigma''
      \rangle \\&\longrightarrow
      \langle 
        \text{\Verb|repeat| } S \text{ \Verb|until| } b,
        \sigma''
      \rangle \\&\longrightarrow^{*}
      \sigma'
    \end{align*}

    On the other hand, suppose that 
    $
      \langle 
        \text{\Verb|repeat| } S \text{ \Verb|until| } b,
        \sigma
      \rangle \longrightarrow^{k}
      \sigma'
    $, so the derivation chain must have the following shape:

    \begin{align*}
      \langle
        \text{\Verb|repeat| } S &\text{ \Verb|until| } b,
        \sigma
      \rangle \\&\longrightarrow
      \langle
        S;
        \text{\Verb|if |} b \text{ \Verb|then skip else| } 
        \text{\Verb|repeat| } S \text{ \Verb|until| } b,
        \sigma
      \rangle \\&\longrightarrow^{k - 1}
      \sigma'
    \end{align*}

    By using a second known lemma for sequencing derivation 
    chains, whose proof can be easily revisited for taking into 
    account this new construct of the language, there must 
    exist a $\sigma''$ such that 

    \begin{align*}
    \langle S, \sigma \rangle &\longrightarrow^{k_1} \sigma'' \\
    \langle 
        \text{\Verb|if |} b \text{ \Verb|then skip else| } 
        \text{\Verb|repeat| } S \text{ \Verb|until| } b,
        \sigma''
    \rangle &\longrightarrow^{k_2} \sigma'
    \end{align*}

    with $k_1$ and $k_2$ smaller than $k - 1$, thus also 
    smaller that $k$. 

    By induction hypothesis on the length
    of the derivation chain for the leftmost side, we have 
    $\langle S, \sigma \rangle \Downarrow \sigma''$ and now
    there are two cases to distinguish:

    \begin{itemize}
      \item If 
        $\mathcal{B} \llbracket b \rrbracket \sigma'' = true$,
        then 
        $
          \langle 
            \text{\Verb|skip|} , \sigma'' 
          \rangle \longrightarrow\sigma'
        $, so $\sigma' = \sigma''$ and we can conclude with
        $
          \langle 
            \text{\Verb|repeat| } S \text{ \Verb|until| } b,
            \sigma
          \rangle \Downarrow \sigma'
        $.
      \item If 
        $\mathcal{B} \llbracket b \rrbracket \sigma'' = false$,
        then
        $
          \langle 
            \text{\Verb|repeat| } S \text{ \Verb|until| } b,
            \sigma''
          \rangle \longrightarrow^{k_2 - 1} \sigma'
        $ and, as $k_2 - 1$ is also smaller than $k$, the 
        induction hypothesis can be applied to the rightmost
        side to obtain 
        $
          \langle 
            \text{\Verb|repeat| } S \text{ \Verb|until| } b,
            \sigma''
          \rangle \Downarrow \sigma'
        $. This path also leads to all the premises satisfied 
        for concluding with
        $
          \langle 
            \text{\Verb|repeat| } S \text{ \Verb|until| } b,
            \sigma
          \rangle \Downarrow \sigma'
        $.
    \end{itemize}

    Once the big-step and small-step semantics have been reaffirmed 
    to be equivalent even with this added construct, any of them 
    can be chosen for proving the requested statements equivalence. 
    We are going to proceed using the big-step semantics.

    For one of the implications, suppose that 
    $
      \langle 
        \text{\Verb|repeat| } S \text{ \Verb|until| } b,
        \sigma
      \rangle \Downarrow \sigma'
    $. One of the possible rules requires that
    $\langle S, \sigma \rangle \Downarrow \sigma'$ and
    $\mathcal{B}\llbracket b \rrbracket \sigma' = true$ so,
    as $
      \mathcal{B}\llbracket \neg b \rrbracket \sigma' = 
      \neg \mathcal{B}\llbracket  b \rrbracket \sigma' = false
    $, it holds that 
    $
      \langle 
        \text{\Verb|while| } \neg b \text{ \Verb|do| } S,
        \sigma' 
      \rangle \Downarrow \sigma'
    $ and therefore
    $
      \langle 
        S;\text{\Verb|while| } \neg b \text{ \Verb|do| } S,
        \sigma 
      \rangle \Downarrow \sigma'
    $.

    The other possible rule requires that
    $\langle S, \sigma \rangle \Downarrow \sigma''$,
    $\mathcal{B} \llbracket b \rrbracket \sigma'' = false$ and 
    $
      \langle 
        \text{\Verb|repeat| } S \text{ \Verb|until| } b,
        \sigma''
      \rangle \Downarrow \sigma'
    $. By applying the induction hypothesis to the proof
    subtree of the latter premise, we have 

    $$
      \langle 
        S;\text{\Verb|while| } \neg b \text{ \Verb|do| } S,
        \sigma'' 
      \rangle \Downarrow \sigma'
    $$

    This sequencing requires itself
    $
      \langle 
        S,
        \sigma''
      \rangle \Downarrow \sigma'''
    $ and 
    $
      \langle 
        \text{\Verb|while| } \neg b \text{ \Verb|do| } S,
        \sigma'''
      \rangle \Downarrow \sigma'
    $ which, by knowing 
    $
      \mathcal{B}\llbracket \neg b \rrbracket \sigma'' = 
      \neg \mathcal{B}\llbracket  b \rrbracket \sigma'' = true
    $ imply 

    $$
      \langle 
        \text{\Verb|while| } \neg b \text{ \Verb|do| } S,
        \sigma''
      \rangle \Downarrow \sigma'
    $$

    and finally 

    $$
      \langle 
      S;\text{\Verb|while| } \neg b \text{ \Verb|do| } S,
        \sigma
      \rangle \Downarrow \sigma'
    $$

    For the converse implication, suppose that 
    $
      \langle 
        S;\text{\Verb|while| } \neg b \text{ \Verb|do| } S,
        \sigma
      \rangle \Downarrow \sigma'
    $, which requires 
    $\langle S, \sigma \rangle \Downarrow \sigma''$
    and 
    $
      \langle 
        \text{\Verb|while| } \neg b \text{ \Verb|do| } S,
        \sigma''
      \rangle \Downarrow \sigma'
    $.

    If $\mathcal{B}\llbracket \neg b \rrbracket \sigma'' = false$,
    then 
    $
      \mathcal{B}\llbracket b \rrbracket \sigma'' =
      \neg \neg \mathcal{B}\llbracket b \rrbracket \sigma'' =
      \neg \mathcal{B}\llbracket \neg b \rrbracket \sigma'' = true
    $ and $\sigma' = \sigma''$, so 
    $
      \langle 
        \text{\Verb|repeat| } S \text{ \Verb|until| } b,
        \sigma
      \rangle \Downarrow \sigma'
    $.

    Finally, if 
    $\mathcal{B}\llbracket \neg b \rrbracket \sigma'' = true$,
    then the \verb|while| rule requires 
    $\langle S, \sigma'' \rangle \Downarrow \sigma'''$ and 
    $
      \langle 
        \text{\Verb|while| } \neg b \text{ \Verb|do| } S,
        \sigma'''
      \rangle \Downarrow \sigma'
    $, so 
    $
      \langle 
        S;\text{\Verb|while| } \neg b \text{ \Verb|do| } S,
        \sigma''
      \rangle \Downarrow \sigma'
    $ and, by rule induction on this last proof subtree, we have

    $$
      \langle 
        \text{\Verb|repeat| } S \text{ \Verb|until| } b,
        \sigma''
      \rangle \Downarrow \sigma'
    $$

    which, by knowing that
    $
      \mathcal{B}\llbracket b \rrbracket \sigma'' =
      \neg \neg \mathcal{B}\llbracket b \rrbracket \sigma'' =
      \neg \mathcal{B}\llbracket \neg b \rrbracket \sigma'' = false
    $ yields to all the premises satisfied for 

    $$
      \langle 
        \text{\Verb|repeat| } S \text{ \Verb|until| } b,
        \sigma
      \rangle \Downarrow \sigma'
    $$

  \item Add the following iterative construct to \textit{While}:
    \verb|for| $x := e_1$ \verb|to| $e_2$ \verb|do| $S$. Define
    its big-step and small-step semantic rules. You cannot rely 
    on the \verb|while| or \verb|repeat| construct to do this 
    exercise.

    Assuming that we want to be able to traverse both increasing
    and decreasing sequences, and include the ending value, 
    we can define its big-step semantic rules to be 

    $$
    \frac
      {
        \mathcal{A}\llbracket e_2 - e_1 \rrbracket \sigma 
        =_\mathbb{Z} 0_\mathbb{Z} \;\;\;\;
        \langle 
          S, 
          \sigma 
            \left[
              x \mapsto \mathcal{A} 
                \llbracket e_1 \rrbracket \sigma
            \right] 
        \rangle \Downarrow \sigma' 
      }
      {
        \langle 
          \text{\Verb|for| } x:= e_1 \text{ \Verb|to| } e_2
            \text{ \Verb|do| } S,
          \sigma
        \rangle \Downarrow \sigma'
      }
    \left[\text{ForEQ}_\text{BS}\right]
    $$

    $$
    \frac
      {\splitfrac{
        \mathcal{A}\llbracket e_2 - e_1 \rrbracket \sigma 
        >_\mathbb{Z} 0_\mathbb{Z} \;\;\;\;
        \langle 
          S, 
          \sigma 
          \left[
            x \mapsto \mathcal{A} 
              \llbracket e_1 \rrbracket \sigma
          \right] 
        \rangle \Downarrow \sigma'
      }
      {
        \langle
           \text{\Verb|for| } x:= e_1 + 1 \text{ \Verb|to| } e_2
            \text{ \Verb|do| } S,
          \sigma'
        \rangle \Downarrow \sigma''
      }
      }
      {
        \langle 
          \text{\Verb|for| } x:= e_1 \text{ \Verb|to| } e_2
            \text{ \Verb|do| } S,
          \sigma
        \rangle \Downarrow \sigma''
      }
    \left[\text{ForGT}_\text{BS}\right]
    $$

    $$
    \frac
      {\splitfrac{
        \mathcal{A}\llbracket e_2 - e_1 \rrbracket \sigma 
        <_\mathbb{Z} 0_\mathbb{Z} \;\;\;\;
        \langle 
          S, 
          \sigma
          \left[
            x \mapsto \mathcal{A} 
              \llbracket e_1 \rrbracket \sigma
          \right] 
        \rangle \Downarrow \sigma'
      }
      {
        \langle
           \text{\Verb|for| } x:= e_1 - 1 \text{ \Verb|to| } e_2
            \text{ \Verb|do| } S,
          \sigma'
        \rangle \Downarrow \sigma''
      }
      }
      {
        \langle 
          \text{\Verb|for| } x:= e_1 \text{ \Verb|to| } e_2
            \text{ \Verb|do| } S,
          \sigma
        \rangle \Downarrow \sigma''
      }
    \left[\text{ForLT}_\text{BS}\right]
    $$

    If we want to admit only increasing sequences, we can omit the 
    $[\text{ForLT}_\text{BS}]$ rule. $[\text{ForEQ}_\text{BS}]$
    can also be easily modified in order to avoid the ending sequence
    value.

    Keeping the same assumptions followed for the big-step semantics,
    its small-step semantics can be defined as

    $$
    \frac
      {
        \mathcal{A}\llbracket e_2 - e_1 \rrbracket \sigma 
        =_\mathbb{Z} 0_\mathbb{Z} 
      }
      {
        \langle 
          \text{\Verb|for| } x:= e_1 \text{ \Verb|to| } e_2
            \text{ \Verb|do| } S,
          \sigma
        \rangle \longrightarrow
        \langle
          S,
          \sigma
          \left[
            x \mapsto \mathcal{A} 
              \llbracket e_1 \rrbracket \sigma
          \right] 
        \rangle
      }
    \left[\text{ForEQ}_\text{SS}\right]
    $$

    $$
    \frac
      {
        \mathcal{A}\llbracket e_2 - e_1 \rrbracket \sigma 
        >_\mathbb{Z} 0_\mathbb{Z}
      }
      {\splitfrac{
        \langle 
          \text{\Verb|for| } x:= e_1 \text{ \Verb|to| } e_2
            \text{ \Verb|do| } S,
          \sigma
        \rangle \longrightarrow
      }
      {
        \langle 
          S;\text{\Verb|for| } x:= e_1 + 1 
            \text{ \Verb|to| } e_2
            \text{ \Verb|do| } S,
          \sigma
          \left[
            x \mapsto \mathcal{A} 
              \llbracket e_1 \rrbracket \sigma
          \right] 
        \rangle 
      }
      }
    \left[\text{ForGT}_\text{SS}\right]
    $$

    $$
    \frac
      {
        \mathcal{A}\llbracket e_2 - e_1 \rrbracket \sigma 
        <_\mathbb{Z} 0_\mathbb{Z}
      }
      {\splitfrac{
        \langle 
          \text{\Verb|for| } x:= e_1 \text{ \Verb|to| } e_2
            \text{ \Verb|do| } S,
          \sigma
        \rangle \longrightarrow
      }
      {
        \langle 
          S;\text{\Verb|for| } x:= e_1 - 1 
            \text{ \Verb|to| } e_2
            \text{ \Verb|do| } S,
          \sigma
          \left[
            x \mapsto \mathcal{A} 
              \llbracket e_1 \rrbracket \sigma
          \right] 
        \rangle 
      }
      }
    \left[\text{ForLT}_\text{SS}\right]
    $$

    where it is easy to adapt the rules if we want to change
    the construct requirements in the same way as discussed 
    for the big-step semantics.
\end{enumerate}

\end{document}
