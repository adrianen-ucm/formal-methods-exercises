\documentclass{article}
\usepackage[english]{babel}
\usepackage[shortlabels]{enumitem}
\usepackage{hyperref}
\usepackage{amsfonts}
\usepackage{amsmath}
\usepackage{amssymb}
\usepackage{mathtools}
\usepackage{xcolor}

\renewcommand{\thesubsection}{Exercise \arabic{subsection}}

\newcommand*{\colorboxed}{}
\def\colorboxed#1#{
  \colorboxedAux{#1}
}

\newcommand*{\colorboxedAux}[3]{
  \begingroup
    \colorlet{cb@saved}{.}
    \color#1{#2}
    \boxed{
      \color{cb@saved}
      #3
    }
  \endgroup
}

\title{Untyped $\lambda$-calculus exercises}
\author{Adrián Enríquez Ballester}

\begin{document}
\maketitle
\section*{Foundations}

\subsection{}\label{ex:1}

\begin{enumerate}[a)]
  \item Classify the following terms according to 
    $\alpha$-equivalence:

    \begin{align*}
  \lambda& x.x\;y & \lambda& z.z\;z & \lambda& f.f\;f           \\
  \lambda& x.x\;z & \lambda& z.z\;y & \lambda& y.\lambda x.x\;y \\
  \lambda& y.y\;z & \lambda& f.f\;y & \lambda& z.\lambda y.y\;z
    \end{align*}

    \textbf{Solution:} we can perform an $\alpha$-conversion to 
    the following terms in order to reach another one of the list:

    \begin{align*}
      \lambda x.x\;y 
        &\rightsquigarrow_\alpha \lambda z.(x\;y)[z/x] 
        \equiv \lambda z.z\;y \\
      \lambda x.x\;y 
        &\rightsquigarrow_\alpha \lambda f.(x\;y)[f/x] 
        \equiv \lambda f.f\;y \\
      \lambda x.x\;z 
        &\rightsquigarrow_\alpha \lambda y.(x\;z)[y/x] 
        \equiv \lambda y.y\;z \\ 
      \lambda z.z\;z 
        &\rightsquigarrow_\alpha \lambda f.(z\;z)[f/z] 
        \equiv \lambda f.f\;f
    \end{align*}

    The last one requires of two $\alpha$-conversions:

    \begin{align*}
      \lambda y.\lambda x.x\;y 
        &\rightsquigarrow_\alpha \lambda z.(\lambda x.x\;y)[z/y] 
        \equiv \lambda z.\lambda x.x\;z \\
        &\rightsquigarrow_\alpha \lambda z.\lambda y.(x\;z)[y/x] 
        \equiv \lambda z.\lambda y.y\;z
    \end{align*}

    To sum up, we have found the following $\alpha$-equivalences,
    grouping the terms into four equivalence classes:

    \begin{gather*}
      \lambda x.x\;y 
        =_\alpha \lambda z.z\;y 
        =_\alpha \lambda f.f\;y \\ 
      \lambda x.x\;z 
        =_\alpha \lambda y.y\;z \\ 
      \lambda z.z\;z
        =_\alpha \lambda f.f\;f \\ 
      \lambda y.\lambda x.x\;y 
        =_\alpha \lambda z.\lambda y.y\;z
    \end{gather*}

  \item Provide an $\alpha$-equivalent term where each abstraction 
    uses a different variable name:

    $$
      \lambda x.
        ((x\;(\lambda y.x\;y))\;(\lambda x.x))\;
        (\lambda y.y\;x)
    $$

    \textbf{Solution:} it suffices to perform an 
    $\alpha$-conversion to two of its abstraction subterms 
    using a new variable name:

    \begin{align*}
      \lambda x.
        ((x\;(\lambda y.x\;y))\;(&\lambda x.x))\;
        (\lambda y.y\;x) \\
      \rightsquigarrow_\alpha\lambda x.
        &((x\;(\lambda y.x\;y))\;(\lambda z.x[z/x]))\;
        (\lambda y.y\;x) \\
      \equiv\lambda x.
        &((x\;(\lambda y.x\;y))\;(\lambda z.z))\;
        (\lambda y.y\;x) \\
      \rightsquigarrow_\alpha\lambda x.
        &((x\;(\lambda y.x\;y))\;(\lambda z.z))\;
        (\lambda w.(y\;x)[w/y]) \\
      \equiv\lambda x.
        &((x\;(\lambda y.x\;y))\;(\lambda z.z))\;
        (\lambda w.w\;x)
    \end{align*}

    The resulting term has a different variable name for each 
    one of its abstraction subterms variable names.

\end{enumerate}

\subsection{}\label{ex:2}

Normalize the following term: 

$$(\lambda x.(\lambda y.x\;y))\;y$$

\textbf{Solution:} although it seems to be a simple $\beta$-reduction, 
we must perform capture-avoiding substitution carefully in order to 
get the right normalized term:

\begin{align*}
  (\lambda x.(\lambda y.x\;y))\;y \\
    \rightsquigarrow_\beta &(\lambda y.x\;y)[y/x] \\ 
    \equiv &\lambda z.(x\;y)[z/y][y/x] \\
    \equiv &\lambda z.(x\;z)[y/x] \\
    \equiv &\lambda z.y\;z
\end{align*}

\subsection{}\label{ex:3}

The de Bruijn index notation is a way of avoiding the problems 
related to substitution and variable capture in Church’s original 
presentation of the $\lambda$-calculus, thus facilitating its 
mechanized treatment. The key idea is to replace variable names 
by numbers denoting the depth of the scope of that variable. For 
example, the familiar terms $\lambda x.x$, $\lambda x.\lambda y.x$ 
and $\lambda x.\lambda y.y$ are represented as $\lambda1$, 
$\lambda\lambda 2$ and $\lambda\lambda 1$ in de Bruijn’s notation. 
Free variables are represented by numbers higher than the maximum
depth in its location. For example, $\lambda\lambda 3$ is a 
possible representation for $\lambda x.\lambda y.w$.

\begin{enumerate}[a)]
  \item\label{ex:3:a} Represent the term 
    $(\lambda x.\lambda y.\lambda z.x\;z\;y)(\lambda x.\lambda y.x)$
    in de Brujin 's notation.

    \textbf{Solution:}

    $$(\lambda\lambda\lambda 3\;1\;2)\;(\lambda\lambda 2)$$
  \item Explain how $\beta$-reduction of terms in de Brujin
    notation can be implemented.
  
    \textbf{Solution:} we are going to explain it informally 
    and show an example in \ref{ex:3:c}. Given an application 
    of an abstraction $N$ to a term $M$ (i.e. $N\;M$):

    \begin{itemize}
      \item The variables in the body of $N$ which are bounded 
        by its formal parameter must be replaced by $M$. 
      \item In each particular replacement, the free variables in 
       $M$ must be incremented by the scope depth at the variable 
        subterm where each replacement is taking place.
      \item The free variables which were free in the body of $N$ 
        before the reduction must be decremented by one because 
        the depth of its scope is being decremented.
    \end{itemize}
  \item\label{ex:3:c} Apply your ideas to the application 
    in \ref{ex:3:a}.
    
    \textbf{Solution:} 

    \begin{align*}
      (\lambda\lambda\lambda 3\;1\;2)\;(\lambda\lambda 2) \\ 
        &\rightsquigarrow_\beta 
          \lambda\lambda (\lambda\lambda 2)\;1\;2 
          \label{ex:3:c:1}\tag{1} \\ 
        &\rightsquigarrow_\beta 
          \lambda\lambda (\lambda 2)\;2 
          \label{ex:3:c:2}\tag{2} \\ 
        &\rightsquigarrow_\beta 
          \lambda\lambda 1 
          \label{ex:3:c:3}\tag{3}
    \end{align*}

    In the $\beta$-reduction at (\ref{ex:3:c:1}), the bound 
    variable was $3$ and there were not free variables 
    in the abstraction body, so it was just replaced by 
    $\lambda\lambda2$, which also has not free variables. 

    In the $\beta$-reduction at (\ref{ex:3:c:2}), the bound 
    variable was the innermost $2$, which has been replaced 
    by $1$ incremented by $1$. This is due to $1$ being free 
    in itself and replaced inside an abstraction of depth $1$.

    Finally, in the $\beta$-reduction at (\ref{ex:3:c:3}), 
    there were no occurrences of the bound variable of the 
    abstraction being reduced, so the body remains the same but 
    decrementing its single free variable.
\end{enumerate}

\subsection{}\label{ex:4}

Combinators can be seen as $\lambda$-terms without free variables, 
although they were actually proposed independently from the 
$\lambda$-calculus. Given the combinators 

\begin{align*}
  \mathsf{S} &\triangleq \lambda x.\lambda y.\lambda z. 
    (x\;z)\;(y\;z) \\ 
  \mathsf{K} &\triangleq \lambda x.\lambda y.x \\ 
  \mathsf{I} &\triangleq \lambda x.x
\end{align*}

prove the equivalence $\mathsf{SKK} = \mathsf{I}$.

\textbf{Solution:} an advantage of applying an abstraction to 
a combinator is that, as a combinator has not free variables, 
there is no chance for variable capture to happen.

\begin{align*}
  \mathsf{SKK} & \\
    \equiv &(\lambda x.\lambda y.\lambda z. (x\;z)\;(y\;z))\;
      (\lambda x.\lambda y.x)\;
      (\lambda x.\lambda y.x) \\ 
    \rightsquigarrow_\beta 
      &(\lambda y.\lambda z. 
        ((x\;z)\;(y\;z))
      )[(\lambda x.\lambda y.x)/x]\;
      (\lambda x.\lambda y.x) \\ 
    \equiv 
      &(\lambda y.\lambda z. 
        ((\lambda x. \lambda y.x)\;z)\;(y\;z)
      )\;
      (\lambda x.\lambda y.x) \\ 
    \rightsquigarrow_\beta 
      &(\lambda z. 
        ((\lambda x. \lambda y.x)\;z)\;(y\;z)
      )[(\lambda x.\lambda y. x)/y] \\ 
    \equiv 
      &\lambda z. 
        ((\lambda x. \lambda y.x)\;z)\;
        ((\lambda x. \lambda y.x)\;z) \\ 
    \rightsquigarrow_\beta 
      &\lambda z. 
        (\lambda y.x)[z/x]\;
        ((\lambda x. \lambda y.x)\;z) \\ 
    \equiv 
      &\lambda z. 
        (\lambda y.z)\;
        ((\lambda x. \lambda y.x)\;z) \\ 
    \rightsquigarrow_\beta 
      &\lambda z. 
        z[((\lambda x. \lambda y.x)\;z)/y] \\
    \equiv &\lambda z.z \\ 
    =&_\alpha \mathsf{I}
\end{align*}

\subsection{}\label{ex:5}

Combinator systems are commonly presented as equational 
theories where a combinator base is defined using oriented 
equational rules and new combinators are created by means 
of application alone (i.e. no abstraction is required). For 
example, the aforementioned combinators would be defined by 
the equations $\mathsf{S}\;m\;n\;o = m\;o\;(n\;o)$, 
$\mathsf{K}\;a\;b = a$ and $\mathsf{I}\;x = x$. As variables 
only appear in definitions where no confusion of scopes can 
happen, combinators solve in a natural way many of the nuisances
associated with variable names in Church’s formulation of the 
$\lambda$-calculus.

Take as base the following two combinators: 

\begin{align*}
  \mathsf{B}\;f\;g\;x &= f\;(g\;x) \\ 
  \mathsf{M}\;x &= x\;x 
\end{align*}

Using $\mathsf{B}$ and $\mathsf{M}$ alone prove the existence of 
a narcissistic combinator $n$ such that $nn = n$.

\textbf{Solution:} I have not found a better way for finding such 
a term than 'brute force'. As it was none of the possible 
combinators of size one, two and three, I wrote a program that 
generates all the possible combinators of increasing size and tests 
for this property to hold. The smallest result was

$$
  n \triangleq \mathsf{M}\;
    (\mathsf{B}\;\mathsf{M}\;\mathsf{M})
$$

which can be verified to hold the property:

\begin{align*}
  n &= \mathsf{M}\;(\mathsf{B}\;\mathsf{M}\;\mathsf{M}) \\
    &= \mathsf{B}\;\mathsf{M}\;\mathsf{M}\;
      (\mathsf{B}\;\mathsf{M}\;\mathsf{M}) \\
    &= \mathsf{M}\;(\mathsf{M}\;
      (\mathsf{B}\;\mathsf{M}\;\mathsf{M})) \\
    &= \mathsf{M}\;
      (\mathsf{B}\;\mathsf{M}\;\mathsf{M})\;
      (\mathsf{M}\;(\mathsf{B}\;\mathsf{M}\;\mathsf{M})) \\
    &= n\;n
\end{align*}

Just for fun, these are some other bigger solutions 
that the program has found:

\begin{center}
  $$
    \mathsf{M}\;
      (\mathsf{B}\;(\mathsf{B}\;\mathsf{M})\;\mathsf{M})\;
      \mathsf{B}
  $$ 
  $$
    \mathsf{M}\;
      (\mathsf{B}\;(\mathsf{B}\;\mathsf{M})\;\mathsf{M})\;
      \mathsf{M}
  $$ 
  $$
    \mathsf{M}\;
      (\mathsf{B}\;(\mathsf{B}\;\mathsf{M})\;\mathsf{M})\;
      (\mathsf{B}\;\mathsf{B})
  $$ 
  $$
    \mathsf{M}\;
      (\mathsf{B}\;(\mathsf{B}\;\mathsf{M})\;\mathsf{M})\;
      (\mathsf{B}\;\mathsf{M})
  $$ 
  $$
    \mathsf{M}\;
      (\mathsf{B}\;(\mathsf{B}\;\mathsf{M})\;\mathsf{M})\;
      (\mathsf{M}\;\mathsf{B})
  $$
  $$
    \mathsf{M}\;
      (\mathsf{B}\;(\mathsf{B}\;\mathsf{M})\;\mathsf{M})\;
      (\mathsf{M}\;\mathsf{M})
  $$ 
  $$
    \mathsf{M}\;
      (\mathsf{B}\;\mathsf{B}\;\mathsf{B}\;\mathsf{M}\;\mathsf{M})\;
      \mathsf{B}
  $$
  $$
    \mathsf{M}\;
      (\mathsf{B}\;\mathsf{B}\;\mathsf{B}\;\mathsf{M}\;\mathsf{M})\;
      \mathsf{M}
  $$
  $$
    \mathsf{M}\;
      (\mathsf{M}\;\mathsf{B}\;\mathsf{B}\;\mathsf{M}\;\mathsf{M})\;
      \mathsf{B}
  $$
  $$
    \mathsf{M}\;
      (\mathsf{M}\;\mathsf{B}\;\mathsf{B}\;\mathsf{M}\;\mathsf{M})\;
      \mathsf{M}
  $$
\end{center}

By observing these results, it is possible that

$$
  \mathsf{M}\;
    (\mathsf{B}\;(\mathsf{B}\;\mathsf{M})\;\mathsf{M})\;
    x
$$  

may be narcissistic for any $\mathsf{B}\mathsf{M}$-combinator $x$, 
which is indeed the case:

\begin{align*}
    \mathsf{M}\; 
    (\mathsf{B}\;(\mathsf{B}\;\mathsf{M})\;\mathsf{M})\;
    x &= 
    \mathsf{B}\;(\mathsf{B}\;\mathsf{M})\;\mathsf{M}\;
    (\mathsf{B}\;(\mathsf{B}\;\mathsf{M})\;\mathsf{M})\; 
    x
    \\
    &=
    \mathsf{B}\;\mathsf{M}\;
    (\mathsf{M}\;
      (\mathsf{B}\;(\mathsf{B}\;\mathsf{M})\;\mathsf{M})
    )\;
    x \\ 
    &= 
    \mathsf{M}\; 
    (\mathsf{M}\;(
      \mathsf{B}\;(\mathsf{B}\;\mathsf{M})\;\mathsf{M}
    )\;x) \\ 
    &= 
    \mathsf{M}\;
    (\mathsf{B}\;(\mathsf{B}\;\mathsf{M})\;\mathsf{M})\;
    x\; 
    (\mathsf{M}\;
    (\mathsf{B}\;(\mathsf{B}\;\mathsf{M})\;\mathsf{M})\;
    x) \\ 
\end{align*}

\section*{Constructive mathematics in the $\lambda$-calculus}

All the encodings and terms which represent operations 
appearing in this section have been automatically tested 
in a more extensive way than some of the examples shown here. 
The corresponding code is due to be presented in an
upcoming exercises delivery.

Also, the reductions in this section are shown in a less explicit 
way than in the previous one, avoiding the intermediate steps of 
safely replacing variables and even performing more than one 
reduction at a time between steps (e.g. two $\beta$-reductions
are indicated as $\rightsquigarrow_\beta^2$). 
For long reductions we have used 
the syntax 
$\colorboxed{red}{abstraction}\;\colorboxed{blue}{parameter}$
to show the application term(s) being $\beta$-reduced in order 
to improve its readability.

\subsection{}\label{ex:6}

Using Church’s encoding of booleans in the pure 
$\lambda$-calculus, define normalized $\lambda$-terms
$\mathsf{CONJ}$, $\mathsf{DISJ}$ and $\mathsf{NEG}$ 
to represent conjunction, disjunction and negation, 
respectively. 

\textbf{Solution:} recall the definition of the boolean terms as

\begin{align*}
  \mathsf{TRUE} &\triangleq \lambda x.\lambda y.x \\
  \mathsf{FALSE} &\triangleq \lambda x.\lambda y.y \\
\end{align*}

First, $\mathsf{NEG}$ must be a term which when applied to a 
boolean term becomes the other one. This can be achieved 
by embedding it into an abstraction which applies it to its 
parameters in reverse order:

$$
\mathsf{NEG} 
  \triangleq \lambda b.\lambda x.\lambda y. 
    b\;y\;x
$$

Its behavior is the following:

\begin{align*}
  \mathsf{NEG}\;\mathsf{TRUE} & \\
    \equiv&
      (\lambda b.\lambda x.\lambda y.b\;y\;x)\;
      (\lambda x.\lambda y.x) \\
    \rightsquigarrow_\beta&
      \lambda x.\lambda y.(
          \lambda x.\lambda y.x
        )\;
        y\;
        x \\ 
    \rightsquigarrow_\beta^2&
      \lambda x.\lambda y.y \\ 
    \equiv& \mathsf{FALSE}
\end{align*}

\begin{align*}
  \mathsf{NEG}\;\mathsf{FALSE} & \\
    \equiv&
      (\lambda b.\lambda x.\lambda y.b\;y\;x)\;
      (\lambda x.\lambda y.y) \\
    \rightsquigarrow_\beta&
      \lambda x.\lambda y.(
        \lambda x.\lambda y.y
      )\;
      y\;
      x \\ 
    \rightsquigarrow_\beta^2&
      \lambda x.\lambda y.x \\ 
    \equiv& \mathsf{TRUE}
\end{align*}

Second, $\mathsf{CONJ}$ must be a term which when applied to two 
boolean terms becomes into $\mathsf{FALSE}$ if the first one 
also is, and into the second one otherwise. It can be defined 
as

$$
\mathsf{CONJ} 
  \triangleq \lambda b.\lambda c.\lambda x.\lambda y.
    b\;(c\;x\;y)\;y
$$

because

\begin{align*}
  \mathsf{CONJ}\;\mathsf{FALSE} & \\
    \equiv&
      (\lambda b.\lambda c.\lambda x.\lambda y.b\;(c\;x\;y)\;y)\;
      (\lambda x.\lambda y.y) \\
    \rightsquigarrow_\beta&
      \lambda c.\lambda x.\lambda y.
        (\lambda x.\lambda y.y)\;(c\;x\;y)\;y \\
    \rightsquigarrow_\beta^2&
      \lambda c.\lambda x.\lambda y.y \\
    \equiv&
      \lambda c.\mathsf{FALSE}
\end{align*}

and 

\begin{align*}
  \mathsf{CONJ}\;\mathsf{TRUE} & \\
    \equiv&
      (\lambda b.\lambda c.\lambda x.\lambda y.b\;(c\;x\;y)\;y)\;
      (\lambda x.\lambda y.x) \\
    \rightsquigarrow_\beta&
      \lambda c.\lambda x.\lambda y.
        (\lambda x.\lambda y.x)\;(c\;x\;y)\;y \\
    \rightsquigarrow_\beta^2&
      \lambda c.\lambda x.\lambda y.c\;x\;y \\
    \rightsquigarrow_\eta^2&
      \lambda c.c
\end{align*}

where we have used also $\eta$-reduction for the ease of 
reasoning about the property.

Finally, $\mathsf{DISJ}$ must be a term which when applied 
to two boolean terms becomes into $\mathsf{TRUE}$ if the 
first one also is, and into the second one otherwise.
It can be defined as 

$$
\mathsf{DISJ} 
  \triangleq \lambda b.\lambda c.\lambda x.\lambda y. 
    b\;x\;(c\;x\;y)
$$

because

\begin{align*}
  \mathsf{DISJ}\;\mathsf{TRUE} & \\
    \equiv&
      (\lambda b.\lambda c.\lambda x.\lambda y.b\;x\;(c\;x\;y))\;
      (\lambda x.\lambda y.x) \\
    \rightsquigarrow_\beta&
      \lambda c.\lambda x.\lambda y.
        (\lambda x.\lambda y.x)\;x\;(c\;x\;y) \\
    \rightsquigarrow_\beta^2&
      \lambda c.\lambda x.\lambda y.x \\
    \equiv&
      \lambda c.\mathsf{TRUE}
\end{align*}

and 

\begin{align*}
  \mathsf{DISJ}\;\mathsf{FALSE} & \\
    \equiv&
      (\lambda b.\lambda c.\lambda x.\lambda y.b\;x\;(c\;x\;y))\;
      (\lambda x.\lambda y.y) \\
    \rightsquigarrow_\beta&
      \lambda c.\lambda x.\lambda y.
        (\lambda x.\lambda y.y)\;x\;(c\;x\;y) \\
    \rightsquigarrow_\beta^2&
      \lambda c.\lambda x.\lambda y.c\;x\;y \\
    \rightsquigarrow_\eta^2&
      \lambda c.c
\end{align*}

where, again, we have used also $\eta$-reduction for the 
ease of reasoning about the property.

\subsection{}\label{ex:7}

Using Church’s encoding of natural numbers define addition, 
multiplication and exponentiation.

\textbf{Solution:} recall that natural numbers are encoded as

$$
  \lambda s.\lambda z.
    s\;(s\;(...\;s(z)\;...))
$$

where the number of times that $s$ is applied determines the 
natural number it represents. For example, we have the following 
encodings:

\begin{align*}
  \mathsf{0} &\triangleq \lambda s.\lambda z.z \\
  \mathsf{1} &\triangleq \lambda s.\lambda z.s\;z \\
  \mathsf{2} &\triangleq \lambda s.\lambda z.s\;(s\;z) \\
\end{align*}

First, $\mathsf{ADD}$ must be a natural number term that given 
two natural number terms $m$ and $n$, which respectively represent 
two natural numbers $a$ and $b$, must represent the natural number 
$a + b$. It can be defined as

$$
\mathsf{ADD} 
  \triangleq \lambda m.\lambda n.\lambda s.\lambda z.
    n\;s\;(m\;s\;z)
$$

because, under the initial explanation conditions, it applies 
$b$ times $s$ to the result of applying $a$ times $s$ starting
with $z$ and therefore $s$ is applied $a + b$ times.

This is an example of its behavior:

\begin{align*}
  \mathsf{ADD}\;\mathsf{3}\;\mathsf{2} &\\
  &\equiv
    \colorboxed{red}{
      (\lambda m.\lambda n.\lambda s.\lambda z.n\;s\;(m\;s\;z))
    }\;
    \colorboxed{blue}{
      (\lambda s.\lambda z.s\;(s\;(s\;z)))
    }\;
    \colorboxed{blue}{
      (\lambda s.\lambda z.s\;(s\;z))
    } \\
  &\rightsquigarrow_\beta^2
    \lambda s.\lambda z.
      \colorboxed{red}{
        (\lambda s.\lambda z.s\;(s\;z))
      }\;
      \colorboxed{blue}{s}\;
      (
        \colorboxed{red}{(\lambda s.\lambda z.s\;(s\;(s\;z)))}\;
        \colorboxed{blue}{s}\;\colorboxed{blue}{z}
      ) \\
  &\rightsquigarrow_\beta^3
    \lambda s.\lambda z.
      \colorboxed{red}{(\lambda z.s\;(s\;z))}\;
      \colorboxed{blue}{(s\;(s\;(s\;z)))} \\
  &\rightsquigarrow_\beta
    \lambda s.\lambda z.
      s\;(s\;(s\;(s\;(s\;z)))) \\
  &\equiv \mathsf{5}
\end{align*}

Second, $\mathsf{MULT}$ must be a natural number term that given 
two natural number terms $m$ and $n$, which respectively represent 
two natural numbers $a$ and $b$, must represent the natural number 
$a \cdot b$. It can be defined as

$$
\mathsf{MULT} 
  \triangleq \lambda m.\lambda n.\lambda s.\lambda z. 
    n\;(m\;s)\;z
$$

because, under the initial explanation conditions, it applies 
$b$ times a term which applies $a$ times $s$ starting with $z$
and therefore $s$ is applied $a$ times $b$ times.

An example of its behavior is as follows:

\begin{align*}
  \mathsf{MULT}\;\mathsf{3}\;\mathsf{2} &\\
  &\equiv
    \colorboxed{red}{
      (\lambda m.\lambda n.\lambda s.\lambda z.n\;(m\;s)\;z)
    }\;
    \colorboxed{blue}{
      (\lambda s.\lambda z.s\;(s\;(s\;z)))
    }\;
    \colorboxed{blue}{
      (\lambda s.\lambda z.s\;(s\;z))
    } \\
  &\rightsquigarrow_\beta^2
    \lambda s.\lambda z.
      (\lambda s.\lambda z.s\;(s\;z))\;
      (\colorboxed{red}{
        (\lambda s.\lambda z.s\;(s\;(s\;z)))
      }\;\colorboxed{blue}{s})\;z \\
  &\rightsquigarrow_\beta
    \lambda s.\lambda z.
      \colorboxed{red}{(\lambda s.\lambda z.s\;(s\;z))}\;
      \colorboxed{blue}{(\lambda z.s\;(s\;(s\;z)))}
      \;z \\
  &\rightsquigarrow_\beta
    \lambda s.\lambda z.
      \colorboxed{red}{
        (\lambda z.
          (\lambda z.\;s\;(s\;(s\;z)))\;
          ((\lambda z.s\;(s\;(s\;z)))\;z)
        )
      }\;
      \colorboxed{blue}{z} \\
  &\rightsquigarrow_\beta
    \lambda s.\lambda z.
      (\lambda z.\;s\;(s\;(s\;z)))\;
      (\colorboxed{red}{
        (\lambda z.s\;(s\;(s\;z)))
      }\;\colorboxed{blue}{z}) \\
  &\rightsquigarrow_\beta
    \lambda s.\lambda z.
      \colorboxed{red}{(\lambda z.\;s\;(s\;(s\;z)))}\;
      \colorboxed{blue}{(s\;(s\;(s\;z)))} \\
  &\rightsquigarrow_\beta
    \lambda s.\lambda z.
      s\;(s\;(s\;(s\;(s\;(s\;z)))))\; \\
  &\equiv \mathsf{6}
\end{align*}

Finally, $\mathsf{EXPO}$ must be a natural number term that given 
two natural number terms $m$ and $n$, which respectively represent 
two natural numbers $a$ and $b$, must represent the natural number 
$a^b$. It can be defined as

$$
\mathsf{EXPO} 
  \triangleq \lambda m.\lambda n.\lambda s.\lambda z. 
    n\;m\;s\;z
$$

because, under the initial explanation conditions, it applies 
$b$ times $m$ starting with $s$ and, as $m$ applies $a$ times 
its parameter, the result is $s^{a^b}$ (i.e. $s$ applied $a^b$ 
times).

We also show an example of its behavior:

\begin{align*}
  \mathsf{EXPO}\;\mathsf{3}\;\mathsf{2} &\\
  &\equiv
    \colorboxed{red}{
      (\lambda m.\lambda n.\lambda s.\lambda z.n\;m\;s\;z)
    }\;
    \colorboxed{blue}{
      (\lambda s.\lambda z.s\;(s\;(s\;z)))
    }\;
    \colorboxed{blue}{
      (\lambda s.\lambda z.s\;(s\;z))
    } \\
  &\rightsquigarrow_\beta^2
    \lambda s.\lambda z.
      \colorboxed{red}{
        (\lambda s.\lambda z.s\;(s\;z))
      }\;
      \colorboxed{blue}{
        (\lambda s.\lambda z.s\;(s\;(s\;z)))
      }\;
      s\;
      z \\
  &\rightsquigarrow_\beta
    \lambda s.\lambda z.
      \colorboxed{red}{
        (\lambda z.
          (\lambda s.\lambda z.s\;(s\;(s\;z)))\;
          ((\lambda s.\lambda z.s\;(s\;(s\;z)))\;z)
      )}\;
      \colorboxed{blue}{s}\;
      z \\
  &\rightsquigarrow_\beta
    \lambda s.\lambda z.
      (
        (\lambda s.\lambda z.s\;(s\;(s\;z)))\;
        (
          \colorboxed{red}{
            (\lambda s.\lambda z.s\;(s\;(s\;z)))
          }\;
          \colorboxed{blue}{s}
        )
      )\;
      z \\
  &\rightsquigarrow_\beta
    \lambda s.\lambda z.
      (
        \colorboxed{red}{
          (\lambda s.\lambda z.s\;(s\;(s\;z)))
        }\;
        \colorboxed{blue}{
          (\lambda z.s\;(s\;(s\;z)))
        }
      )\;
      z \\
  &\rightsquigarrow_\beta
    \lambda s.\lambda z.
      \colorboxed{red}{(\lambda z.
        (\lambda z.s\;(s\;(s\;z)))\;
        ((\lambda z.s\;(s\;(s\;z)))\;
        ((\lambda z.s\;(s\;(s\;z)))\;z))
      )}\;
      \colorboxed{blue}{z} \\
  &\rightsquigarrow_\beta
    \lambda s.\lambda z.
      (\lambda z.s\;(s\;(s\;z)))\;
      ((\lambda z.s\;(s\;(s\;z)))\;
      (
        \colorboxed{red}{(\lambda z.s\;(s\;(s\;z)))}\;
        \colorboxed{blue}{z}
      )) \\
  &\rightsquigarrow_\beta
    \lambda s.\lambda z.
      (\lambda z.s\;(s\;(s\;z)))\;
      (
        \colorboxed{red}{
          (\lambda z.s\;(s\;(s\;z)))
        }\;
        \colorboxed{blue}{
          (s\;(s\;(s\;z)))
        }
      ) \\
  &\rightsquigarrow_\beta
    \lambda s.\lambda z.
      \colorboxed{red}{
        (\lambda z.s\;(s\;(s\;z)))
      } \;
      \colorboxed{blue}{
        (s\;(s\;(s\;(s\;(s\;(s\;z))))))
      } \\
  &\rightsquigarrow_\beta
    \lambda s.\lambda z.
      s\;(s\;(s\;(s\;(s\;(s\;(s\;(s\;(s\;z)))))))) \\
  &\equiv \mathsf{9}
\end{align*}

Although $\mathsf{MULT}$ and $\mathsf{EXPO}$ as defined here 
are $\beta$-normal terms, they are not $\eta$-normal terms 
and can be expressed in a more concise way:

\begin{align*}
  \mathsf{MULT} &\rightsquigarrow_\eta 
    \lambda m.\lambda n.\lambda s. 
      m\;(n\;s) \\
  \mathsf{EXPO} &\rightsquigarrow_\eta^2
    \lambda m.\lambda n.
      n\;m \\
\end{align*}

\subsection{}\label{ex:8}

Devise an encoding for pairs in the $\lambda$-calculus. 
Provide $\lambda$-terms $\mathsf{PAIR}$ (a pair constructor)
and the projections $\mathsf{FST}$ and $\mathsf{SND}$. What 
properties would be required in order to check the correctness
of the encoding?

\textbf{Solution:} a pair can be represented as

$$
\lambda f.f\;a\;b
$$

which applies a given term $f$ to the pair contents $a$ and $b$.

Such a term can be constructed waiting for its contents to be 
provided as

$$
\mathsf{PAIR} 
  \triangleq \lambda a.\lambda b.\lambda f. 
    f\;a\;b
$$

and its respective projections, which allow to retrieve them as

$$
\mathsf{FST} 
  \triangleq \lambda p. 
    p (\lambda x.\lambda y.x)
$$

$$
\mathsf{SND} 
  \triangleq \lambda p. 
    p (\lambda x.\lambda y.y)
$$

This encoding must satisfy the properties

\begin{align*}
  \mathsf{FST}\;(\mathsf{PAIR}\;a\;b)
    &\rightsquigarrow^*_\beta a \\ 
  \mathsf{SND}\;(\mathsf{PAIR}\;a\;b)
    &\rightsquigarrow^*_\beta b \\
\end{align*}

which state that the projections over a pair must effectively 
allow to retrieve the corresponding terms used when 
constructing it.

We show that this is the case for the $\mathsf{FST}$ term, and 
it can also be checked in a similar way for $\mathsf{SND}$:

\begin{align*}
  \mathsf{FST}\;(\mathsf{PAIR}\;a\;b) & \\
  \equiv&
    (\lambda p.p\;(\lambda x.\lambda y.x))\;
    ((\lambda a.\lambda b.\lambda f.f\;a\;b)\;a\;b) \\
  \rightsquigarrow_\beta^2&
    (\lambda p.p\;(\lambda x.\lambda y.x))\;
    (\lambda f.f\;a\;b) \\
  \rightsquigarrow_\beta&
    (\lambda f.f\;a\;b)\;
    (\lambda x.\lambda y.x) \\
  \rightsquigarrow_\beta&
    (\lambda x.\lambda y.x)\;a\;b \\
  \rightsquigarrow_\beta^2&
    a
\end{align*}

\subsection{}\label{ex:9}

Define a predecessor function and subtraction for Church numerals.

\textbf{Solution:} we can define a term that given a natural 
number term $n$ wraps it and becomes into a natural number term 
that makes $n$ to avoid one of its $s$ applications. This means 
that, at least for our definition, the predecessor of 0 is going 
to be 0:

\begin{align*}
\lambda n.\lambda s.\lambda z.\mathsf{FST}\;( 
    \label{ex:9:idea:3}\tag{3} \\
    &\;\;\;\;n \\
    &\;\;\;\;(
      \lambda p.
        \mathsf{PAIR}\;(\mathsf{SND}\;p\;(\mathsf{FST}\;p))\;s
    ) \label{ex:9:idea:2}\tag{2} \\
    &\;\;\;\;(\mathsf{PAIR}\;z\;(\lambda x.x))
      \label{ex:9:idea:1}\tag{1} \\
  &)
\end{align*}

At (\ref{ex:9:idea:1}), we wrap the initial element for $n$ within 
a pair, where the first element is $z$ and the second 
element is the next function to apply to it. The identity 
abstraction is given at first.

At (\ref{ex:9:idea:2}), the function which $n$ applies expects to 
take a pair and yield a new pair, where the second element is $s$ 
and the first element is its second applied to its first. As 
initially we have provided the identity abstraction, the original 
$s$ is going to be applied the times $n$ does except for the 
first one.

To conclude with the explanation, the application of $n$ to a 
function between pairs and a pair would also yield a pair, 
so we apply $\mathsf{FST}$ at (\ref{ex:9:idea:3}) in order to end 
up with the desired result, which has been laying into the first
element of the pair.

In order to achieve a concise and normalized definition, we 
expand the outermost $\mathsf{FST}$ definition and the 
$\mathsf{PAIR}$ ones, but instead of expanding 
$\mathsf{SND}\;p\;(\mathsf{FST}\;p)$, we can write directly a 
function which given a pair applies its second element to its 
first:

\begin{align*}
  \mathsf{PRED} 
    \triangleq
  &\lambda n.\lambda s.\lambda z.( \\
    &\;\;\;\;\;\;n \\
    &\;\;\;\;\;\;(
      \lambda p.\lambda f.f\;(p\;(\lambda x.\lambda y. y\;x))\;s
    ) \\
    &\;\;\;\;\;\;(\lambda f. f\;z\;(\lambda x.x)) \\
  &)\;(\lambda x.\lambda y.x)
\end{align*}

This is an example of its behavior:

\begin{align*}
  \mathsf{PRED}\;\mathsf{2} &\\
  &\equiv
    \colorboxed{red}{
      (\lambda n.\lambda s.\lambda z.
        (
          n\;
          (\lambda p.\lambda f.
            f\;(p\;(\lambda x.\lambda y. y\;x))\;s
          )\;
          (\lambda f. f\;z\;(\lambda x.x))
      )\;(\lambda x.\lambda y.x)
    )}\;
    \colorboxed{blue}{
      (\lambda s.\lambda z.s\;(s\;z)) 
    } \\
  &\rightsquigarrow_\beta
    \lambda s.\lambda z.
      (
        \colorboxed{red}{
          (\lambda s.\lambda z.s\;(s\;z))
        }\;
        \colorboxed{blue}{
          (\lambda p.\lambda f.
            f\;(p\;(\lambda x.\lambda y. y\;x))\;s
          )
        }\;
        \colorboxed{blue}{
          (\lambda f. f\;z\;(\lambda x.x))
        }
      )\;(\lambda x.\lambda y.x) \\
  &\rightsquigarrow_\beta^2
    \lambda s.\lambda z.
      (
        (\lambda p.\lambda f.
          f\;(p\;(\lambda x.\lambda y. y\;x))\;s
        )\;
        (
          \colorboxed{red}{
            (\lambda p.\lambda f.
              f\;(p\;(\lambda x.\lambda y. y\;x))\;s
            )
          }\;
          \colorboxed{blue}{
            (\lambda f. f\;z\;(\lambda x.x))
          }
        )
      )\;
      (\lambda x.\lambda y.x) \\
  &\rightsquigarrow_\beta
    \lambda s.\lambda z.
      (
        \colorboxed{red}{
          (\lambda p.\lambda f.
            f\;(p\;(\lambda x.\lambda y. y\;x))\;s
          )
        }\;
        \colorboxed{blue}{
          (\lambda f.
            f\;
            (
              (\lambda f. f\;z\;(\lambda x.x))\;
              (\lambda x.\lambda y. y\;x)
            )\;
            s
          )
        }
      )\;
      (\lambda x.\lambda y.x) \\
  &\rightsquigarrow_\beta
    \lambda s.\lambda z.
      \colorboxed{red}{(
        \lambda f.
          f\;
          (
            (\lambda f.
              f\;
              (
                (\lambda f. 
                  f\;z\;(\lambda x.x)
                )\;
                (\lambda x.\lambda y. y\;x)
              )\;s
            )\;
            (\lambda x.\lambda y. y\;x)
          )\;s
      )}\;
      \colorboxed{blue}{(\lambda x.\lambda y.x)} \\
  &\rightsquigarrow_\beta
    \lambda s.\lambda z.
      \colorboxed{red}{(\lambda x.\lambda y.x)}\;
      \colorboxed{blue}{(
        (\lambda f.
          f\;(
            (\lambda f. f\;z\;(\lambda x.x))\;
            (\lambda x.\lambda y. y\;x)
          )\;s
        )\;
        (\lambda x.\lambda y. y\;x)
      )}\;
      \colorboxed{blue}{s} \\
  &\rightsquigarrow_\beta^2
    \lambda s.\lambda z.
      \colorboxed{red}{(
        \lambda f.
          f\;(
            (\lambda f. f\;z\;(\lambda x.x))\;
            (\lambda x.\lambda y. y\;x)
          )\;s
      )}\;
      \colorboxed{blue}{(\lambda x.\lambda y. y\;x)} \\
  &\rightsquigarrow_\beta
    \lambda s.\lambda z.
      \colorboxed{red}{(\lambda x.\lambda y. y\;x)}\;
      \colorboxed{blue}{
        (
          (\lambda f. f\;z\;(\lambda x.x))\;
          (\lambda x.\lambda y. y\;x)
        )
      }\;
      \colorboxed{blue}{s} \\
  &\rightsquigarrow_\beta^2
    \lambda s.\lambda z.
      s\;
      (
        \colorboxed{red}{
          (\lambda f. f\;z\;(\lambda x.x))
        }\;
        \colorboxed{blue}{
          (\lambda x.\lambda y. y\;x)
        }
      ) \\
  &\rightsquigarrow_\beta
    \lambda s.\lambda z.
      s\;
      (
        \colorboxed{red}{
          (\lambda x.\lambda y. y\;x)
        }\;
        \colorboxed{blue}{z}\;
        \colorboxed{blue}{(\lambda x.x)}
      ) \\
  &\rightsquigarrow_\beta^2
    \lambda s.\lambda z.
      s\; 
      (\colorboxed{red}{(\lambda x.x)}\;
      \colorboxed{blue}{z})
      \\
  &\rightsquigarrow_\beta
    \lambda s.\lambda z.
      s\;z \\
  &\equiv \mathsf{1}
\end{align*}

Once we have defined the $\mathsf{PRED}$ term, the subtraction 
of two natural number terms $m$ and $n$, which respectively 
represent two natural numbers $a$ and $b$, can be defined by 
applying $b$ times $\mathsf{PRED}$ to $m$. We keep the 
$\mathsf{PRED}$ definition unexpanded due to the size of the term:

$$
  \mathsf{SUB} 
    \triangleq
      \lambda m.\lambda n.\lambda s.\lambda z.
        n\;\mathsf{PRED}\;m\;s\;z
$$

$n\;\mathsf{PRED}\;m$ reduces to the natural number term which 
applies its parameter $s$ $a$ times but avoiding $b$ of them, 
starting with $z$, so $s$ is applied $max(0, a - b)$ times.

Although $\mathsf{SUB}$ as defined here is a $\beta$-normal term, 
it is not an $\eta$-normal term, thus it can be expressed in a 
more concise way:

\begin{align*}
  \mathsf{SUB} &\rightsquigarrow_\eta^2
    \lambda m.\lambda n.
      n\;\mathsf{PRED}\;m
\end{align*}

\subsection{}\label{ex:10}

Imagine that the list $[a_1,a_2 ...a_n]$ is represented in the 
lambda calculus by the term

$$
\lambda f.\lambda x.f\;a_1\;(f\;a_2\;(...\;f\;a_n\;x\;...))
$$

Define:

\begin{enumerate}[a)]
  \item $\mathsf{NIL}$, the empty list constructor.
  
    \textbf{Solution:} the term which fits the stated encoding 
    when there are no elements is

    $$
      \mathsf{NIL} 
        \triangleq \lambda f.\lambda x. 
          x
    $$

  \item $\mathsf{APP}$, a function to concatenate two lists.
  
    \textbf{Solution:} this term can be defined as

    $$
      \mathsf{APP} 
        \triangleq \lambda l.\lambda m.\lambda f.\lambda x. 
          l\;f\;(m\;f\;x)
    $$

    because, if $l$ and $m$ represent the lists 
    $[a_1,a_2 ...a_n]$ and $[b_1,b_2 ...b_n]$, then 
    $\mathsf{APP}\;l\;m$ can be $\beta$-reduced to 

    $$
      \lambda f.\lambda x.f\;a_1\;(f\;a_2\;(...\;f\;a_n\;\Big(
        b_1\;(f\;b_2\;(...\;f\;b_n\;x\;...))
      \Big)\;...))
    $$

    which represents the list $[a_1,a_2 ...a_n, b_1, b_2 ...b_n]$.
  \item $\mathsf{HD}$, which returns the first element of a 
    nonempty list.

    \textbf{Solution:} this term can be defined as an abstraction 
    which, given a list $l$, provides as the actual parameter 
    an abstraction that returns its parameter:

    $$
      \mathsf{HD} 
        \triangleq \lambda l.
          l\;(\lambda x.\lambda y.x)\;nilhead
    $$

    This mentioned parameter is not used when the list is 
    $\mathsf{NIL}$ and the second one is used instead, which we 
    have provided as a free variable named $nilhead$ to be the 
    result of reducing $\mathsf{HD}\;\mathsf{NIL}$.

    These are two examples to show both cases:

    \begin{align*}
      \mathsf{HD}\;\mathsf{NIL} & \\ 
        \equiv& (
          \lambda l.
            l\;(\lambda x.\lambda y.x)\;nilhead
        )\;(\lambda f.\lambda x.x) \\ 
        \rightsquigarrow_\beta&
          (\lambda f.\lambda x.x)\;
          (\lambda x.\lambda y.x)\;
          nilhead \\ 
        \rightsquigarrow_\beta&
          (\lambda x.x)\;
          nilhead \\
        \rightsquigarrow_\beta&
          nilhead
    \end{align*}

    \begin{align*}
      \mathsf{HD}\;(\lambda f.\lambda x.f\;a\;x) & \\ 
        \equiv& (
          \lambda l.
            l\;(\lambda x.\lambda y.x)\;nilhead
        )\;(\lambda f.\lambda x.f\;a\;x) \\ 
        \rightsquigarrow_\beta&
          (\lambda f.\lambda x.f\;a\;x)\;
          (\lambda x.\lambda y.x)\;
          nilhead \\ 
        \rightsquigarrow_\beta&
          (\lambda x.(\lambda x.\lambda y.x)\;a\;x)\;
          nilhead \\ 
        \rightsquigarrow_\beta&
          (\lambda x.\lambda y.x)\;a\;nilhead\\ 
        \rightsquigarrow_\beta^2& a
    \end{align*}

  \item $\mathsf{ISEMPTY}$, a function to check whether a 
    list is empty.

    \textbf{Solution:} it can be defined as follows:

    $$
      \mathsf{ISEMPTY} 
        \triangleq \lambda l.
          l\;
          (\lambda h.\lambda t.\lambda x.\lambda y.y)\;
          (\lambda x.\lambda y.x)
    $$

    Given a list $l$, if it is the empty list, then it reduces to 
    the second parameter, which we have provided as the Church 
    encoding $\mathsf{TRUE}$. Otherwise, it will use the second 
    parameter which we have provided as an abstraction that 
    ignores two arguments and becomes into $\mathsf{FALSE}$.

    Again, these are two examples to show both different cases:

    \begin{align*}
      \mathsf{ISEMPTY}\;\mathsf{NIL} & \\ 
        \equiv& (
          \lambda l.
            l\;
            (\lambda h.\lambda t.\lambda x.\lambda y.y)\;
            (\lambda x.\lambda y.x)
        )\;(\lambda f.\lambda x.x) \\ 
        \rightsquigarrow_\beta&
          (\lambda f.\lambda x.x)\;
          (\lambda h.\lambda t.\lambda x.\lambda y.y)\;
          (\lambda x.\lambda y.x) \\ 
        \rightsquigarrow_\beta&
          (\lambda x.x)\;
          (\lambda x.\lambda y.x) \\
        \rightsquigarrow_\beta&
          \lambda x.\lambda y.x \\ 
        \equiv& \mathsf{TRUE}
    \end{align*}

    \begin{align*}
      \mathsf{ISEMPTY}\;(\lambda f.\lambda x.f\;a\;x) & \\ 
        \equiv& (
          \lambda l.
            l\;
            (\lambda h.\lambda t.\lambda x.\lambda y.y)\;
            (\lambda x.\lambda y.x)
        )\;(\lambda f.\lambda x.f\;a\;x) \\ 
        \rightsquigarrow_\beta&
          (\lambda f.\lambda x.f\;a\;x)\;
          (\lambda h.\lambda t.\lambda x.\lambda y.y)\;
          (\lambda x.\lambda y.x) \\ 
        \rightsquigarrow_\beta&
          (\lambda x.(
            \lambda h.\lambda t.\lambda x.\lambda y.y)\;a\;x
          )\;
          (\lambda x.\lambda y.x) \\
        \rightsquigarrow_\beta&
          (\lambda h.\lambda t.\lambda x.\lambda y.y)\;a\;(
            \lambda x.\lambda y.x
          ) \\
        \rightsquigarrow_\beta^2&
          \lambda x.\lambda y.y \\ 
        \equiv& \mathsf{FALSE}
    \end{align*}

\end{enumerate}

% \section*{The $\lambda$-calculus in Haskell}

% \subsection{}\label{ex:11}

% Using Haskell, define functions boolChurch and boolUnchurch 
% which translate Haskell booleans into Church booleans and 
% vice versa. Use them to check the correctness of your 
% solutions to \ref{ex:6}.

% TODO

% \subsection{}\label{ex:12}

% Analogously to the previous exercise, define Haskell 
% functions to convert between Haskell and Church natural 
% numbers, and check the correctness of your solutions to 
% \ref{ex:7}.

% TODO

% \subsection{}\label{ex:13}

% Define a Haskell datatype to represent the abstract 
% syntax of the $\lambda$-calculus.

% TODO

% \subsection{}\label{ex:14}

% Based on the previous exercise, define a Haskell function 
% that obtains the free variables of a lambda term.

% TODO

% \subsection{}\label{ex:15}

% Based on the previous exercises, define a Haskell function 
% that implements capture avoiding substitution.

% TODO

% \subsection{}\label{ex:16}

% Based on the previous exercises, define a Haskell function 
% that implements $\beta$-reduction (one step).

% TODO

% \subsection{}\label{ex:17}

% Based on the previous exercises, define a Haskell function 
% that reduces a lambda term into $\beta$-normal form when 
% possible.

% TODO

\end{document}
