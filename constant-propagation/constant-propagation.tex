\documentclass{article}
\usepackage[english]{babel}
\usepackage{fancyvrb}
\usepackage{stmaryrd}
\usepackage{amsthm}
\usepackage{amsfonts}
\usepackage{amssymb}
\usepackage{amsmath}
\usepackage{hyperref}

\newtheorem{definition}{Definition}
\newtheorem{proposition}{Proposition}

\title{Constant propagation analysis by abstract interpretation}
\author{Adrián Enríquez Ballester}

\begin{document}
\maketitle

A \textit{constant propagation analysis} tries to determine, at 
each program point, whether a variable will always have the same 
value at that point and, in case it does, it tells us which value.

For example, given the following program:

\begin{Verbatim}
z := 4;
n := 0;
x := 0;
while x <= 10 do
    v := z * z;
    n := x + v;
    x := x + 1;
end
\end{Verbatim}

We know that, after the assignment \Verb|v := z * z|, the variable
\Verb|z| always contains the value 4, since once \Verb|z| is 
assigned to at the beginning of the program, it is no longer 
modified. We also know that \Verb|v| always contains 16 after that
assignment, since it depends only on \Verb|z|. Neither \Verb|n| 
nor \Verb|x| are constant after that assignment, since their value
changes at each iteration of the loop.

We are going to define an abstract interpreter for knowing whether
the result of an expression \textit{must} be constant at runtime 
and prove its correctness in terms of an interval analysis that 
has been previously studied.

\begin{definition}[
  The \textbf{Interval} lattice
]\label{def:ilattice}
  The \textbf{Interval} lattice is defined as

  $$
    \textbf{Interval} = 
      \{\bot\} \cup 
      \{[a, b] \;
        | \; a \in \mathbb{Z} \cup \{-\infty\}
        , \; b \in \mathbb{Z} \cup \{+\infty\}
        , \; a \leq b 
      \}
  $$

  with the order relation

  \begin{align*}
    \bot &\sqsubseteq int \;\;\; 
      \forall int \in \textbf{Interval} \\ 
    [a_1, b_1] &\sqsubseteq [a_2, b_2] 
      \iff a_2 \leq a_1 \land b_1 \leq b_2 \\
        &\forall a_1, a_2 \in \mathbb{Z} \cup \{-\infty\}, 
      \; \forall b_1, b_2 \in \mathbb{Z} \cup \{+\infty\}
  \end{align*}
\end{definition}

We already know that \textbf{Interval} is a lattice where 
$\bot$ represents the empty set, $\top$ is $[-\infty, +\infty]$
and, for every pair of intervals, its least upper bound is the 
smallest interval containing both and its greatest lower bound 
is their intersection.

It can be used to represent the possible values that a variable 
may take at each program point and, although we already know an 
abstract analysis which relates this lattice as an abstract domain 
with the collecting semantics as a concrete domain, in this case
it will be used as our concrete domain.

\begin{definition}[
  The $\mathbb{Z}_\bot^\top$ lattice
]\label{def:clattice}
  The set $\mathbb{Z}_\bot^\top = \mathbb{Z} \cup \{\top, \bot\}$
  is a lattice with the order relation
  
  \begin{align*}
    t &\sqsubseteq \top \;\;\; 
      \forall t \in \mathbb{Z}_\bot^\top \\ 
    \bot &\sqsubseteq t \;\;\; 
      \forall t \in \mathbb{Z}_\bot^\top \\ 
    t &\sqsubseteq t \;\;\; 
      \forall t \in \mathbb{Z}_\bot^\top 
  \end{align*}

  where the respective least upper bound and greatest lower bound
  of every pair $t_1, t_2 \in \mathbb{Z}_\bot^\top$ are

  $$
    t_1 \sqcup t_2 = 
    \begin{cases}
      t_1 & \text{if } t_1 = t_2 \\
      \top & \text{otherwise}
    \end{cases}
  $$

  $$
  t_1 \sqcap t_2 = 
  \begin{cases}
    t_1 & \text{if } t_1 = t_2 \\
    \bot & \text{otherwise}
  \end{cases}
  $$
\end{definition}

This will be our abstract domain, which represents the constant 
value that a variable contains if it is known and $\top$ otherwise.
$\bot$ is included in order to construct a lattice, but we are 
not going to attach any special meaning to it.

\begin{proposition}[
  A Galois connection between \textbf{Interval}
  and $\mathbb{Z}_\bot^\top$
]\label{prop:sgal}

  The quadruple $
    (\textbf{Interval}, \alpha, \gamma, \mathbb{Z}_\bot^\top)
  $ is a Galois connection with the abstraction function

  \begin{align*}
    \alpha(\bot) &= \bot \\
    \alpha([a, b]) &= 
      \begin{cases}
        a & \text{if } a = b \\
        \top & \text{otherwise}
      \end{cases}
    \;\;\; \forall a \in \mathbb{Z} \cup \{-\infty\}, 
    \; \forall b \in \mathbb{Z} \cup \{+\infty\}
  \end{align*}

  and the concretization function

  \begin{align*}
    \gamma(\bot) &= \bot \\
    \gamma(\top) &= \top \\
    \gamma(k) & = [k, k] \;\;\; \forall k \in \mathbb{Z}
  \end{align*}
\end{proposition}

\begin{proof}
  We have that $\textbf{Interval}$ and $\mathbb{Z}_\bot^\top$ are 
  lattices with the respective order relations stated in the 
  Definitions \ref{def:ilattice} and \ref{def:clattice}. Let us 
  check the required properties for $\alpha$ and $\gamma$ (i.e. 
  both functions are monotonically increasing, 
  $\gamma \circ \alpha \sqsupseteq id$ and 
  $\alpha \circ \gamma \sqsubseteq id$).

  The monotonicity of $\alpha$ can be studied by case distinction 
  on its domain:

  \begin{itemize}
    \item $\bot \sqsubseteq int 
      \; \forall int \in \textbf{Interval}$ and, as $\alpha(\bot) = 
      \bot$, then $\alpha(\bot) \sqsubseteq \alpha(int)$.
    
    \item If $[a, b] \sqsubseteq int$ for some $int \in 
      \textbf{Interval}$, $a \in \mathbb{Z} \cup \{-\infty\}$ and 
      $b \in \mathbb{Z} \cup \{+\infty\}$ with $a \neq b$, then 
      $int$ must be $[c, d]$ for some $c \in \mathbb{Z} \cup 
      \{-\infty\}$ and $d \in \mathbb{Z} \cup \{+\infty\}$
      with $c \neq d$. As $\alpha([a, b]) = \top$ and $\alpha(int) =
      \alpha([c, d]) = \top$, it is satisfied that $\alpha([a, b]) 
      \sqsubseteq \alpha(int)$.

    \item If $[a, a] \sqsubseteq int$ for some $int \in 
      \textbf{Interval}$ and $a \in \mathbb{Z}$, then $int$ by must 
      be as one of the following cases:
      
      \begin{itemize}
        \item[$\bullet$] $int = [a, a]$, so $\alpha([a, a]) = 
          \alpha(int) = a$
          
        \item[$\bullet$] $int = [b, c]$ for some $b \in \mathbb{Z} 
          \cup \{-\infty\}$ and $c \in \mathbb{Z} \cup \{+\infty\}$ 
          with $b \neq c$. As $\alpha([a, a]) = a$, $\alpha([b, c]) =
          \top$ and $a \sqsubseteq \top$, it follows that 
          $\alpha([a, a]) \sqsubseteq \alpha(int) = \top$.
        \end{itemize}
  \end{itemize}

  The monotonicity of $\gamma$ can likewise be studied by case 
  distinction:

  \begin{itemize}
    \item $\bot \sqsubseteq k \; 
      \forall k \in \mathbb{Z}_\bot^\top$ and, as $\gamma(\bot) =
      \bot$, then $\gamma(\bot) \sqsubseteq \gamma(k)$.
    
    \item If $\top \sqsubseteq k$ for some $k \in 
      \mathbb{Z}_\bot^\top$, then $k$ must also be $\top$,
      so $\gamma(\top) = \gamma(k) = \top$.

    \item If $a \sqsubseteq k$ for some $a \in \mathbb{Z}$ and
      $k \in \mathbb{Z}_\bot^\top$, then $k$ by must be as one 
      of the following cases:
      
      \begin{itemize}
        \item[$\bullet$] $k = a$, so $\gamma(a) = \gamma(k) = 
          [a, a]$.
          
        \item[$\bullet$] $k = \top$, so $\gamma(a) \sqsubseteq 
          \gamma(k) = \top$.
        \end{itemize}
  \end{itemize}

  With the same explicit technique, we can check that $\gamma 
  \circ \alpha \sqsupseteq id$:

  \begin{itemize}
    \item $\alpha(\gamma(\bot)) = \alpha(\bot) = \bot$.
    
    \item $\alpha(\gamma([a, a])) = \alpha(a) = [a, a] \; 
      \forall a \in \mathbb{Z}$.

    \item $\alpha(\gamma([b, c])) = \alpha(\top) = \top \; 
      \forall a \in \mathbb{Z} 
      \; \forall b \in \mathbb{Z} \cup \{-\infty\}$ and $\forall 
      c \in \mathbb{Z} \cup 
      \{+\infty\}$ with $b \neq c$. This satisfies the property 
      because $id([b, c]) = [b, c] \sqsubseteq \top$.
  \end{itemize}

  And, finally, it remains to be checked that $\alpha \circ 
  \gamma \sqsubseteq id$:

  \begin{itemize}
    \item $\gamma(\alpha(\bot)) = \gamma(\bot) = \bot$.
    
    \item $\gamma(\alpha(\top)) = \gamma(\top) = \top$.

    \item $\gamma(\alpha(a)) = \gamma([a, a]) = a \; 
      \forall a \in \mathbb{Z}$.
  \end{itemize}
\end{proof}

Note that, as $\alpha \circ \gamma = id$, this defined Galois 
connection is also a Galois insertion. 

These lattices and their connection would be useful if we were only 
interested in the analysis of a single program variable, so we are
going to extend it for being able to keep track of multiple 
variables at once in the same analysis.

\begin{definition}[
  The \textbf{State} and \textbf{State$^{\#}$} lattices
]\label{def:slattice}
  We define the following lattices as the corresponding lattice 
  extension to funcions of $\textbf{Interval}$ and 
    $\mathbb{Z}_\bot^\top$:

    \begin{align*}
      \textbf{State } &= 
        \textbf{Var } \rightarrow \textbf{ Interval} \\
      \textbf{State}^\# &= 
        \textbf{Var } \rightarrow \mathbb{Z}_\bot^\top \\
    \end{align*}

    where \textbf{Var} is the set of variable names.
\end{definition}

As we know, this kind of extension of a lattice to functions 
yields also a lattice with the order relation 
$f_1 \sqsubseteq f_2 \iff f_1(s) \sqsubseteq f_2(s) \; \forall s$ 
in the former lattice.

\begin{proposition}[
  A Galois connection between \textbf{State} 
  and \textbf{State$^{\#}$}
]\label{prop:mgal}
  The quadruple 
  $(\textbf{State}, \alpha', \gamma', \textbf{State}^\#)$ is a 
  Galois connection with the abstraction function

  $$
    \alpha'(f) = \alpha \circ f \;\;\; \forall f \in \textbf{State}
  $$

  and the concretization function

  $$
    \gamma'(f) = \gamma \circ f \;\;\; \forall f \in 
      \textbf{State}^\#
  $$

\end{proposition}

\begin{proof}
  We have seen the Galois connection between \textbf{Interval}
  and $\mathbb{Z}_\bot^\top$ in the Proposition \ref{prop:sgal} 
  and this is its extension to funcions with domain \textbf{Var}, 
  which we know that is a systematic method for deriving a new 
  Galois connection.
\end{proof}

Given this relation between the two lattices from the  
Definition \ref{def:slattice}, it can be derived an abstract
interpreter for the abstract domain in terms of another
interpreter for the concrete domain which can be used as
a reference to guarantee the correctness of new interpreters.

\begin{definition}[
  An abstract interpreter for expressions in the \textbf{State}
  lattice
]\label{def:iabs}
  Let $\llbracket e \rrbracket \textbf{State}
  \rightarrow \textbf{Interval}$ be the function 
  defined as follows:

  \begin{align*}
    \llbracket n \rrbracket &= \lambda \sigma. [n, n] \\
    \llbracket x \rrbracket &= \lambda \sigma. \sigma(x) \\
    \llbracket e_1 + e_2 \rrbracket &= 
      \lambda \sigma. 
        \Big(\llbracket e_1\rrbracket\sigma\Big) \oplus
        \Big(\llbracket e_2\rrbracket\sigma\Big) \\
    \llbracket e_1 - e_2 \rrbracket &= 
      \lambda \sigma. 
        \Big(\llbracket e_1\rrbracket\sigma\Big) \ominus
        \Big(\llbracket e_2\rrbracket\sigma\Big) \\
    \llbracket e_1 * e_2 \rrbracket &= 
      \lambda \sigma. 
        \Big(\llbracket e_1\rrbracket\sigma\Big) \otimes
        \Big(\llbracket e_2\rrbracket\sigma\Big)
  \end{align*}

  where $n \in \mathbb{Z}$, $x \in$ \textbf{Var}, $e_1, e_2 \in$ 
  \textbf{Exp} and the definitions of $\oplus, \ominus$ and 
  $\otimes$ are respectively

  \begin{align*}
    \bot \oplus int &= \bot \;\;\;
      \forall int \in \textbf{Interval} \\
    int \oplus \bot &= \bot \;\;\;
      \forall int \in \textbf{Interval} \\
    [a_1, b_1] \oplus [a_2, b_2] &= 
      [a_1 + a_2, b_1 + b_2] \\
        &\forall a_1, a_2 \in \mathbb{Z} \cup \{-\infty\}, 
        \; \forall b_1, b_2 \in \mathbb{Z} \cup \{+\infty\}
  \end{align*}

  \begin{align*}
    \bot \ominus int &= \bot \;\;\;
      \forall int \in \textbf{Interval} \\
    int \ominus \bot &= \bot \;\;\;
      \forall int \in \textbf{Interval} \\
    [a_1, b_1] \ominus [a_2, b_2] &= 
      [a_1 - b_2, b_1 - a_2] \\
        &\forall a_1, a_2 \in \mathbb{Z} \cup \{-\infty\}, 
        \; \forall b_1, b_2 \in \mathbb{Z} \cup \{+\infty\}
  \end{align*}

  \begin{align*}
    \bot \otimes int &= \bot \;\;\;
      \forall int \in \textbf{Interval} \\
    int \otimes \bot &= \bot \;\;\;
      \forall int \in \textbf{Interval} \\
    [a_1, b_1] \otimes [a_2, b_2] &= 
      [min(V), max(V)] \\
        &\forall a_1, a_2 \in \mathbb{Z} \cup \{-\infty\}, 
        \; \forall b_1, b_2 \in \mathbb{Z} \cup \{+\infty\} \\
        & \text{ and } V = \{ a_1 \cdot a_2, a_1 \cdot b_2, b_1 \cdot a_2, b_1 \cdot b_2 \}
  \end{align*}

  $\llbracket e \rrbracket$ is an abstract interpreter for 
  expressions in the \textbf{State} lattice.
\end{definition}

We assume to be known that this interpreter is correct and,
although $\alpha \circ \llbracket e \rrbracket \circ \gamma'$
is an interpreter in the \textbf{State$^\#$} lattice, it is in 
general not computable. We are going to directly define a candidate 
for this latter.

\begin{definition}[
  An abstract interpreter candidate for expressions in 
  the \textbf{State$^\#$} lattice
]\label{def:cabs}
  Let $\llbracket e \rrbracket^\# \textbf{State}^\# \rightarrow 
  \mathbb{Z}_\bot^\top$ be the function defined compositionally as
  
  \begin{align*}
    \llbracket n \rrbracket^\# &= 
      \lambda \sigma^\#. n \\ 
    \llbracket x \rrbracket^\# &= 
      \lambda \sigma^\#. \sigma^\#(x) \\ 
    \llbracket e_1 \text{ op } e_2 \rrbracket^\# &= 
      \lambda \sigma^\#.
      \begin{cases}
        \top & \text{if } 
          \llbracket e_1 \rrbracket^\# \sigma^\# 
            \in \{\top, \bot\} \\ 
          &\;\;\;\lor
          \llbracket e_2 \rrbracket^\# \sigma^\# 
            \in \{\top, \bot\} \\
          \llbracket e_1 \rrbracket^\# \sigma^\# 
            \text{ op$_\mathbb{Z}$ } \llbracket e_2 
            \rrbracket^\# \sigma^\# 
            & \text{otherwise} 
      \end{cases}
  \end{align*}

  where $n \in \mathbb{Z}$, $x \in$ \textbf{Var}, $e_1, e_2 \in$ 
  \textbf{Exp}, $\text{op} \in \{+, -, *\}$ and 
  $\text{op}_\mathbb{Z}$ denotes its corresponding counterpart
  operations in $\mathbb{Z}$.

  This function is an interpreter which determines whether the 
  result of an expression must be constant at runtime given an 
  abstract state.
\end{definition}

Finally, we are going to prove the correctness of the defined 
interpreter in terms of the one from the Definition \ref{def:iabs}.

\begin{proposition}[
  $\llbracket e \rrbracket^\#$ is a correct abstract interpreter
]

The abstract interpreter from the Definition \ref{def:cabs} 
is a correct abstract interpreter, thus it can be used in the 
constant propagation analysis. 
\end{proposition}

\begin{proof}
  Recall the abstract interpreter $\llbracket e \rrbracket$ 
  in the lattice of intervals from the definition \ref{def:iabs},
  whose correctness we assume to be known. Due to the Galois 
  connection of the Proposition \ref{prop:mgal}, 
  $\alpha \circ \llbracket e \rrbracket \circ \gamma'$ is 
  also a correct interpreter for the \textbf{State$^\#$} 
  lattice, so any other interpreter which always leads to less 
  specific results than it (i.e. greater results) must also be 
  considered to be correct.
  
  We are going to show that $\llbracket e \rrbracket^\# 
  \sqsupseteq \alpha \circ \llbracket e \rrbracket \circ \gamma'$
  by structural induction on the expression $e$.

  Given a literal expression $n \in \mathbb{Z}$, the defined 
  abstract interpreter yields the same literal

  $$
    \llbracket n \rrbracket^\# \sigma^\# = n
  $$

  and the derived one produces the same result

  $$
    (\alpha \circ \llbracket n \rrbracket \circ \gamma') \sigma^\# = 
      \alpha (\llbracket n \rrbracket (\gamma' (\sigma^\#))) =
      \alpha (\llbracket n \rrbracket (\gamma \circ \sigma^\#)) =
      \alpha ([n, n]) = 
      n
  $$

  so $\llbracket n \rrbracket^\# \sigma^\# = 
  (\alpha \circ \llbracket n \rrbracket \circ \gamma') \sigma^\#$
  
  When given a variable $x \in$ \textbf{Var}, the defined
  abstract interpreter takes its value from the abstract 
  state

  $$
    \llbracket x \rrbracket^\# \sigma^\# = \sigma^\#(x)
  $$

  and the derived interpreter does the same but 
  going and returning to the concrete domain by means of 
  the concretization and the abstraction functions 

  $$
    (\alpha \circ \llbracket x \rrbracket \circ \gamma') \sigma^\# = 
      \alpha (\llbracket x \rrbracket (\gamma' (\sigma^\#))) =
      \alpha (\llbracket x \rrbracket (\gamma \circ \sigma^\#)) =
      \alpha (\gamma(\sigma^\#(x)))
  $$

  As this is a Galois connection, we know that 
  $\alpha \circ \gamma \sqsubseteq id$, so
  $\alpha (\gamma(\sigma^\#(x))) \sqsubseteq \sigma^\#(x)$ and 
  then 
  $(\alpha \circ \llbracket x \rrbracket \circ \gamma') \sigma^\#
  \sqsubseteq \llbracket x \rrbracket^\# \sigma^\#
  $.

  The base expression cases has been proved, so we proceed 
  to the inductive steps. For an expression 
  $e_1 + e_2$ with $e_1, e_2 \in$ \textbf{Exp}, the defined
  abstract interpreter yields:

  $$
    \llbracket e_1 + e_2 \rrbracket^\# \sigma^\# = 
    \begin{cases}
      \top & \text{if } 
        \llbracket e_1 \rrbracket^\# \sigma^\# 
          \in \{\top, \bot\} \\ 
        &\;\;\;\lor
        \llbracket e_2 \rrbracket^\# \sigma^\# 
          \in \{\top, \bot\} \\
        \llbracket e_1 \rrbracket^\# \sigma^\# 
          + \llbracket e_2 \rrbracket^\# \sigma^\# 
          & \text{otherwise} 
    \end{cases}
  $$

  If $\llbracket e_1 \rrbracket^\# \sigma^\# 
    \in \{\top, \bot\}$ or 
  $\llbracket e_2 \rrbracket^\# \sigma^\# 
    \in \{\top, \bot\}$, then the property trivially holds.

  We are going to develop the derived abstract interpreter
  result in order to find more trivial situations:

  \begin{align*}
    (
      \alpha \circ \llbracket e_1 + e_2 \rrbracket \circ \gamma'
    )\sigma^\# &= 
    \alpha (
      \llbracket e_1 + e_2 \rrbracket (\gamma' (\sigma^\#))
    ) \\ &=
    \alpha (
      \llbracket e_1 + e_2 \rrbracket (\gamma \circ \sigma^\#)
    ) \\ &=
    \alpha \Bigg(
      \Big(\llbracket e_1\rrbracket(\gamma \circ \sigma^\#)\Big) 
      \oplus
      \Big(\llbracket e_2\rrbracket(\gamma \circ \sigma^\#)\Big)
    \Bigg)
  \end{align*}

  If $\Big(\llbracket e_1\rrbracket(\gamma \circ \sigma^\#)\Big)
  = \bot$ or 
  $\Big(\llbracket e_2\rrbracket(\gamma \circ \sigma^\#)\Big)
  = \bot$, then $\alpha(\bot) = \bot$ and the property also 
  holds.

  If it is not the case of any of the trivial ones, then 
  $\llbracket e_1 \rrbracket^\# \sigma^\# = a$ and 
  $\llbracket e_1 \rrbracket^\# \sigma^\# = b$ for some 
  $a, b \in \mathbb{Z}$. The induction hypothesis states that 
  
  \begin{align*}
    (
      \alpha \circ \llbracket e_1\rrbracket \circ \gamma'
    ) \sigma^\# = 
    \alpha(\llbracket e_1\rrbracket(\gamma \circ \sigma^\#)) 
      &\sqsubseteq \llbracket e_1 \rrbracket^\# \sigma^\# = a\\
    (
      \alpha \circ \llbracket e_2\rrbracket \circ \gamma'
    ) \sigma^\# = 
    \alpha(\llbracket e_2\rrbracket(\gamma \circ \sigma^\#))
      &\sqsubseteq \llbracket e_2 \rrbracket^\# \sigma^\# = b
  \end{align*}

  As $\alpha(\llbracket e_1\rrbracket(\gamma \circ \sigma^\#))$
  and $\alpha(\llbracket e_2\rrbracket(\gamma \circ \sigma^\#))$
  cannot be $\bot$ nor $\top$, they must also be respectively $a$
  and $b$, which requires that 
  $\llbracket e_1\rrbracket(\gamma \circ \sigma^\#) = [a,a]$ and 
  $\llbracket e_2\rrbracket(\gamma \circ \sigma^\#) = [b,b]$. This 
  means that 

  \begin{align*}
    \alpha \Bigg(
      \Big(\llbracket e_1\rrbracket(\gamma \circ \sigma^\#)\Big) 
      \oplus
      \Big(\llbracket e_2\rrbracket(\gamma \circ \sigma^\#)\Big)
    \Bigg) &= \alpha([a,a] \oplus [b, b]) \\
      &= \alpha([a + b,a + b]) \\
      &= a + b
  \end{align*}

  which is the same as 
  $\llbracket e_1 + e_2 \rrbracket^\# \sigma^\# = 
  \llbracket e_1 \rrbracket^\# \sigma^\# 
  + \llbracket e_2 \rrbracket^\# \sigma^\# 
  = a + b$.

  For an expression $e_1 - e_2$ with $e_1, e_2 \in$
  \textbf{Exp}, the reasoning is as the above one until 
  we have to develop the non trivial case, where we have:

  \begin{align*}
    \alpha \Bigg(
      \Big(\llbracket e_1\rrbracket(\gamma \circ \sigma^\#)\Big) 
      \ominus
      \Big(\llbracket e_2\rrbracket(\gamma \circ \sigma^\#)\Big)
    \Bigg) &= \alpha([a,a] \ominus [b, b]) \\
      &= \alpha([a - b,a - b]) \\
      &= a - b
  \end{align*}

  which is the same as 
  $\llbracket e_1 - e_2 \rrbracket^\# \sigma^\# = 
  \llbracket e_1 \rrbracket^\# \sigma^\# 
  - \llbracket e_2 \rrbracket^\# \sigma^\# 
  = a - b$.

  Finally, for an expression $e_1 * e_2$ with $e_1, e_2
  \in$ \textbf{Exp}, the reasoning is again as the composite 
  expressions above until we 
  have to develop the non trivial case

  \begin{align*}
    \alpha \Bigg(
      \Big(\llbracket e_1\rrbracket(\gamma \circ \sigma^\#)\Big) 
      \otimes
      \Big(\llbracket e_2\rrbracket(\gamma \circ \sigma^\#)\Big)
    \Bigg) &= \alpha([a,a] \otimes [b, b]) \\
      &= \alpha([min(\{a \cdot b\}),max(\{a \cdot b\})]) \\
      &= \alpha([a \cdot b,a \cdot b]) \\
      &= a \cdot b
  \end{align*}

  which is the same as 
  $\llbracket e_1 * e_2 \rrbracket^\# \sigma^\# = 
  \llbracket e_1 \rrbracket^\# \sigma^\# 
  \cdot \llbracket e_2 \rrbracket^\# \sigma^\# 
  = a \cdot b$.
\end{proof}

\end{document}
