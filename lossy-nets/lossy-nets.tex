\documentclass{article}
\usepackage[english]{babel}
\usepackage{amsthm}
\usepackage{amsfonts}
\usepackage{amssymb}
\usepackage{amsmath}
\usepackage{hyperref}

\newtheorem{definition}{Definition}
\newtheorem{lemma}{Lemma}
\newtheorem{theorem}{Theorem}
\newtheorem{corollary}{Corollary}

\newcommand{\overbar}[1]{
  \mkern 1.5mu\overline{
    \mkern-1.5mu#1\mkern-1.5mu
  }\mkern 1.5mu
}

\title{Lossy Petri nets}
\author{Adrián Enríquez Ballester}

\begin{document}
\maketitle

A \textit{Lossy} Petri net is a variant which can loose tokens 
at any time.

In this document we define them and study the decidability of 
its boundedness problem.

\begin{definition}[
  Lossy Petri net
]\label{def:lossy-net}
  Given a Petri net model with its enabling and firing rules, 
  its lossy variant is syntactically identical but with 
  the following semantics:

  A lossy transition $M_1 \stackrel{t}{\rightarrow}_l M_2$
  can happen whenever there are markings $M'_1$ and $M'_2$ such 
  that:

  \begin{itemize}
    \item $M_1 \geq M'_1$
    \item $M'_2 \geq M_2$
    \item $M'_1 \stackrel{t}{\rightarrow} M'_2$ according to the 
      usual enabling and firing rules of its counterpart model.
  \end{itemize}

  As we have used the symbol $\rightarrow_l$ to differentiate the 
  lossy semantics from the ones of its associated model, we also 
  use $[\rangle_l$ to differentiate the set of reachable markings 
  with this lossy semantics.
\end{definition}

\begin{lemma}[]\label{lem:regular-lossy}
  Given a Petri net model and a marking $M$, its reachable markings 
  are also reachable for its lossy variant: 

  $$[M\rangle \subseteq [M\rangle_l$$
\end{lemma}
\begin{proof}
  Let $M_2 \in [M\rangle$. As it is reachable, we have 
  $M \rightarrow^* M_2$ for some chain of 
  transitions. We are going to prove the property by induction
  on the length of this chain.

  If it has length one, then $M \stackrel{t}{\rightarrow} M_2$ for 
  some transition $t$ and we know that $M \in [M\rangle$. 
  These markings satisfy 
  $M \geq M$ and $M_2 \geq M_2$, so we also have 
  $M \stackrel{t}{\rightarrow}_l M_2$ and therefore $M_2 \in [M\rangle_l$.
  
  Proceeding with the inductive case, if it has length $k > 1$, 
  then we have 
  $$M \rightarrow^{k - 1} M_1 \stackrel{t}{\rightarrow} M_2$$
  for some transition $t$ and marking $M_1$. 
  
  As before, $M_1 \geq M_1$ and 
  $M_2 \geq M_2$, so we have
  $M_1 \stackrel{t}{\rightarrow}_l M_2$. We also have that 
  $M_1 \in [M\rangle_l$ by the induction hypothesis, so 
  $M_2 \in [M\rangle_l$.
\end{proof}

\begin{lemma}[]\label{lem:lossy-smaller}
  Given a Petri net model which satisfies at least the weak 
  monotonicity lemma and a marking $M$, and let 
  $M' \in [M\rangle_l$ be a lossy-reachable marking from $M$, then 
  $\exists \overbar{M'} \in [M\rangle$ such that $\overbar{M'} 
  \geq M'$. 
\end{lemma}
\begin{proof}
  As $M'$ is lossy-reachable from $M$, we have 
  $M \rightarrow^k_l M'$. We are going to prove the property
  by natural induction over $k$.

  If $k = 0$, then $M = M'$ and, as $M \in [M\rangle$,
  we trivially have $M' \in [M\rangle$ with $M' \geq M'$.

  Proceeding with the inductive step, if $k > 0$ we have 
  $$M \rightarrow_l^{k-1} M'' \stackrel{t}{\rightarrow}_l M'$$
  for some transition t and some lossy-reachable marking 
  $M'' \in [M\rangle_l$.

  On the one hand, this last lossy transition of the sequence
  requires that $M_1 \stackrel{t}{\rightarrow} M_2$ for some 
  markings $M_1$ and $M_2$ with $M'' \geq M_1$ and 
  $M_2 \geq M'$.

  On the other hand, by the induction hypothesis, there exists 
  a reachable marking $\overbar{M} \in [M\rangle$ such that 
  $\overline{M} \geq M''$.

  As this Petri net satisfies the weak monotonicity lemma,
  $\overbar{M} \geq M'' \geq M_1$ and 
  $M_1 \stackrel{t}{\rightarrow} M_2$,  we have 
  $\overbar{M} \stackrel{t}{\rightarrow} \overbar{M_2}$
  with $\overbar{M_2} \geq M_2 \geq M'$ and 
  $\overbar{M_2} \in [M\rangle$.
\end{proof}

\begin{theorem}[]\label{teo:bounded}
  Given a Petri net $(N, M_0)$ from a model which satisfies 
  at least the weak monotonicity lemma, it is bounded if and 
  only if it is bounded with lossy semantics.
\end{theorem}
\begin{proof}
  For the left to right implication, let $s$ be a place of $N$. 
  If the net is bounded with its regular model semantics, then 
  there exists a number $b \geq 0$ such that $M(s) \leq b$ 
  $\forall M \in [M_0\rangle$.

  By the Lemma \ref{lem:lossy-smaller}, for any lossy-reachable
  marking $M' \in [M_0\rangle_l$, there exists a reachable marking
  $\overline{M'} \in [M_0\rangle$ such that $\overline{M'} \geq M'$,
  so $b \geq \overline{M'}(s) \geq M'(s)$ and therefore the net is 
  lossy-bounded.

  For the right to left implication, let $s$ be a place of $N$. 
  If the net is lossy-bounded, then there exists a number 
  $b \geq 0$ such that $M(s) \leq b$ $\forall M \in [M_0\rangle_l$.

  By the Lemma \ref{lem:regular-lossy}, any reachable marking 
  $M \in [M_0\rangle$ is also lossy-reachable 
  (i.e. $M \in [M_0\rangle_l$), thus $M(s) \leq b$ and the net 
  is bounded.
\end{proof}

\begin{corollary}[]\label{cor:regular-lossy-bound}
By the Theorem \ref{teo:bounded}, as regular Petri nets satisfy 
the weak monotonicity lemma and boundedness is decidable for them,
boundedness is also decidable for regular Petri nets with lossy 
semantics.

For studying its boundedness, it suffices to study the property
without taking into account the lossy transitions by using the 
procedure that we know for regular Petri nets.
\end{corollary}

\begin{corollary}[]\label{cor:reset-lossy-bound}
By the Theorem \ref{teo:bounded}, as Petri nets with reset arcs 
satisfy the weak monotonicity lemma and boundedness is undecidable 
for them, boundedness is also undecidable for lossy Petri nets 
with reset arcs.
\end{corollary}

This theorem cannot be applied to Petri nets with inhibitor arcs 
because they do not satisfy any version of the monotonicity lemma.
However, we can find a close relation between lossy Petri nets with 
reset arcs and lossy Petri nets with inhibitor arcs.

\begin{lemma}[]\label{lem:iso}
  Given a Petri net with reset arcs $(R, M_0)$ and the Petri net 
  with inhibitor arcs $(I, M_0)$ which consists of replacing the
  reset arcs of $R$ by inhibitor arcs:

  $$
    M_1 \stackrel{t}{\rightarrow}_l M_2 \text{ for } R \iff 
    M_1 \stackrel{t}{\rightarrow}_l M_2 \text{ for } I
  $$

  As they have the same lossy transitions and initial marking, the 
  reachability graph of both Petri nets with lossy semantics is 
  isomorphic.
\end{lemma}
\begin{proof}
First of all, note that the set of transitions $T$ is the same for both 
nets. Let $t \in T$ be a transition and $M_1$ be a marking.

Due to the definition of $I$, there are reset arcs relating some place 
and $t$ for $R$ if and only if there are inhibitor arcs relating some place 
and $t$ for $I$.

On the one hand, if $t$ does not involve any reset arc, then it is trivial that 
$M_1$ lossy-enables $t$ leading to a marking $M_2$ for $R$ if an 
only if that is also the case for $I$, as the preconditions and 
postconditions of $t$ are the same for both nets.

On the other hand, if there is any reset arc $(s, t)$ for some place $s$, the enabling 
of $t$ for $R$ does not depend on $M_1(s)$ due to its regular enabling rule and the 
fact that if $M_1 \stackrel{t}{\rightarrow} M_2$ then 
$M_1 \stackrel{t}{\rightarrow}_l M_2$. As it is an
inhibitor arc for $I$, the enabling of $t$ also does not depend on
$M_1(s)$ with lossy semantics because we can always consider

$$
M'_1(p) = 
\begin{cases}
  0 &\text{if there is an inhibitor arc } (p, t) \\
  M_1(p) &\text{otherwise}
\end{cases}
$$

and, if $M'_1 \stackrel{t}{\rightarrow} M_2$, then
as $M_1 \geq M'_1$ and $M_2 \geq M_2$ we have
$M_1 \stackrel{t}{\rightarrow}_l M_2$ independently of
$M_1(s)$. For both nets, $M_2(s)$ will be $1$ or $0$ depending on whether the arc 
$(t, s)$ exists or not.
\end{proof}

\begin{corollary}[]\label{cor:inhib-lossy-bound}
  Boundedness is undecidable for lossy Petri nets with inhibitor 
  arcs.

  This is because, if it were decidable, then we could reduce
  the boundedness problem for lossy Petri nets with reset
  arcs to the boundedness problem for lossy Petri nets with 
  inhibitor arcs by using the transformation from the Lemma 
  \ref{lem:iso}, which preserves the reachability graph,
  and it would be a contradiction due to the 
  Corollary \ref{cor:reset-lossy-bound}.
\end{corollary}

\end{document}
