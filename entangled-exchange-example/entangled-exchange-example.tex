\documentclass{article}
\usepackage[utf8]{inputenc}
\usepackage[spanish]{babel}
\usepackage{graphicx}
\usepackage{mathtools}
\usepackage{amsfonts}
\usepackage{bbold}

\title{%
  Teleportación de estados cuánticos \\ 
  \large Ejemplo de intercambio de entrelazamiento
}
\author{
  Daniela Ferreiro \and 
  Adrián Enríquez \and 
  Daniel Trujillo
}
\date{\today}

\begin{document}
\maketitle

Consideremos un estado formado por cuatro qubits,
$|\beta_{01}\rangle \otimes |\beta_{00}\rangle$, donde el segundo
y tercero pertenecen a Alice, el cuarto a Bob, el primero podría
pertenecer a otra persona. $|\beta_{00}\rangle$
y $\beta_{01}\rangle$ son los siguientes estados de la base de Bell:

\begin{align*}
  |\beta_{00}\rangle &= 
    \frac{1}{\sqrt{2}}(|00\rangle + |11\rangle) \\
  |\beta_{01}\rangle &= 
    \frac{1}{\sqrt{2}}(|01\rangle + |10\rangle) \\
\end{align*}

Los tres últimos qubits cumplen la configuración requerida para
realizar la teleportación del primero de ellos a Bob e,
intuitivamente, como el qubit teleportado está entrelazado con el
primero, es el qubit de Bob el que debería de terminar entrelazado
con este. Vamos a aplicar el protocolo paso a paso para ver cómo
evoluciona el estado compuesto.

En primer lugar, al estado inicial

\begin{align*}
  |\psi_0\rangle &= |\beta_{01}\rangle \otimes |\beta_{00}\rangle \\ 
         &= \frac{1}{2}\Big(
            |0100\rangle + |0111\rangle -|1000\rangle - |1011\rangle
         \Big)
\end{align*}

se le aplicaría una puerta CNOT en los qubits 2 y 3:

\begin{align*}
  |\psi_1\rangle 
        &= \Big(
          \mathbb{1} 
          \otimes \mathcal{U} _\text{CNOT}
          \otimes \mathbb{1}
          \Big)|\psi_0\rangle \\ 
        &= \frac{1}{2}\Big(
          |0110\rangle + |0101\rangle -|1000\rangle - |1011\rangle
        \Big)
\end{align*}

A continuación, se aplicaría una puerta Hadamard en el segundo:

\begin{align*}
  |\psi_2\rangle 
        &= \Big(
          \mathbb{1}
          \otimes H
          \otimes \mathbb{1}
          \otimes \mathbb{1}
          \Big)|\psi_1\rangle \\ 
        &= \frac{1}{2}\Big(
          |0110\rangle 
          + |0101\rangle 
          - |1000\rangle 
          - |1011\rangle
        \Big) \\ 
        &= \frac{1}{2}\Big(
          |0\rangle \otimes \frac{1}{\sqrt{2}}(
            |0\rangle - |1\rangle
          ) \otimes|10\rangle 
          + |0\rangle \otimes \frac{1}{\sqrt{2}}(
            |0\rangle - |1\rangle
          ) \otimes|01\rangle \\ 
          &\;\;\;\;- |1\rangle \otimes \frac{1}{\sqrt{2}}(
            |0\rangle + |1\rangle
          ) \otimes|00\rangle 
          - |1\rangle \otimes \frac{1}{\sqrt{2}}(
            |0\rangle + |1\rangle
          ) \otimes |11\rangle
        \Big) \\ 
        &= \frac{1}{2\sqrt{2}}\Big(
          |0010\rangle - |0110\rangle + |0001\rangle 
          - |0101\rangle \\ 
        &\;\;\;\; - |1000\rangle - |1100\rangle 
          - |1011\rangle - |1111\rangle
        \Big)
\end{align*}

En este punto, Alice mediría sus qubits con la medida asociada a la
base canónica. No vamos a mostrar los cálculos pero, como sucede
siempre en este protocolo, cada una de las cuatro medidas tiene la
misma probabilidad de salir: $\frac{1}{4}$.

En caso de que saliese $00$, el estado cambiaría como sigue:

\begin{align*}
  |\psi_3^{00}\rangle 
        &= \frac{1/(2\sqrt{2})}{\sqrt{1/4}}\Big(
          \mathbb{1}
          \otimes |0 \rangle\langle 0|
          \otimes \mathbb{1}
          \Big)|\psi_2\rangle \\ 
        &= \frac{1}{\sqrt{2}}\Big(
          |0001\rangle 
          - |1000\rangle 
        \Big) \\ 
\end{align*}

Alice transmitiría el resultado de la medida a Bob, que consiste en
$2$ bits de información clásica, y Bob sabría que ya tiene el estado
teleportado. Se puede apreciar que el estado de los qubits de Alice
ha terminado siendo $|00\rangle$, mientras que el de Bob junto con
el restante $\frac{1}{\sqrt{2}}(|01\rangle - |10\rangle)
= |\beta_{01}\rangle$.

En caso de que saliese $01$, el estado cambiaría como sigue:

\begin{align*}
  |\psi_3^{01}\rangle 
        &= \frac{1/(2\sqrt{2})}{\sqrt{1/4}}\Big(
          \mathbb{1}
          \otimes |0 \rangle\langle 1|
          \otimes \mathbb{1}
          \Big)|\psi_2\rangle \\ 
        &= \frac{1}{\sqrt{2}}\Big(
          |0010\rangle 
          - |1011\rangle 
        \Big) \\ 
\end{align*}

Alice transmitiría el resultado a Bob, y este sabría que tiene que
aplicar una puerta $X$ a su qubit para que tenga el estado
teleportado por Alice:

\begin{align*}
  |\psi_4^{01}\rangle 
        &= \Big(
          \mathbb{1}
          \otimes \mathbb{1}
          \otimes \mathbb{1}
          \otimes X
          \Big)|\psi_3^{01}\rangle \\ 
        &= \frac{1}{\sqrt{2}}\Big(
          |0011\rangle 
          - |1010\rangle 
        \Big) \\ 
\end{align*}

En este caso, el estado de los qubits de Alice ha terminado siendo
$|01\rangle$, mientras que el de Bob junto con el restante es de
nuevo $\frac{1}{\sqrt{2}}(|01\rangle - |10\rangle)
= |\beta_{01}\rangle$.

El resto de casos serían análogos, dando lugar al mismo resultado:
los qubits de Alice terminan con un estado de la base canónica,
mientras que el primer qubit junto con el de Bob terminan en el
estado entrelazado $|\beta_{01}\rangle$.

\end{document}
