\documentclass{article}
\usepackage[english]{babel}
\usepackage[T1]{fontenc}
\usepackage{amsfonts}
\usepackage{amsmath}
\usepackage{mathtools}
\usepackage{fancyvrb}
\usepackage{fixltx2e}

\renewcommand{\thesubsection}{\thesection.\alph{subsection}}

\newcommand{\bit}[1]{\{0,1\}^{#1}}
\newcommand{\zo}{\{0,1\}}
\newcommand{\xor}{\oplus}

\title{Homework 3}
\author{Adrián Enríquez Ballester}

\begin{document}
\maketitle

\section{Collision resistant hash function?}

Let $H: \mathcal{M} \rightarrow \mathcal{T}$ be a collision resistant 
hash function. We define $\hat{H}(x||b) := H(x)||b$ where $b$ denotes
one bit and || denotes concatenation. 

We are going to prove that 
$\hat{H}$ is also a collision resistant hash function.
For that, let $\mathcal{A}_{\hat{H}}$ be an adversary for the 
collision resistance of $\hat{H}$. We can define an adversary $\mathcal{A}_H$
for the collision resistance of $H$ which also plays the role of 
oracle for $\mathcal{A}_{\hat{H}}$:

\begin{Verbatim}[commandchars=\\\{\},codes={\catcode`$=3\catcode`_=8}]
$\mathcal{A}_H$:
  (m\textsubscript{1}$||$b\textsubscript{1}, m\textsubscript{2}$||$b\textsubscript{2}) $\leftarrow$ $\mathcal{A}_{\hat{H}}$()
  O(m\textsubscript{1}, m\textsubscript{2})
\end{Verbatim}

It receives a pair of messages provided by 
$\mathcal{A}_{\hat{H}}$, removes its last bit and sends them to the
oracle for the collision resistance of $\mathcal{A}_H$. Note that
it is an efficient adversary, as it only removes the last bit of 
two messages.

Recall the collision resistance advantage of $\mathcal{A}_H$, which is 
the probability of it to win (i.e. the event 
$H(m_1) = H(m_2) \land m_1 \neq m_2$, where $(m_1, m_2)$ is the pair 
of messages provided by $\mathcal{A}_H$ to its oracle). We are going to 
analyze this probability in terms of the  advantage of 
$\mathcal{A}_{\hat{H}}$ itself.

On the one hand, if $\mathcal{A}_{\hat{H}}$ wins, then we have 

\begin{align*}
  \hat{H}(m_1||b_1) = \hat{H}(m_2||b_2) \\
  H(m_1)||b_1 = H(m_2)||b_2
\end{align*}

and $m_1||b_1 \neq m_2||b_2$. From the equality of the hashes
we can extract $b_1 = b_2$ and $H(m_1) = H(m_2)$, so
$m_1 \neq m_2$ and therefore $\mathcal{A}_H$ also wins.

On the other hand, if $\mathcal{A}_{\hat{H}}$ does not win, then

\begin{align*}
  \hat{H}(m_1||b_1) \neq \hat{H}(m_2||b_2) \\
  H(m_1)||b_1 \neq H(m_2)||b_2
\end{align*}

or $m_1||b_1 = m_2||b_2$. In this case, $\mathcal{A}_H$ can still
win if $H(m_1) = H(m_2)$ and $m_1 \neq m_2$, which is only possible
if $b_1 \neq b_2$.

Given the above cases, the probability of $\mathcal{A}_H$ to win is as 
follows:

\begin{align*}
\mathsf{CRadv}[\mathcal{A}_H, H] &= 
  \mathsf{CRadv}[\mathcal{A}_{\hat{H}}, \hat{H}] \cdot 1 + 
  (1 - \mathsf{CRadv}[\mathcal{A}_{\hat{H}}, \hat{H}]) \cdot \varepsilon \\
  &\geq \mathsf{CRadv}[\mathcal{A}_{\hat{H}}, \hat{H}]
\end{align*}

where

$$
  \varepsilon = 
    \mathsf{Pr}[m_1 \neq m_2 \land b_1 \neq b_2 \land H(m_1) = H(m_2)]
$$

If $\mathsf{CRadv}[\mathcal{A}_{\hat{H}}, \hat{H}]$ were not negligible,
then $\mathsf{CRadv}[\mathcal{A}_H, H]$ would also be non negligible,
which is not possible because $H$ is collision resistant. This implies
that $\hat{H}$ is also collision resistant.

\section{Signatures vs. encryption}

In some literature, digital signatures are sometimes described as an 
``inversion'' of public key encryption schemes, where we treat a message
$m$ as a ciphertext of the encryption scheme and decrypt it using $sk$
to produce a signature. To verify, the idea is then to encrypt and check 
whether cyphertext matches the message.

More concretely, let 
$\mathcal{E} = (\mathsf{G}_\mathcal{E}, \mathsf{E}, \mathsf{D})$ be a 
deterministic encryption scheme (i.e. algorithms $\mathsf{E}$ and 
$\mathsf{D}$ are deterministic). Then we can define a signature 
scheme $\Pi = (\mathsf{G}_{\Pi}, \mathsf{S}, \mathsf{V})$ as follows:

\begin{minipage}[t][3.1cm][t]{.3\textwidth}
\begin{Verbatim}[commandchars=\\\{\},codes={\catcode`$=3\catcode`_=8}]

G$_{\Pi}$(n):
  (sk, pk) $\leftarrow$ G$_{\mathcal{E}}$(n)
  return (sk, pk)
\end{Verbatim}
\end{minipage}
\hfill
\begin{minipage}[t]{.3\textwidth}
\begin{Verbatim}[commandchars=\\\{\},codes={\catcode`$=3\catcode`_=8}]

S(m):
  $\sigma$ $\leftarrow$ D\textsubscript{sk}(m)
  return $\sigma$
\end{Verbatim}
\end{minipage}
\hfill
\begin{minipage}[t]{.3\textwidth}
\begin{Verbatim}[commandchars=\\\{\},codes={\catcode`$=3\catcode`_=8}]

V(m, $\sigma$):
  c $\leftarrow$ E\textsubscript{pk}($\sigma$)
  if m = c:
    return 1 
  else:
    return 0
\end{Verbatim}
\end{minipage}

For this exercise, we consider a weak security notion of digital signature
scheme where the adversary does not get access to a signature oracle.

A signature scheme $\Pi = (\mathsf{G}_{\Pi}, \mathsf{S}, \mathsf{V})$ is 
unforgeable under a key only attack if for any PPT algorithm $\mathcal{A}$,
the probability that the experiment 
$\mathsf{Sig-forge}^{ka}_{\mathcal{A},\Pi}(n)$ evaluates to 1 is negligible,
where

\begin{Verbatim}[commandchars=\\\{\},codes={\catcode`$=3\catcode`_=8}]
Sig-forge$\textsuperscript{$ka$}_{\mathcal{A},\Pi}$(n):
  (sk, pk) $\leftarrow$ G$_{\Pi}$(n)
  (m, $\sigma$) $\leftarrow$ $\mathcal{A}$(pk)
  return V(m, $\sigma$)
\end{Verbatim}

The advantage of the adversary $\mathcal{A}$ over $\Pi$ under this 
security notion is the aforementioned probability:

$$
\mathsf{Sig-forge}^{ka}\mathsf{adv}[\mathcal{A}_{\Pi}, \Pi] = 
  Pr[\mathsf{Sig-forge}^{ka}_{\mathcal{A},\Pi}(n) = 1]
$$

Te following attack shows that $\Pi$ is not 
$\mathsf{Sig-forge}^{ka}$ secure:

\begin{Verbatim}[commandchars=\\\{\},codes={\catcode`$=3\catcode`_=8}]
$\mathcal{A}$:
  pk $\leftarrow$ O()
  $\sigma$ $\overset{\$}{\leftarrow}$ $\mathcal{S}$
  m $\leftarrow$ E\textsubscript{pk}($\sigma$)
  O(m, $\sigma$)
\end{Verbatim}

This adversary generates a random $\sigma$ from 
$\mathcal{S}$, namely the signature space of $\Pi$ 
and also the message space of $\mathcal{E}$. It then obtains
the message $m$ that matches this signature by performing
the same operation used by $\mathsf{V}$ to verify. This trivially
leads to $\mathsf{V}(m, \sigma)$ to output $1$, as 
$\mathsf{E}_{pk}(\sigma) = m$.

Note that this adversary is efficient because it generates 
a random signature and uses $\mathsf{E}$, which also has to be 
efficient. It always succeeds in creating a valid forgery, so its 
advantage is

$$
\mathsf{Sig-forge}^{ka}\mathsf{adv}[\mathcal{A}_{\Pi}, \Pi] = 1
$$

and then $\Pi$ is not $\mathsf{Sig-forge}^{ka}$ secure.

\section{$\Sigma$-protocol for Pedersen commitments}

Let $\mathcal{G}$ be a group with a prime number $q$ of elements
where the discrete logarithm problem is hard. The public parameters 
of the Pedersen commitments are two group elements $g_1$ and $g_2$.

To commit an element $m \in \mathbb{Z}_q$, the committer has to 
choose a random element $x \in \mathbb{Z}_q$ and compute the 
commitment as $c \coloneqq g_1^m \cdot g_2^x$.

We are going to define some $\Sigma$-protocols for proving 
statements within this setting and prove its completeness, 
special soundness and honest-verifier zero-knowledge.

\subsection{$\mathsf{POK}[\exists m \exists x : c = g_1^m \cdot g_2^x]$}

This protocol allows a prover to announce that he has committed
a message without revealing his choice, which cannot be modified
once the announcement has been made:

\begin{Verbatim}[commandchars=\\\{\},codes={\catcode`$=3\catcode`_=8}]
Prover(($\mathcal{G}$, g\textsubscript{1}, g\textsubscript{2}, c), (m, x)):
  (u\textsubscript{1}, u\textsubscript{2}) $\leftarrow$ $\mathbb{Z}$\textsubscript{q}\textsuperscript{2}
  a $\coloneqq$\textsubscript{q} g\textsubscript{1}\textsuperscript{u\textsubscript{1}} $\cdot$ g\textsubscript{2}\textsuperscript{u\textsubscript{2}}
  ch $\leftarrow$ Verifier(a)
  r\textsubscript{1} $\coloneqq$\textsubscript{q} u\textsubscript{1} + ch $\cdot$ m
  r\textsubscript{2} $\coloneqq$\textsubscript{q} u\textsubscript{2} + ch $\cdot$ x
  Verifier(r\textsubscript{1}, r\textsubscript{2})
\end{Verbatim}

\begin{Verbatim}[commandchars=\\\{\},codes={\catcode`$=3\catcode`_=8}]
Verifier(($\mathcal{G}$, g\textsubscript{1}, g\textsubscript{2}, c)):
  a $\leftarrow$ Prover()
  ch $\overset{\$}{\leftarrow}$ $\mathbb{Z}$\textsubscript{q}
  (r\textsubscript{1}, r\textsubscript{2}) $\leftarrow$ Prover(ch)
  if g\textsubscript{1}\textsuperscript{r\textsubscript{1}} $\cdot$ g\textsubscript{2}\textsuperscript{r\textsubscript{2}} =\textsubscript{q} a $\cdot$ c\textsuperscript{ch}:
    return 1
  else:
    return 0
\end{Verbatim}

The name $ch$ has been used to denote the challenge, although the 
letter $c$ is used more often, because one of the parameters 
already has the name $c$.

\subsubsection{Completeness}

Completeness holds because

\begin{align*}
  g_1^{r_1} \cdot g_2^{r_2} 
    &= g_1^{u_1 + ch \cdot m} \cdot g_2^{u_2 + ch \cdot x} \\
    &= g_1^{u_1} \cdot g_1^{ch \cdot m} \cdot g_2^{u_2} \cdot g_2^{ch \cdot x} \\
    &= a \cdot g_1^{ch \cdot m} \cdot g_2^{ch \cdot x} \\
    &= a \cdot (g_1^{m} \cdot g_2^{x})^{ch} \\
    &= a \cdot c^{ch} \\
\end{align*}

if the protocol has been followed properly.

\subsubsection{Special soundness}

Given two conversations with the same announcement where the
challenge is different,
$(a, ch, (r_1, r_2))$ and $(a, ch', (r_1', r_2'))$ with $ch \neq ch'$,
the witness can be recovered:

$$
\begin{cases}
  g_1^{r_1} \cdot g_2^{r_2} &= a \cdot c^{ch} \\
  g_1^{r'_1} \cdot g_2^{r'_2} &= a \cdot c^{ch'}
\end{cases}
$$

\begin{align*}
  g_1^{r_1 - r'_1} \cdot g_2^{r_2 - r'_2} &= c^{ch - ch'} \\
  g_1^{(r_1 - r'_1) \cdot (ch - ch')^{-1}} \cdot g_2^{(r_2 - r'_2) \cdot (ch - ch')^{-1}} &= c \\
  g_1^{(r_1 - r'_1) \cdot (ch - ch')^{-1}} \cdot g_2^{(r_2 - r'_2) \cdot (ch - ch')^{-1}} &= g_1^m \cdot g_2^x \\
\end{align*}

$$
\begin{cases}
  m &= (r_1 - r'_1) \cdot (ch - ch')^{-1} \\
  x &= (r_2 - r'_2) \cdot (ch - ch')^{-1}
\end{cases}
$$

\subsubsection{Honest-verifier zero-knowledgeness}

Given a challenge $ch$. Take $r_1$ and $r_2$ at random from $\mathbb{Z}_q$ 
and compute

$$a = g_1^{r_1} \cdot g_2^{r_2} \cdot c^{-ch}$$

in order to generate a simulated conversation $(a, ch, (r_1, r_2))$.

The probability of a simulated conversation to happen 
is $1/q^2$ due to the randomness of $r_1$ and $r_2$ over $\mathbb{Z}_q$. 
It is the same probability that for an honest conversation with 
a fixed challenge $ch$, due to the randomness of $u_1$ and $u_2$ 
over $\mathbb{Z}_q$.

\subsection{$\mathsf{POK}[\exists m \exists x : c = g_1^m \cdot g_2^x \land c' = g_3^x]$}

This protocol combines the previous one with a new statement where $g_3$ 
is also a public element of $\mathcal{G}$:

\begin{Verbatim}[commandchars=\\\{\},codes={\catcode`$=3\catcode`_=8}]
Prover(($\mathcal{G}$, g\textsubscript{1}, g\textsubscript{2}, g\textsubscript{3}, c, c'), (m, x)):
  (u\textsubscript{1}, u\textsubscript{2}, u\textsubscript{3}) $\leftarrow$ $\mathbb{Z}$\textsubscript{q}\textsuperscript{3}
  a\textsubscript{1} $\coloneqq$\textsubscript{q} g\textsubscript{1}\textsuperscript{u\textsubscript{1}} $\cdot$ g\textsubscript{2}\textsuperscript{u\textsubscript{2}}
  a\textsubscript{2} $\coloneqq$ g\textsubscript{3}\textsuperscript{u\textsubscript{3}}
  ch $\leftarrow$ Verifier(a\textsubscript{1}, a\textsubscript{2})
  r\textsubscript{1} $\coloneqq$\textsubscript{q} u\textsubscript{1} + ch $\cdot$ m
  r\textsubscript{2} $\coloneqq$\textsubscript{q} u\textsubscript{2} + ch $\cdot$ x
  r\textsubscript{3} $\coloneqq$\textsubscript{q} u\textsubscript{3} + ch $\cdot$ x
  Verifier(r\textsubscript{1}, r\textsubscript{2}, r\textsubscript{3})
\end{Verbatim}

\begin{Verbatim}[commandchars=\\\{\},codes={\catcode`$=3\catcode`_=8}]
Verifier(($\mathcal{G}$, g\textsubscript{1}, g\textsubscript{2}, g\textsubscript{3}, c, c')):
  (a\textsubscript{1}, a\textsubscript{2}) $\leftarrow$ Prover()
  ch $\overset{\$}{\leftarrow}$ $\mathbb{Z}$\textsubscript{q}
  (r\textsubscript{1}, r\textsubscript{2}, r\textsubscript{3}) $\leftarrow$ Prover(ch)
  if g\textsubscript{1}\textsuperscript{r\textsubscript{1}} $\cdot$ g\textsubscript{2}\textsuperscript{r\textsubscript{2}} =\textsubscript{q} a\textsubscript{1} $\cdot$ c\textsuperscript{ch} and g\textsubscript{3}\textsuperscript{r\textsubscript{3}} =\textsubscript{q} a\textsubscript{2} $\cdot$ c'\textsuperscript{ch}:
    return 1
  else:
    return 0
\end{Verbatim}

\subsubsection{Completeness}

Completeness holds because

\begin{align*}
  g_1^{r_1} \cdot g_2^{r_2} 
    &= g_1^{u_1 + ch \cdot m} \cdot g_2^{u_2 + ch \cdot x} \\
    &= g_1^{u_1} \cdot g_1^{ch \cdot m} \cdot g_2^{u_2} \cdot g_2^{ch \cdot x} \\
    &= a_1 \cdot g_1^{ch \cdot m} \cdot g_2^{ch \cdot x} \\
    &= a_1 \cdot (g_1^{m} \cdot g_2^{x})^{ch} \\
    &= a_1 \cdot c^{ch} \\
\end{align*}

and 

\begin{align*}
  g_3^{r_3}
    &= g_3^{u_3 + ch \cdot x} \\
    &= g_3^{u_3} \cdot g_3^{ch \cdot x} \\
    &= a_2 \cdot c'^{ch} \\
\end{align*}

if the protocol has been followed properly.

\subsubsection{Special soundness}

Given two conversations with the same announcement where the
challenge is different,
$((a_1, a_2), ch, (r_1, r_2, r_3))$ and $((a_1, a_2), ch', (r_1', r_2', r_3'))$ 
with $ch \neq ch'$, the witness can be recovered:

$$
\begin{cases}
  g_1^{r_1} \cdot g_2^{r_2} &= a_1 \cdot c^{ch} \\
  g_1^{r'_1} \cdot g_2^{r'_2} &= a_1 \cdot c^{ch'}
\end{cases}
$$

\begin{align*}
  g_1^{r_1 - r'_1} \cdot g_2^{r_2 - r'_2} &= c^{ch - ch'} \\
  g_1^{(r_1 - r'_1) \cdot (ch - ch')^{-1}} \cdot g_2^{(r_2 - r'_2) \cdot (ch - ch')^{-1}} &= c \\
  g_1^{(r_1 - r'_1) \cdot (ch - ch')^{-1}} \cdot g_2^{(r_2 - r'_2) \cdot (ch - ch')^{-1}} &= g_1^m \cdot g_2^x \\
\end{align*}

$$
\begin{cases}
  m &= (r_1 - r'_1) \cdot (ch - ch')^{-1} \\
  x &= (r_2 - r'_2) \cdot (ch - ch')^{-1}
\end{cases}
$$

But, in this case, we also have an alternative way to recover $x$:

$$
\begin{cases}
  g_3^{r_3} &= a_2 \cdot c'^{ch} \\
  g_3^{r'_3} &= a_2 \cdot c'^{ch'}
\end{cases}
$$

\begin{align*}
  g_3^{r_3 - r'_3} &= c'^{ch - ch'} \\
  g_3^{(r_3 - r'_3) \cdot (ch - ch')^{-1}} &= c' \\
  g_3^{(r_3 - r'_3) \cdot (ch - ch')^{-1}} &= g_3^x \\
\end{align*}

$$x = (r_3 - r'_3) \cdot (ch - ch')^{-1}$$

\subsubsection{Honest-verifier zero-knowledgeness}

Given a challenge $ch$. Take $r_1, r_2$ and $r_3$ at random from $\mathbb{Z}_q$ 
and compute

$$
\begin{cases}
  a_1 &= g_1^{r_1} \cdot g_2^{r_2} \cdot c^{-ch} \\
  a_2 &= g_3^{r_3} \cdot c'^{-ch}
\end{cases}
$$

in order to generate a simulated conversation $((a_1, a_2), ch, (r_1, r_2, r_3))$.

The probability of a simulated conversation to happen 
is $1/q^3$ due to the randomness of $r_1, r_2$ and $r_3$ over $\mathbb{Z}_q$.
It is the same probability that for an honest conversation with 
a fixed challenge $ch$, due to the randomness of $u_1, u_2$ and $u_3$ 
over $\mathbb{Z}_q$.

\end{document}
