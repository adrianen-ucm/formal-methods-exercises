\documentclass{article}
\usepackage[english]{babel}
\usepackage[shortlabels]{enumitem}
\usepackage{amsmath}
\usepackage{amsfonts}
\usepackage{amssymb}
\usepackage{stmaryrd}
\usepackage{fancyvrb}

\title{Assignment 4 on Program Semantics}
\author{Adrián Enríquez Ballester}

\begin{document}
\maketitle

Given the function $F: 
  \left(
    \textbf{State} \rightarrow \textbf{State}_\bot
  \right) 
  \rightarrow 
  \left(
    \textbf{State} \rightarrow \textbf{State}_\bot
  \right)
$ defined as follows:

$$
F(f) = cond
  \left(
    \mathcal{B}\llbracket n > 0\rrbracket,
    f \circ \mathcal{S} 
      \llbracket 
        x := 2 * x; \; n := n - 1
      \rrbracket,
    id
  \right)
$$

\begin{enumerate}[(a)]
  \item Give an explicit definition for 
    $F\left(\lambda\sigma.\bot\right)$,
    $F^2\left(\lambda\sigma.\bot\right)$
    and $F^3\left(\lambda\sigma.\bot\right)$.

    For the first application of $F$, as any function 
    right-composed with $\lambda\sigma.\bot$ is the same
    as $\lambda\sigma.\bot$, we have

    $$F\left(\lambda\sigma.\bot\right) 
      = \lambda\sigma.\begin{cases}
        \bot & \text{if } \sigma(n) > 0 \\
        \sigma & \text{if } \sigma(n) \leq 0
      \end{cases}
    $$

    For its second application,
    let $g: \textbf{State} \rightarrow \textbf{State}$ 
    be the function 

    \begin{align*}
      g &= S
      \llbracket
        x := 2 * x; \; n := n - 1 
      \rrbracket \\ &=
      \lambda \sigma. \sigma
        \left[
          x \mapsto 2 \cdot \sigma(x),
          n \mapsto \sigma(n) - 1
        \right]
    \end{align*}

    As it subtracts $1$ to $n$, by matching the 
    corresponding cases of $F(\lambda\sigma.\bot)$ 
    after applying $g$, 
    we have

    $$F(\lambda\sigma.\bot) \circ g =
      \lambda\sigma. \begin{cases}
        \bot & \text{if } \sigma(n) > 1 \\ 
        g(\sigma) & \text{if } \sigma(n) \leq 1
      \end{cases}
    $$

    which, once the case of $n \leq 0$ is absorbed by
    $cond$, becomes into

    $$F^2\left(\lambda\sigma.\bot\right) 
      = \lambda\sigma.\begin{cases}
        \bot & \text{if } \sigma(n) > 1 \\
        \sigma
          \left[
            x \mapsto 2 \cdot \sigma(x),
            n \mapsto 0
          \right] & \text{if } \sigma(n) = 1 \\
        \sigma & \text{if } \sigma(n) \leq 0
      \end{cases}
    $$

    In a similar way, as 

    $$F^2(\lambda\sigma.\bot) \circ g =
      \lambda\sigma. \begin{cases}
        \bot & \text{if } \sigma(n) > 2 \\ 
        g(g(\sigma)) & \text{if } \sigma(n) = 2 \\
        g(\sigma) & \text{if } \sigma(n) \leq 1 \\
      \end{cases}
    $$

    we reach the next power of $F$:

    $$F^3\left(\lambda\sigma.\bot\right) 
      = \lambda\sigma.\begin{cases}
        \bot & \text{if } \sigma(n) > 2 \\
        \sigma
          \left[
            x \mapsto 4 \cdot \sigma(x),
            n \mapsto 0
          \right] & \text{if } \sigma(n) = 2 \\
        \sigma
          \left[
            x \mapsto 2 \cdot \sigma(x),
            n \mapsto 0
          \right] & \text{if } \sigma(n) = 1 \\
        \sigma & \text{if } \sigma(n) \leq 0
      \end{cases}
    $$

  \item From the results above, conjecture a general defintion for 
    $F^i\left(\lambda\sigma.\bot\right)$ where $i \geq 1$.
    [Optional] Prove by induction on $i$ that your conjecture is 
    correct.

    Our conjecture is the following:

    $$F^i\left(\lambda\sigma.\bot\right) 
      = \lambda\sigma.\begin{cases}
        \bot & \text{if } \sigma(n) \geq i \\
        g^{\sigma(n)}(\sigma) & \text{if } 0 < \sigma(n) < i \\
        \sigma & \text{if } \sigma(n) \leq 0
      \end{cases}
    $$

    where $g^j (\sigma) = \sigma
      \left[
        x \mapsto 2^j \cdot \sigma(x),
        n \mapsto \sigma(n) - j
      \right]
    $. Note that when $j = \sigma(n)$ it yields $ 
      \sigma\left[
        x \mapsto 2^j \cdot \sigma(x),
        n \mapsto 0
      \right]
    $, but we have written it in a more general way for the
    ease of the following proof. 

    It has been shown to hold when $i = 1$ (i.e. recall 
    $F\left(\lambda\sigma.\bot\right)$ from the previous
    section). For proceeding with the inductive proof, 
    suppose that it holds for some $i$, then
    
    $$F^i\left(\lambda\sigma.\bot\right) \circ g
      = \lambda\sigma.\begin{cases}
        \bot & \text{if } \sigma(n) \geq i + 1 \\
        g^{\sigma(n) - 1}(g(\sigma)) & \text{if } 
          1 < \sigma(n) < i + 1 \\
        g(\sigma) & \text{if } \sigma(n) \leq 1
      \end{cases}
    $$

    and the case of $\sigma(n) = 1$ is 
    $g(\sigma) = g^0(g(\sigma)) 
    = g^{\sigma(n) - 1}(g(\sigma))$, thus it 
    can be merged with the $1 < \sigma(n) < i + 1$ case.
    By gathering all these results we obtain

    \begin{align*}
      F^{i + 1} \left(\lambda\sigma.\bot\right) &=
      \begin{cases}
        F^i(
          \lambda\sigma.\bot
        ) \circ g &\text{if } \sigma(n) > 0 \\
        \sigma &\text{if } \sigma(n) \leq 0
      \end{cases} \\ &= 
      \begin{cases}
        \bot & \text{if } \sigma(n) \geq i + 1 \\
        g^{\sigma(n)}(\sigma) & \text{if } 0 < \sigma(n) < i + 1 \\
        \sigma &\text{if } \sigma(n) \leq 0
      \end{cases}
    \end{align*}

  \item Give an explicit definition for 
    $\sqcup_i F^i\left(\lambda\sigma.\bot\right)$.

    $$\sqcup_i F^i\left(\lambda\sigma.\bot\right)
      = \lambda\sigma.\begin{cases}
        \sigma
          \left[
            x \mapsto 2^{\sigma(n)} \cdot \sigma(x),
            n \mapsto 0
          \right] & \text{if } \sigma(n) > 0 \\
        \sigma & \text{if } \sigma(n) \leq 0
      \end{cases}
    $$

    This function is greater than $
      F^i(\lambda\sigma.\bot) \;\forall i \geq 1
    $ in the sense that it has always less values of its domain
    mapped to $\bot$ (i.e. none at all) and it matches 
    the same image for those of 
    $F^i(\lambda\sigma.\bot)$ which are not mapped to $\bot$.

  \item Which is the least fixed point of $F$? Justify 
    you answer.

    The least fixed point of $F$ is the function 
    $\sqcup_i F^i$ stated in the previous section.

    This result follows by $
    \big(
      \textbf{State} \rightarrow \textbf{State}_\bot,
      \sqsubseteq
    \big)$ being a $ccpo$ for the order relation which we 
    are using, the function $F$ being monotonically increasing and 
    continuous, and a theorem which states that, under these conditions, 
    $\sqcup_i F^i$ is the least fixed point of $F$. 

  \item Given the above, compute the state resulting from
    the execution of the following program

    $$
      x := 1;
      \text{ \Verb|while| } n > 0 \text{ \Verb|do| }
        \left( 
          x := 2 * x;\;
          n := n - 1
        \right)
    $$

    under the initial state $\sigma = [n \mapsto 4]$.

    The denotational semantics for its sequenced 
    substatements are:

    \begin{align*}
      \mathcal{S}\llbracket x:= 1\rrbracket &=
        \lambda\sigma.\sigma\left[x \mapsto 1\right] \\ 
      \mathcal{S}
      \llbracket
        \text{ \Verb|while| } n > 0 \text{ \Verb|do| }
        \left( 
          x := 2 * x;\;
          n := n - 1
        \right)
      \rrbracket &= lfp \; F
    \end{align*}
    
    where the required least fixed point is precisely
    the one which we have been obtaining in this exercise: 

    $$
      lfp \; F = \lambda\sigma. 
          \begin{cases}
          \sigma
          \left[
            x \mapsto 2^{\sigma(n)} \cdot \sigma(x),
            n \mapsto 0
          \right] & \text{if } \sigma(n) > 0 \\
          \sigma & \text{if } \sigma(n) \leq 0
          \end{cases}
    $$

    The whole program semantics consist of its composition

    $$
      lfp \; F \circ \lambda\sigma.\sigma
      \left[
        x \mapsto 1
      \right]
    $$

    so let us apply it to the initial state in order to get the 
    final one: 

    \begin{align*}
      \big(lfp \; F \circ \lambda\sigma.\sigma
      \left[
        x \mapsto 1
      \right]\big)\big([n \mapsto 4]\big) &= 
      \big(lfp \; F\big)\big([
        x \mapsto 1,
        n \mapsto 4
      ]\big) \\
      &= [
        x \mapsto 2^4,
        n \mapsto 0
      ]
    \end{align*}
\end{enumerate}

\end{document}
